%%%%%%%%%%%%%%%%%%%%%%%%%%%%%%%%%%%%%%%%%%%%%%%%%%%%%%%%%%%%%%%%%
% Contents: The home chapter
% $Id: grisbi-manuel-home.tex, v 0.4 2002/10/27 Daniel Cartron
% $Id: grisbi-manuel-home.tex, v 0.5.0 2004/06/01 Loic Breilloux
% $Id: grisbi-manuel-home.tex, v 0.6.0 2011/11/17 Jean-Luc Duflot
% some of its content was in menus chapter:
% $Id: grisbi-manuel-menus.tex, v 0.5.0 2004/06/01 Loic Breilloux
% $Id: grisbi-manuel-home.tex, v 0.8.9 2012/04/27 Jean-Luc Duflot
% $Id: grisbi-manuel-home.tex, v 1.0 2014/02/12 Jean-Luc Duflot
%%%%%%%%%%%%%%%%%%%%%%%%%%%%%%%%%%%%%%%%%%%%%%%%%%%%%%%%%%%%%%%%%

\chapter{Entrée dans Grisbi\label{entrance}}

\section{Sélection d'un fichier\label{select-file}}

\vspacepdf{3mm}			% vertical space = 5 mm

Au lancement de l'application, Grisbi affiche la page qui vous permet de démarrer de différentes manières.

\vspacepdf{3mm}			% espace: 5 mm

Vous pouvez afficher la fenêtre de Grisbi en \indexword{plein écran}\index{affichage!plein écran}\index{plein écran!affichage} par la touche de fonction \keys{F11}, et revenir en arrière par la même touche.			% "!" separates the term from the subterm of the index entry

\vspacepdf{3mm}			% espace: 5 mm

\begin{figure}[htbp]			% h=here, t=top, b=bottom, p=page of float to force the figure here, not in a next page.
	\begin{center}					% image centrée
		\includegraphics[width=0.95\textwidth]{image/screenshot/home_start_grisbi}		% width=95% as wide as the current text
	\end{center}
	\caption{Fenêtre de démarrage}			% sous-titre/subtitle
	\label{home_start_grisbi}					% figure's ref., use for link in text with \refimage{}
\end{figure}


\vspacepdf{3mm}			% espace: 5 mm

En plus de la barre de menu, cette fenêtre affiche plusieurs pavés:

\begin{itemize}
	\item le pavé \frquote{Nouveau}, pour lancer l'assistant \frquote{Aide à la création d'un nouveau fichier de comptes};
	\item le pavé \frquote{Ouvrir}, pour afficher un gestionnaire de fichier avec lequel vous pourrez chercher un fichier de comptes existant dans votre ordinateur;
	\item le pavé \frquote{Importer}, pour lancer l'assistant \frquote{Importation des opérations par Grisbi};
	\item un ou plusieurs autres pavés, portant le nom de fichiers de comptes que Grisbi a déjà utilisés.
\end{itemize}

\vspacepdf{5mm}

\Note{}: les pavés portant les noms des fichiers de comptes que Grisbi a déjà utilisés ne sont présents que si ces fichiers existent; si vous voulez les enlever de cette page d'entrée, déplacez-les dans un autre répertoire, ou supprimez-les.

\vspacepdf{3mm}			% espace: 5 mm

En bas de page, un bandeau vous appelle à choisir une action en sélectionnant l'un de ces pavés.

\vspacepdf{3mm}			% espace: 5 mm

Si vous voulez juste découvrir le logiciel Grisbi pour avoir un aperçu de son aspect et de ses possibilités, vous pouvez à la place utiliser un fichier exemple comme celui présent sur le site de \lang{Sourceforge.net}\footnote{\urlSourceForgeDocumentation{}} dans le dossier \frquote{\textsf{examples}}.		% (voir la section \vref{new-example}).

\vspacepdf{5mm}			% espace: 5 mm

\Note{}: en cliquant simplement sur le fichier exemple téléchargé, Grisbi s'exécutera en affichant directement la fenêtre d'accueil\refimage{home_3.0} sans passer par la fenêtre de démarrage.


\section{Accueil\label{home}}

\vspacepdf{3mm}

En ouvrant un fichier de comptes, Grisbi affiche sa page d'accueil\refimage{home_3.0}.
C'est la page de démarrage du programme; on peut y accéder à tout moment en cliquant sur l'onglet \menus{Comptes}. 

\vspacepdf{3mm}

\begin{figure}[htbp]			% h=here, t=top, b=bottom, p=page of float to force the figure here, not in a next page.
\begin{center}
\includegraphics[width=1\textwidth]{image/screenshot/home_3.0.png}
\end{center}
\caption{Page d'accueil}		% sous-titre/subtitle
\label{home_3.0}
\end{figure}

\vspacepdf{3mm}

Grisbi affiche toutes ses pages de la même manière: comme n'importe quel logiciel, il affiche:% une barre de titre, une barre de menus qui donne accès à la plupart des fonctionnalités importantes de Grisbi, et aussi trois zones:

\begin{itemize}%[\large \textcircled{\small 2}]
	\item[\large\textcircled{\small 1}] la barre de titre, qui affiche un des noms suivants (voir \vref{home-details-homepage}):%disponible dans les préférences \vref{setup-dislpay-addresses-titles})
		\begin{itemize}
			\item[\textopenbullet] l'Entité comptable (par défaut),
			\item[\textopenbullet] le Titulaire du compte,
			\item[\textopenbullet] le Nom du fichier,
		\end{itemize}
	\item[\large\textcircled{\small 2}] la barre de menu, qui donne accès à la plupart des fonctionnalités importantes de Grisbi;
\end{itemize}
et aussi trois zones qui sont spécifiques à Grisbi;
\begin{itemize}%[3,4,5]
	\item[\large\textcircled{\small 3}] la barre d'information, sous la barre de menus,
	\item[\large\textcircled{\small 4}] le panneau de navigation,
	\item[\large\textcircled{\small 5}] le panneau des détails.
\end{itemize}


\section{Barre d'information\label{home-synthesis}}

La barre d'information présente le nom de l'onglet courant sélectionné, et peut afficher, complètement à droite, certains soldes en rapport avec ce qui est sélectionné dans le panneau des détails.

\vspacepdf{5mm}

\Note{}: la barre d'information, affichée par défaut, peut être cachée par une case à décocher dans les préférences \vref{setup-display-toolbars}.

\vspacepdf{3mm}

Elle permet, en cliquant successivement sur l'un des deux petits triangles à sa gauche, de sélectionner l'un des onglets affichés dans le panneau de navigation: \menus{Comptes}, \menus{Échéancier}, \menus{Tiers}, \menus{Simulateur de crédits}, \menus{Catégories}, \menus{Imputations budgétaires} et \menus{États}, et aussi l'un des sous-onglets des onglets \menus{Comptes} et \menus{États} s'ils sont déroulés dans le panneau de navigation.

\vspacepdf{5mm}

\Note{}: ces triangles sont remplacés, en fonction du thème de l'environnement de bureau ou du gestionnaire de fenêtres que vous utilisez, par d'autres caractères tels que +, -, >, <, etc.

\vspacepdf{3mm}
Le contenu de la sélection s'affiche dans le panneau des détails.

\vspacepdf{3mm}
Ces fonctionnalités peuvent être utilisées à la place de celles du panneau de navigation lorsque sa largeur est réduite à zéro et que l'on n'y a pas accès directement.


\section{Panneau de navigation\label{home-accounting}}

\begin{wrapfigure}{l}{0.33\textwidth}%50mm}
	\vspace{-\intextsep}				% space above the floating (minus intextsep=separation between float and text in text)
	\centering							% centering the floating figure in the "wrap"
	\includegraphics[width=0.28\textwidth]{image/screenshot/home_navigation}
	\vspace{-5pt}						% space between the floating and the caption below the floating
	\captionsetup{%						% for a change of options of the caption package
		format=plain,					% to avoid error message with labelsep option below, default=hang
		name=Fig.,						% rename label caption, default="Figure" (or "Table")
		justification=centering,		% centring of text and caption label 
		labelsep=newline				% define the separator between the label and the caption text
		}
	\caption{Panneau de navigation}		% \caption is mandatory to reference a figure in lof
	\vspace{-40pt}						% space below the caption
	\label{home_navigation}
\end{wrapfigure}

Le panneau de navigation affiche en caractères gras la liste des onglets:

	\begin{itemize}
		\item \menus{Comptes},
		\item \menus{Échéancier},
		\item \menus{Tiers},
		\item \menus{Simulateur de crédits},
		\item \menus{Catégories},
		\item \menus{Imputations budgétaires}
		\item \menus{États}.
	\end{itemize}
	%\menus{Comptes}, \menus{Échéancier}, \menus{Tiers}, \menus{Simulateur de crédits}, \menus{Catégories}, \menus{Imputations budgétaires} et \menus{États}.
En cliquant sur le petit triangle noir à gauche des onglets \menus{Comptes} ou \menus{États}, on peut dérouler ou enrouler la liste de leurs sous-onglets. Vous pouvez changer l'ordre des onglets et des sous-onglets en cliquant sur l'un d'eux et en le déplaçant plus haut ou plus bas dans la liste.

\vspacepdf{3mm}

\Note{}: ces triangles sont remplacés, en fonction du thème de l'environnement de bureau ou du gestionnaire de fenêtres que vous utilisez, par d'autres caractères tels que +, -, >, <, etc.

\vspacepdf{3mm}
Vous pouvez sélectionner un de ces onglets ou sous-onglets en cliquant sur son nom. Vous pouvez aussi déplacer la sélection dans cette liste d'onglets et de sous-onglets avec les touches du clavier \keys{\arrowkeyup}, \keys{\arrowkeydown}, \keys{Pg.Préc} ou \keys{Pg.Suiv}, ou avec la molette de la souris (option à cocher dans les préférences \vref{setup-display-toolbars}). 

\vspacepdf{3mm}

Le contenu de la sélection s'affiche dans le panneau des détails. 

\vspacepdf{3mm}
On peut réduire ou agrandir la largeur du panneau de navigation en cliquant sur la fine barre verticale entre ce panneau et le panneau des détails, et en la déplaçant. Si la largeur du panneau a été réduite à zéro, ou agrandie au maximum de la largeur de la fenêtre de Grisbi, il faut retrouver cette barre, respectivement complètement à gauche ou à droite de la fenêtre, et la faire glisser à la place désirée. 

\vspacepdf{3mm}
Des \indexword{menus contextuels}\index{menu contextuel}, accessibles par un clic-droit de souris, sont disponibles sur les éléments de ce panneau et proposent les fonctions suivantes:

\begin{itemize}
	\item sur \menus{Comptes}:
		\begin{itemize}
			\item \menus{Nouveau compte};
			\item sur un compte quelconque; 
				\begin{itemize}
					\item \menus{Nouveau compte},
					\item \menus{Supprimer ce compte},
				\end{itemize}
		\end{itemize}
	\item sur \menus{Tiers}:
		\begin{itemize}
			\item \menus{Nouveau tiers};
			\item \menus{Supprimer le tiers sélectionné};
			\item \menus{Éditer le tiers sélectionné};
			\item \menus{Gérer les tiers};
			\item \menus{Supprimer les tiers inutilisés};
		\end{itemize}
	\item sur \menus{Catégories}: 
		\begin{itemize}
			\item \menus{Nouvelle catégorie};
			\item \menus{Supprimer la catégorie sélectionnée};
			\item \menus{Éditer la catégorie sélectionnée};
			\item \menus{Importer un fichier de catégories (.csgb)};
			\item \menus{Exporter la liste des catégories (.csgb)};
		\end{itemize}
	\item sur \menus{Imputations budgétaires}:
		\begin{itemize}
			\item \menus{Nouvelle imputation budgétaire};
			\item \menus{Supprimer l'imputation sélectionnée};
			\item \menus{Éditer l'imputation sélectionnée};
			\item \menus{Importer un fichier d'imputations budgétaires (.isgb)};
			\item \menus{Exporter la liste des imputations budgétaires (.isgb)};
		\end{itemize}
	\item sur \menus{État}:
		\begin{itemize}
			\item \menus{Nouvel état};
			\item sur un état quelconque; 
				\begin{itemize}
					\item \menus{Nouvel état},
					\item \menus{Supprimer cet état}.
				\end{itemize}
		\end{itemize}
\end{itemize}

\section{Panneau des détails\label{home-details}}

Le panneau des détails affiche tous les détails sur les onglets ou sous-onglets sélectionnés par la barre d'information ou le panneau de navigation. C'est la zone de travail principale de Grisbi.

On peut réduire ou agrandir sa largeur en cliquant sur la fine barre verticale entre ce panneau et le panneau de navigation, et en la déplaçant. Si la largeur du panneau a été réduite à zéro ou agrandie au maximum de la largeur de la fenêtre de Grisbi, il faut retrouver cette barre, respectivement complètement à droite ou à gauche de la fenêtre, et la faire glisser à la place désirée. 

\begin{figure}[htbp]			% h=here, t=top, b=bottom, p=page of float to force the figure here, not in a next page.
	\begin{center}
		\includegraphics[width=1\textwidth]{image/screenshot/home_details.png}
	\end{center}
	\caption{Panneau des détails modifié}		% sous-titre/subtitle
	\label{home_details}
\end{figure}

\subsection{Affichage des détails dans la page d'accueil\label{home-details-homepage}}

En sélectionnant l'onglet \menus{Comptes}, le panneau des détails affiche:

\begin{itemize}
	\item en haut, dans un bandeau gris foncé:
		\begin{itemize}
	 		\item à gauche, l'icône de \menus{Grisbi} (qui peut être masquée, voir \vref{setup-display-logo-icon}),
	 		\item et à droite un \indexword{titre}\index{affichage!titre}\index{titre!page d'accueil} qui permet d'identifier sur quelle comptabilité vous travaillez actuellement, sous la forme \frquote{libellé - Grisbi}; vous pouvez définir ce libellé, parmi trois possibilités, dans le menu \menus{Éditer - Préférences} (voir le paragraphe \vref{setup-display-addresses-titles}, Adresses et titres):
				\begin{itemize}
			 		\item l'\menus{Entité comptable} (par defaut): c'est le nom du domaine de comptabilité sur lequel vous travaillez, par exemple \frquote{Ma comptabilité} ou \frquote{Association}, et que vous avez saisi à la création du fichier de comptes; vous pouvez le modifier ici dans le champ \menus{Nom de l'entité comptable}; cela peut être utile si vous gérez plusieurs \indexword{entités comptables}\index{entité comptable}, 
			 		\item le \menus{Titulaire du compte}: c'est le nom du titulaire (ou du propriétaire) du dernier compte consulté; si le titulaire n'est pas renseigné dans les propriétés du compte, Grisbi affiche le nom de ce compte,
			 		\item le \menus{Nom du fichier}: c'est le nom du fichier dans le répertoire courant, sous la forme \file{nom\_de\_votre\_fichier.gsb};
				\end{itemize}
		\end{itemize}
	\item dans la partie principale gris clair sous le bandeau:
		\begin{itemize} 
			\item pour chaque devise séparément, pour tous les comptes et \indexword{groupes de comptes}\index{groupe de comptes}, sous les libellés \menus{Solde rapproché} et \menus{Solde courant}:
				\begin{itemize}
					\item le solde des comptes bancaires et comptes de caisse, le solde partiel des groupes de comptes et leur solde final,
			\newline
			\Note{}: vous pouvez ajuster l'ordre d'affichage des soldes partiels des groupes de comptes (voir le paragraphe \vref{setup-general-home-partBalance}, \menus{Position dans la liste des soldes partiels}).			 
					\item le solde des comptes de passif et leur solde final,
					\item le solde des comptes d'actif et leur solde final;
				\end{itemize}
			\item les \indexword{alertes des opérations planifiées}\index{alerte!opération planifiée} à échéance ou clôturées, avec leurs date, libellé et montant, selon les choix faits dans le menu \menus{Éditer - Préférences} (voir la section \vref{setup-general-planned}, \menus{Échéancier});
			\item la liste des comptes dont le solde est passé sous le \menus{Solde minimal autorisé};
			\item la liste des comptes dont le solde est passé sous le \menus{Solde minimal voulu}.
	\end{itemize}
\end{itemize}

\vspacepdf{5mm}

\Note{}: pour les définitions du \menus{Solde minimal autorisé} et du \menus{Solde minimal voulu}, voir la section \vref{accounts-properties}, \menus{Propriétés d'un compte}.

\vspacepdf{3mm}

Les libellés des comptes s'affichent en \textcolor{black}{noir}; au passage du pointeur de la souris sur la ligne de l'un d'eux, cette couleur passe au \textcolor{gray}{gris}.

Un solde supérieur au \menus{Solde minimal voulu} s'affiche en \textcolor[RGB]{0,126,0}{vert}; au passage du pointeur sur sa ligne, cette couleur passe au \textcolor[RGB]{0,227,0}{vert clair}.

Un solde inférieur au \menus{Solde minimal voulu} et supérieur au \menus{Solde minimal autorisé} s'affiche en \textcolor[RGB]{230,155,0}{orange}; au passage du pointeur sur sa ligne, cette couleur passe à l'\textcolor[RGB]{255,200,0}{orange clair}.

Un solde inférieur au \menus{Solde minimal autorisé} s'affiche en \textcolor[RGB]{153,0,0}{rouge foncé}; au passage du pointeur sur sa ligne, cette couleur passe au \textcolor{red}{rouge}.

Au passage du pointeur de la souris sur la ligne d'un compte, tout changement de couleur indique que si l'on clique (droit ou gauche) avec la souris, ce compte s'affiche, comme s'il avait été sélectionné avec la barre d'information ou le panneau de navigation; il affiche alors la page qui contient la ligne d'opération affichant ce solde.

Un solde partiel, qui correspond à un groupe de comptes, s'affiche en \textcolor[RGB]{40,40,255}{bleu foncé} (comme dans la figure \vref{home_details}). S'il est négatif, il peut s'afficher en \textcolor[RGB]{153,0,0}{rouge foncé}, mais seulement si cela a été configuré ainsi (voir le paragraphe \vref{setup-general-home-partBalance}, \menus{Soldes partiels de la liste des comptes}). Une ligne de solde partiel ne change pas de couleur au passage du pointeur de la souris dessus, car on ne peut pas afficher les opérations d'un groupe de comptes.

% espace pour changement de thème
\vspacepdf{3mm}
Vous pouvez configurer certains aspects de l'affichage du panneau des détails:
\begin{itemize}
	\item dans le menu \menus{Éditer - Préférences}:%TODO verify sections/paragraphs with \vref{setup-xxx}
		\begin{itemize}
			\item \menus{\indexword{Géneralités}\index{généralités}}:
				\begin{itemize}
					\item \menus{Paramètres divers}, onglet \menus{Échéancier}\index{échéancier}: section \vref{setup-general-planned};
					\item \menus{Acceuil}:
						\begin{itemize}
							\item \menus{Calcul des soldes}: paragraphe \vref{setup-general-home-balance},
							\item \menus{Position dans la liste des soldes partiels}: paragraphe \vref{setup-general-home-partBalance};
						\end{itemize}
				\end{itemize}
			\item \menus{\indexword{Affichage}\index{affichage}}:
				\begin{itemize}
					\item \menus{Polices et Logo}\index{polices}\index{logo}: section \vref{setup-display-logo},
					\item \menus{Adresses et titres}\index{titre}\index{adresses}: section \vref{setup-display-addresses-titles};
				\end{itemize}
		\end{itemize}
	\item dans l'onglet \menus{Propriétés}\index{propriétés} de chaque compte à la rubrique \menus{Soldes}:
		\begin{itemize}
			 \item comptes sous le \menus{Solde minimal autorisé}\index{solde!minimal autorisé}: section \vref{accounts-properties},
			 \item comptes sous le \menus{Solde minimal voulu}\index{solde!minimal voulu}: section \vref{accounts-properties}.
		\end{itemize}
\end{itemize}

%En particulier, si vous trouvez une erreur d’orthographe dans cette page, vous pourrez la corriger: voir le paragraphe \vref{setup-general-home-final}, \menus{Pluriel de final}!

\section{Barre de menus\label{home-menus}}


Comme dans de nombreuses applications graphiques, la plupart des fonctionnalités importantes de Grisbi sont accessibles au moyen des menus de la \indexword{barre de menus}\index{barre de menus}. Nous détaillons ici leurs fonctionnalités.


\subsection{Menu \menus{Fichier}\label{home-menus-file}}

Ce menu comprend les fonctions suivantes:

\vspace{3mm}
\begin{itemize}[rightmargin=.6cm]
	\item \menus{Nouvelle fenêtre}: non fonctionnel (dans le futur peut-être?);	%TODO: to update
\end{itemize}

\noindent
\begin{minipage}{.7\linewidth}
	\begin{itemize}[rightmargin=.6cm]
		\item \menus{Nouveau fichier de comptes}: crée un nouveau fichier Grisbi d'\gls{extension} \file{.gsb}; le fichier courant est donc fermé et un nouveau fichier vide est créé avec un compte vide (raccourci-clavier \keys{Ctrl+N}), voir la section \vref{start-newfile}; à ne pas confondre avec la création d'un nouveau compte;	 
		\item \menus{Ouvrir}: ouvre votre gestionnaire de fichiers, permettant de rechercher, sélectionner et ouvrir un fichier de comptes d'\gls{extension} \file{.gsb} (raccourci-clavier \keys{Ctrl+O});
		\item \menus{Derniers fichiers}: affiche la liste des n derniers fichiers ouverts avec Grisbi (seulement s'il y en a eu plusieurs); ce nombre est configurable dans le menu \menus{Éditer - Préférences}, voir la section \vref{setup-general-files-manage}, \menus{Gestion des fichiers de compte};
	\end{itemize}
\end{minipage}
\hspace{10pt}	
\begin{minipage}{.3\linewidth}
	\vspace{-10pt}					% space above the floating (minus intextsep=separation between float and text in text)
	\centering						% centering the floating figure in the "wrap"
	\includegraphics[width=1\textwidth]{image/screenshot/home_menubar_file}
	\vspace{-10pt}					% space between the floating and the caption below the floating
	\captionsetup{
		type=figure,%				% define "figure" or "table" type, mandatory
		name=Fig.,%					% rename label caption, default="Figure" (or "Table")
		labelsep=newline}			% define the separator between the label and the caption text
	\caption{Menu \menus{Fichier}}	% \caption is mandatory to reference a figure in lof
	%\vspace{30pt}					% space below the caption
	\label{home_menubar_file}
\end{minipage}

\begin{itemize}
	\item \menus{Enregistrer}: enregistre le fichier de comptes en cours (raccourci-clavier \keys{Ctrl+S});
	\item \menus{Enregistrer sous}: ouvre un gestionnaire de fichiers pour enregistrer le fichier de comptes en cours avec le nom et à l'emplacement de votre choix; Grisbi vous propose par défaut le répertoire courant, le nom du fichier de comptes en cours, avec l'\gls{extension} \file{.gsb};
	\item \menus{Importer un fichier}: démarre l'assistant d'importation de fichiers d'un autre logiciel (raccourci-clavier \keys{Ctrl+I}); voir la section \vref{move-import-importinit};
	\item \menus{Exporter vers un fichier \gls{QIF}/\gls{CSV}}: démarre l'assistant d'exportation de fichiers de compte (raccourci-clavier \keys{Ctrl+E}); voir la section \vref{move-export};	
	\item \menus{Créer une archive}: démarre l'assistant de création d'archive; voir la section \vref{datamanagement-history-new};	
	\item \menus{Exporter une archive vers un fichier \gls{GSB}/\gls{QIF}/\gls{CSV}}: démarre l'assistant d'exportation d'archive; voir la section \vref{datamanagement-history-export};
	\item \menus{Déboguer le fichier de comptes}: démarre l'assistant de débogage de ce fichier, qui va vous aider à chercher des incohérences dans votre fichier de comptes; voir la section \vref{maintenance-file-debug};
	\item \menus{Rendre anonyme le fichier de comptes}: démarre l'assistant qui produit une copie anonymée (de manière irréversible) de votre fichier de comptes; ce fichier pourra être joint à un rapport de bogue; voir la section \vref{maintenance-file-anonymous};	
	\item \menus{Rendre anonyme le fichier \gls{QIF}}: démarre l'assistant qui produit une copie anonymée (de manière irréversible) de ce fichier; ce fichier pourra être joint à un rapport de bogue; voir la section \vref{maintenance-QIF-anonymous};	
	\item \menus{Mode de débogage}: met Grisbi en mode de débogage, qui crée un fichier-journal des évènements; voir la section \vref{maintenance-debug-mode}; 	
	\item \menus{Fermer}: ferme le fichier de comptes en cours; Grisbi vous propose de l'enregistrer si ce n'est déjà fait (raccourci-clavier \keys{Ctrl+W});
	\item \menus{Quitter}: ferme Grisbi; Grisbi vous propose auparavant d'enregistrer le fichier de comptes, si ce n'est pas déjà fait (raccourci-clavier \keys{Ctrl+Q}).
\end{itemize}


\subsection{Menu \menus{Éditer}\label{home-menus-edit}}

\Note{}: dans le menu \menus{Éditer}, certaines entrées ne sont actives que lorsqu'un compte ou une opération est sélectionné(e).

Ce menu comprend les fonctions suivantes:

\vspace{3mm}
\noindent
\begin{minipage}{.7\linewidth}
	\begin{itemize}[rightmargin=.6cm]
		\item \menus{Éditer l'opération}: permet la rectification d'une opération sélectionnée, voir la section \vref{transactions-modify}, \menus{Modification d'une opération};
		\item \menus{Nouvelle opération}: permet la création d'une nouvelle opération dans un compte (raccourci-clavier \keys{Ctrl+T}), voir la section \vref{transactions-new}, \menus{Saisie d'une nouvelle opération};
		\item \menus{Supprimer une opération}: supprime une opération sélectionnée, voir la section \vref{transactions-delete}, \menus{Suppression d'une opération};
	\end{itemize}
\end{minipage}
\hspace{10pt}	
\begin{minipage}{.3\linewidth}
	\vspace{-5pt}					% space above the floating (minus intextsep=separation between float and text in text)
	\centering						% centering the floating figure in the "wrap"
	\includegraphics[width=1\textwidth]{image/screenshot/home_menubar_edit}
	\vspace{-15pt}					% space between the floating and the caption below the floating
	\captionsetup{
		type=figure,%				% define "figure" or "table" type, mandatory
		name=Fig.,%					% rename label caption, default="Figure" (or "Table")
		labelsep=newline}			% define the separator between the label and the caption text
	\caption{Menu \menus{Éditer}}	% \caption is mandatory to reference a figure in lof
	%\vspace{30pt}					% space below the caption
	\label{home_menubar_edit}
\end{minipage}

\begin{itemize}	
\item \menus{Utiliser l'opération sélectionnée comme modèle}: crée la copie d'une opération sélectionnée, la date actuelle étant inscrite dans le formulaire de saisie, voir la section \vref{transactions-model}, \menus{Opération sélectionnée comme modèle};
\item \menus{Cloner l'opération}: crée une copie identique à l'opération sélectionnée et ouvre le formulaire de saisie, voir la section \vref{transactions-duplicate}, \menus{Clonage d'une opération};
\item \menus{Convertir en opération planifiée}: voir la section \vref{transactions-schedule}, \menus{Conversion d'une opération en opération planifiée};
\item \menus{Déplacer l'opération vers un autre compte}: déplace l'opération vers le compte sélectionné, voir la section \vref{transactions-move}, \menus{Déplacement d'une opération vers un autre compte};
\item \menus{Rechercher}:
	\begin{itemize}
 		\item ouvre la fenêtre de propriétés d'un état quand un onglet du panneau de navigation est sélectionné, voir le chapitre \vref{reports-creation}, \menus{Création d'un état};
		\item affiche la fenêtre de recherche quand un compte ou une opération est sélectionné, <<<voir le chapitre \vref{accounts-search}, \menus{Recherche alphanumérique}>>> A CRÉER; %TODO to create
	\end{itemize}
\item \menus{Nouveau compte}: démarre l'assistant de création d'un nouveau compte dans votre fichier Grisbi (raccourci-clavier \keys{Maj \shift+Ctrl+N}), voir la section \vref{accounts-new}, \menus{Création d'un nouveau compte};
\item \menus{Supprimer le compte courant}: efface le compte sélectionné de votre fichier Grisbi, voir la section \vref{accounts-delete}, \menus{Suppression d'un compte};
\item \menus{Préférences}: permet de configurer Grisbi (raccourci-clavier \keys{Maj \shift+Ctrl+P}); voir le chapitre \vref{setup}, \menus{Configuration de Grisbi}.
\end{itemize}


\subsection{Menu \menus{Affichage}\label{home-menus-display}}

\Note{}: dans le menu \menus{Affichage}, les entrées ne sont actives que lorsqu'un compte est sélectionné.

Ce menu comprend les fonctions suivantes:

\vspace{3mm}
\noindent
\begin{minipage}{.7\linewidth}
	\begin{itemize}[rightmargin=.6cm]
		\item \menus{Montrer le formulaire de saisie des opérations}: permet de développer le formulaire de saisie des opérations du compte sélectionné;
		\item \menus{Montrer les opérations rapprochées}: permet l'affichage des opérations rapprochées du compte sélectionné (raccourci-clavier \keys{Alt+R});
		\item \menus{Montrer les lignes d'archives}: affiche les lignes d'archives du compte sélectionné (raccourci-clavier \keys{Alt+L});
	\end{itemize}
\end{minipage}
\hspace{10pt}	
\begin{minipage}{.3\linewidth}
	%\vspace{-10pt}					% space above the floating (minus intextsep=separation between float and text in text)
	\centering						% centering the floating figure in the "wrap"
	\includegraphics[width=1\textwidth]{image/screenshot/home_menubar_view}
	\vspace{-20pt}					% space between the floating and the caption below the floating
	\captionsetup{
		type=figure,%				% define "figure" or "table" type, mandatory
		name=Fig.,%					% rename label caption, default="Figure" (or "Table")
		labelsep=newline}			% define the separator between the label and the caption text
	\caption{Menu \menus{Affichage}}	% \caption is mandatory to reference a figure in lof
	%\vspace{30pt}					% space below the caption
	\label{home_menubar_view}
\end{minipage} 
\begin{itemize}
	\item \menus{Montrer les \indexword{comptes clos}}\index{compte!clos}: affiche le(s) compte(s) clos et non supprimé(s), voir la section \vref{accounts-properties}, \menus{Propriétés d'un compte};
\begin{addmargin*}[-10pt]{0cm} 	% modify margin, * : left = right, [] = obligatory argument indentation, {} = optional argument left indentation 
	Les quatres fonctions suivantes permettent de configurer l'affichage des opérations du compte sélectionné:
\end{addmargin*}
	\item \menus{Montrer une ligne par opération};
	\item \menus{Montrer deux lignes par opération};
	\item \menus{Montrer trois lignes par opération};
	\item \menus{Montrer quatre lignes par opération};
\begin{addmargin*}[-10pt]{0cm} 	% modify margin, * : left = right, [] = obligatory argument indentation, {} = optional argument left indentation 
	Et enfin la dernière fonction du menu d'affichage:
\end{addmargin*}	
	\item \menus{Réinitialiser la largeur des colonnes}: permet de remettre les colonnes des listes d'opérations du compte sélectionné ou de l'échéancier à leur largeur d'origine.
\end{itemize}


\subsection{Menu \menus{Aide}\label{home-menus-help}}

La plupart des choix de ce menu donnent accès à des sites Web. Pour que ces accès fonctionnent, il faut avoir indiqué à Grisbi le logiciel de navigation (ou navigateur) que vous souhaitez utiliser, dans le menu \menus{Éditer - Préférences} (voir la section \vref{setup-general-programs}, \menus{Programmes}). Le menu \menus{Aide} comprend les choix suivants:

\vspace{3mm}
\noindent
\begin{minipage}{.7\linewidth}
	\begin{itemize}[rightmargin=.6cm]
		\item \menus{Manuel}: ouvre le \frquote{Manuel de Grisbi} dans votre navigateur ou dans votre lecteur \gls{PDF}\index{PDF} selon votre choix (voir \vref{setup-general-various-general-display}, \menus{Affichage de l'aide}) (raccourci-clavier \keys{F1});
		\item \menus{Démarrage rapide}: ouvre votre navigateur à la page \frquote{Démarrage Rapide de Grisbi};
		\item \menus{À propos}: affiche la fenêtre d'information sur l'application; vous y trouverez des détails sur la version, le lien vers le site de Grisbi, les remerciements (contributeurs au projet) et la licence d'utilisation;
	\end{itemize}
\end{minipage}
\hspace{10pt}	
\begin{minipage}{.3\linewidth}
	%\vspace{-15pt}					% space above the floating (minus intextsep=separation between float and text in text)
	\centering						% centering the floating figure in the "wrap"
	\includegraphics[width=1\textwidth]{image/screenshot/home_menubar_help}
	\vspace{-15pt}					% space between the floating and the caption below the floating
	\captionsetup{
		type=figure,%				% define "figure" or "table" type, mandatory
		name=Fig.,%					% rename label caption, default="Figure" (or "Table")
		labelsep=newline}			% define the separator between the label and the caption text
	\caption{Menu \menus{Aide}}	% \caption is mandatory to reference a figure in lof
	%\vspace{30pt}					% space below the caption
	\label{home_menubar_help}
\end{minipage} 

\begin{itemize}
	\item \menus{Site Web de Grisbi}; ouvre votre navigateur à la page du site de \lang{Grisbi}\footnote{\urlGrisbi{}};
	\item \menus{Signaler une anomalie}; ouvre votre navigateur à la page du \lang{traqueur de bogues de Grisbi}\footnote{\urlBugTracker{}} pour vous permettre de signaler un bogue que vous auriez découvert. Vous pouvez également suivre sur cette page l'évolution des corrections apportées aux bogues signalés;
	\item \menus{Astuce du jour}; ouvre une boîte de dialogue qui affiche une astuce d'utilisation, différente à chaque démarrage de Grisbi; vous pouvez y afficher successivement toutes les astuces, et choisir ou non l'affichage de l'astuce du jour au démarrage de Grisbi. Pour activer ou supprimer l'astuce du jour, vous pouvez cocher/décocher la case \frquote{\menus{Afficher l'astuce lors du prochain démarrage}}, voir le paragraphe \vref{setup-display-messages-trick}, \menus{Astuce du jour}.
\end{itemize}


\section{Raccourcis-clavier\label{home-shortcuts}}


Les raccourcis-clavier facilitent la saisie des données et la navigation dans les fenêtres de Grisbi, en évitant le recours systématique au déplacement et au clic de la souris. En utilisant ceux correspondant aux manipulations les plus courantes pour vous, vous améliorez votre \indexword{ergonomie}\index{ergonomie} en limitant les mouvements importants de vos bras.
 
Grisbi dispose d'un certain nombre de raccourcis-clavier, listés dans les préférences de Grisbi (voir \vref{setup-display-toolbars}), présentés ici selon différents thèmes, (voir aussi la section \vref{introduction-manual-conventions}, \menus{Conventions typographiques du présent manuel}).


\subsection{Application et fichiers}

\begin{itemize}
	\item créer un Nouveau fichier Grisbi: \keys{Ctrl+N}
	\item Ouvrir un fichier Grisbi: \keys{Ctrl+O}
	\item ajouter un Nouveau compte au fichier Grisbi: \keys{Ctrl+Maj \shift+N}
	\item Enregistrer le fichier Grisbi: \keys{Ctrl+S}
	\item Importer un fichier: \keys{Ctrl+I}
	\item Exporter vers un fichier \gls{QIF}/\gls{CSV}: \keys{Ctrl+E}
	\item Fermer le fichier Grisbi: \keys{Ctrl+W}
	\item Fermer Grisbi: \keys{Ctrl+Q}
\end{itemize}


\subsection{Panneau de navigation}

\begin{itemize}
	\item Sélectionner un onglet ou un compte: \keys{\arrowkeyup}, \keys{\arrowkeydown}
	\item Sur l'onglet Comptes:
		\begin{itemize}
			\item Ouvrir la liste des comptes: \keys{\arrowkeyright}
			\item Fermer la liste des comptes: \keys{\arrowkeyleft}
		\end{itemize}
	\item Sur un compte ou sur l'onglet Échéancier:
		\begin{itemize}
			\item Bascule vers la liste des opérations ou des opérations planifiées: \keys{\arrowkeyright}
			\item Bascule vers le panneau de navigation: \keys{\arrowkeyleft}
		\end{itemize}
	\item Sur un compte sélectionné:%TODO to update
		\begin{itemize}
			\item Affiche la fenêtre de recherche de caractères: \keys{Ctrl+F}
		\end{itemize}
	\item Sur l'onglet États:%TODO to update
		\begin{itemize}
			\item Ouvre la fenêtre des propriétés d'un état de recherche: \keys{Ctrl+F}
		\end{itemize}	
	\item Sélectionner le premier/dernier onglet (Comptes/États): \keys{Pg.Préc} ou \keys{Pg.Suiv}
\end{itemize}


\subsection{Liste des opérations ou des opérations planifiées}

\begin{itemize}
	\item Déplacer la sélection: \keys{\arrowkeyup} ou \keys{\arrowkeydown}
	\item Sélectionner une opération: \keys{Entrée \return}
	\item Nouvelle opération:
		\begin{itemize}
			\item Sur une ligne vide: \keys{Entrée \return}
			\item Sur une opération existante: \keys{Ctrl+T}
		\end{itemize}
	\item Modifier une opération: \keys{Entrée \return}
	\item Supprimer une opération: \keys{Suppr}
	\item Pointer ou dépointer une opération:
		\begin{itemize}
			\item dans le mode rapprochement: \keys{Ctrl+P}, \keys{F12} ou \keys{Espace},
			\item hors mode rapprochement: \keys{Ctrl+P} ou \keys{F12}
		\end{itemize}
	\item Rapprocher ou dé-rapprocher une opération: \keys{Ctrl+R}\newline
	\Note: hors mode rapprochement, Grisbi ouvre une fenêtre vous demandant de sélectionner un rapprochement.
	\item Déplier/replier une opération ventilée: \keys{Espace}
	\item Montrer ou masquer les opérations rapprochées: \keys{Alt+R}
	\item Montrer ou masquer les lignes d'archives: \keys{Alt+L}
	\item Dans la liste des opérations d'un compte:
		\begin{itemize}
			\item Ouvrir la fenêtre de recherche de caractères: \keys{Ctrl+F}
		\end{itemize}
\end{itemize}


\subsection{Formulaire de saisie des opérations}

\begin{itemize}
	\item La touche \keys{Entrée \return} est configurable (voir les préférences \vref{setup-form-behaviour-enter}): elle permet soit de se déplacer dans le formulaire de saisie, soit de valider l'entrée;
	\item Se déplacer au champ suivant: \keys{Tab \tab} (selon votre choix de configuration);
	\item Annuler la saisie en cours: \keys{\esc}
	\item Accepter l'auto-complètement: \keys{Tab \tab} ou \keys{Entrée \return} (selon votre choix de configuration);
	\item Symbole de l'euro: \keys{\AltGr+E}
\end{itemize}


\subsection{Listes déroulantes}

\begin{itemize}
	\item Ouvrir une liste: \keys{Pg.Préc}, \keys{Pg.Suiv} ou \keys{\arrowkeydown}
	\item Se placer en haut de la liste: \keys{Pg.Préc}
	\item Se placer en bas de la liste: \keys{Pg.Suiv}
	\item Se déplacer dans la liste: \keys{\arrowkeyup} ou \keys{\arrowkeydown}
	\item Valider un choix à l'intérieur d'une liste: \keys{Entrée \return} ou \keys{Tab \tab} en fonction de votre configuration (ici \vref{setup-form-behaviour-enter});
	\item Devises, exercices et modes de règlement:
		\begin{itemize}
			\item Ouvrir la liste: \keys{\Space{Espace}\Space}
			\item Se déplacer dans la liste: \keys{\arrowkeyup} ou \keys{\arrowkeydown}
			\item Valider l'item de la liste: \keys{\Space{Espace}\Space}
		\end{itemize}
\end{itemize}


\subsection{Dates saisies au calendrier}

\begin{itemize}
	\item Ouvrir un calendrier mensuel (sur le champ de date): \keys{Ctrl+Entrée \return}
	\item Fermer le calendrier sans modifier la date: \keys{Ech}
	\item Valider la date sélectionnée: \keys{Entrée \return}
	\item Jour suivant ou précédent: \keys{{+}} ou \keys{{-}}, \keys{\arrowkeyright} ou \keys{\arrowkeyleft}
	\item Semaine précédente ou suivante: \keys{\arrowkeyup} ou \keys{\arrowkeydown}
	\item Mois précédent ou suivant: \keys{Pg.Préc} ou \keys{Pg.Suiv}
	\item Premier jour ou dernier jour du mois: \keys{Début / Orig / \,$\nwarrow$\,} ou \keys{Fin}
\end{itemize}


\subsection{Dates saisies au clavier}

\begin{itemize}
	\item Jour suivant ou précédent: \keys{{+}} ou \keys{{-}}
	\item Semaine précédente ou suivante: \keys{Ctrl+{+}} ou \keys{Ctrl+{-}}
	\item Mois précédent ou suivant: \keys{Pg.Suiv} ou \keys{Pg.Préc}
	\item Année précédente ou suivante: \keys{Ctrl+Pg.Suiv} ou \keys{Ctrl+Pg.Préc}
	\item Valide la date sélectionnée: \keys{Entrée \return}
\end{itemize}


\subsection{Tiers, catégories, imputations budgétaires, simulateur de crédits, données historiques et prévisions}

\begin{itemize}
	\item Déplacer la sélection: \keys{\arrowkeyup}, \keys{\arrowkeydown}, \keys{Pg.Préc} ou \keys{Pg.Suiv}
	\item Afficher les sous-catégories ou sous-imputations budgétaires (sur une catégorie ou une imputation budgétaire): \keys{{+}}
	\item Afficher les opérations des sous-catégories ou sous-imputations budgétaires (sur une sous-catégorie ou une sous-imputation budgétaire): \keys{Entrée \return}
\end{itemize}


\subsection{Configuration}

\begin{itemize}
	\item Afficher la fenêtre des préférences de Grisbi: \keys{Ctrl+Maj \shift+P}
\end{itemize}

\subsection{Aide}

\begin{itemize}
	\item Ouvrir le Manuel de Grisbi: \keys{F1}
\end{itemize}
