%%%%%%%%%%%%%%%%%%%%%%%%%%%%%%%%%%%%%%%%%%%%%%%%%%%%%%%%%%%%%%%%%
% Contents : The first start chapter
% $Id : grisbi-manuel-start.tex, v 0.4 2002/10/27 Daniel Cartron
% $Id : grisbi-manuel-start.tex, v 0.5.0 2004/06/01 Loic Breilloux
% $Id : grisbi-manuel-start.tex, v 0.6.0 2011/11/17 Jean-Luc Duflot
% $Id : grisbi-manuel-start.tex, v 0.8.9 2012/04/27 Jean-Luc Duflot
% $Id : grisbi-manuel-start.tex, v 1.0 2014/02/12 Jean-Luc Duflot
% $Id : grisbi-manuel-start.tex, v 3.0 2024/11 Dominique Brochard
%	- comment \ifIllustration to always have pictures in manuel
%	- accounts' file to accounting's file / fichier de compteS vers fichier de comptabilité
%	- add new "start_category_select" screenshot (4/6)
%	- rewriting
%%%%%%%%%%%%%%%%%%%%%%%%%%%%%%%%%%%%%%%%%%%%%%%%%%%%%%%%%%%%%%%%%


\chapter{Premier démarrage de Grisbi\label{start}}


\section{Assistant premier démarrage\label{start-first}}


Après l'installation de Grisbi, au premier lancement du logiciel, l'assistant premier démarrage s'affiche pour
vous aider à configurer l'application. Il comprend deux étapes, dont la deuxième qui concerne
la gestion du \indexword{fichier de comptabilité}\index{fichier de comptabilité} (chargement et enregistrement automatiques, chiffrement et copies de sauvegarde).

\begin{figure}[htbp]
	\begin{center}
		\includegraphics[width=0.9\textwidth]{image/screenshot/start_first_launch}
	\end{center}
	\caption{Configuration initiale du fichier de comptabilité.}
	\label{start_first_launch}
\end{figure}

Il est conseillé de cocher les options :

\begin{itemize}
 \item chargement automatique du dernier fichier consulté;
 \item enregistrer automatiquement en fermant;
 \item effectuer une copie de sauvegarde avant d'enregistrer les fichiers (coché par défaut).
\end{itemize}

\vspacepdf{5mm}
\textcolor{red}{\strong{Attention}}: si vous cochez l'option \menu{Chiffrer le fichier Grisbi}, la perte du mot de passe rendra votre fichier inutilisable.
\vspacepdf{5mm}

Cet assistant est suivi automatiquement par un deuxième, l'assistant de création du \indexword{fichier de comptabilité}\index{fichier de comptabilité}. Puis vient automatiquement un troisième assistant, l'assistant de création de compte, qui permet de créer le premier compte. Tout cela est décrit en détail dans la section \ref{start-newfile} ci-dessous.

À tout moment vous pouvez sortir de n'importe quel assistant par le bouton \menu{Annuler}.

Si vous ne voulez pas utiliser l'assistant premier démarrage, vous pouvez à la
place utiliser un fichier exemple (voir la section ci-dessous).


\section{Fichier exemple\label{start-example}}


Si vous voulez utiliser Grisbi immédiatement sans être obligé de rentrer tout un tas d'opérations, par exemple pour vous faire une idée des possibilités de ce logiciel, vous pouvez télécharger le fichier \file{Example\_3.0-fr.gsb} sur le site de \lang{Sourceforge.net}\footnote{\urlSourceForgeDocumentation{}} dans le dossier << \textsf{examples} >>.

% espace avant Attention ou Note  :  5mm
\vspacepdf{5mm}
\textbf{Note} : dans ce fichier exemple, les noms des tiers sont de pure invention ; seul un hasard indépendant de notre volonté peut avoir fait que ce soit celui d'une personne ou d'une entité existante.


\section{Création d'un nouveau fichier de comptabilité\label{start-newfile}}


La première fois que vous utiliserez Grisbi, vous devrez d'abord créer un \indexword{fichier
de comptabilité}\index{fichier de comptabilité} (ou fichiers de compte\textbf{S} pour Grisbi, notez le \textbf{S}). L'\gls{extension} de ce fichier sera \file{.gsb} et son nom \file{nom-de-votre-fichier.gsb}. 

Immédiatement après, il vous faudra créer au minimum un compte (bancaire, de caisse, de passif ou d'actif, décrits au chapitre \vref{accounts} \menu{Gestion des comptes}), et par la suite quelques autres comptes (comptes courants, d'épargne, de crédit, éventuellement un compte d'espèces et quelques comptes de transition) qui contiendront leurs opérations respectives. 

Pour une gestion familiale, vous n'aurez normalement qu'un seul fichier de comptabilité (que Grisbi appelle fichier de compte\textbf{S}, notez le \textbf{S}), car cela permet tous les échanges entre vos différents comptes. Si vous gérez une association, ou une autre famille sans rapport comptable avec la première, vous créerez un autre fichier de comptabilité, qui portera un autre nom \file{nom-de-votre-deuxième-fichier.gsb}. Ainsi les \indexword{entités comptables}\index{entité comptable} resteront bien séparées.

% espace avant Attention ou Note  : 5 mm
\vspacepdf{5mm} % TODO à mettre à jour au changement dans Grisbi du fichier de "comptes" par "comptabilité"
\textcolor{red}{\strong{Attention}} : pour une entité comptable donnée, il est nécessaire et important de bien distinguer LE \frquote{\indexword{fichier de comptes}\index{fichier de comptes}} et LES \frquote{\indexword{fichiers de compte}\index{fichiers de compte}} :

\begin{itemize}
	\item LE \frquote{fichier de comptes} que vous aurez créé aura pour extension \file{.gsb} et pour nom \file{nom-de-votre-fichier.gsb} ; il contient toutes les données de tous les comptes créés pour la gestion d'une entité comptable ;
	\item LES \frquote{fichiers de compte} sont des fichiers que vous pourriez être amené(e) à utiliser ou à créer pour importer ou exporter des données d'une application de comptabilité à une autre ; ces fichiers ne contiendront que les données d'un seul compte (courant ou autre) à la fois ; ils auront des extensions différentes (\file{.ofx}, \file{.csv} ou \file{.qif}) suivant leur contenu ; pour plus de détails, voir le chapitre \vref{move}, \menu{Export et import de comptes}.
\end{itemize}
% espace après Attention ou Note  : 5 mm
\vspacepdf{5mm}

Autrement dit, l'ensemble des comptes de votre foyer est enregistré dans un fichier de comptabilité, et l'ensemble des comptes de votre association l'est dans un autre fichier de comptabilité; et un compte dans Grisbi peut correspondre à un fichier de compte, mais seulement lorsqu'on parle d'importation ou d'exportation de données.

% espace pour changement de thème
\vspacepdf{5mm}
Le déroulement général de la procédure de création d'un fichier de comptabilité est le suivant : cliquez sur le menu \menu{Fichier > Nouveau fichier de comptabilité} ; l'assistant de création de fichier de comptabilité s'ouvre, qui comprend six étapes. À la sixième étape, l'assistant vous propose :

\begin{itemize}
	\item  soit de créer un nouveau compte, et il enchaîne sur l'assistant de création de compte, qui comprend lui-même cinq étapes, pour créer le premier compte (car il est indispensable de disposer d'au moins un compte) ;
	\item soit d'utiliser des données déjà existantes, et il enchaîne alors sur l'assistant d'importation des opérations, qui comprend aussi cinq étapes, pour importer des opérations de comptes existants.
\end{itemize}

Après la création de ce premier compte ou l'importation d'opérations de comptes existants, si vous voulez créer d'autres comptes, vous retournerez à la fin de la procédure de création du fichier de comptes, ce qui vous renverra dans les deux cas vers la création d'un nouveau compte.

% espace pour changement de thème
\vspacepdf{5mm}
Pour créer votre fichier de comptabilité, cliquez sur le menu \menu{Fichier > Nouveau fichier de comptabilité}, la procédure détaillée est la suivante:

\begin{enumerate}
	\item Fenêtre d'accueil : validez par le bouton \menu{Suivant} (étape 1/6) :
	\item Configuration
%		\ifIllustration
 générale (étape 2/6)%\refimage{start_file_create-img}
 :

%		\else générale :
%		\fi

%		\ifIllustration
		% image centrée
		\begin{figure}[htbp]
		\begin{center}
		\includegraphics[width=0.9\textwidth]{image/screenshot/start_file_create}
		\end{center}
		\caption{Configuration générale du fichier de comptabilité}
		\label{start-file-create}
		\end{figure}
		% image centrée
%		\fi
		
		\begin{enumerate} 
		 	\item choisissez le nom de l'entité comptable dont vous gérez la comptabilité, par exemple \frquote{Ma comptabilité}, qui pourra être choisi comme titre de la page d'accueil de l'application Grisbi,
			\item saisissez le nom du fichier de comptes avec son arborescence complète ; Grisbi vous propose par défaut le même nom que celui de l'entité comptable, mais vous pouvez le modifier,
			\item cochez la case \menu{Chiffrer le fichier Grisbi} si vous voulez \gls{chiffrer} le fichier de comptes,
			
			\vspacepdf{2mm}
			\textcolor{red}{\strong{Attention}}: si vous cochez l'option \menu{Chiffrer le fichier Grisbi}, la perte du mot de passe rendra votre fichier inaccessible.
			\vspacepdf{2mm}
			
			\item sélectionnez le \indexword{format de la date}\index{format de date} avec l'un des quatre boutons :
			%\begin{addmargin}{-0.2cm}
				\begin{itemize}	
				\item[\textopenbullet] "dd/mm/yyyy" pour "jour/mois/année",
				\item[\textopenbullet] "mm/dd/yyyy" pour "mois/jour/année",
				\item[\textopenbullet] "dd.mm.yyyy" pour "jour.mois.année",
				\item[\textopenbullet] "yyyy-mm-dd" pour "année-mois-jour",
				\end{itemize}
			%\end{addmargin}
			\item choisissez le \indexword{séparateur}\index{séparateur} décimal et celui des milliers dans les listes déroulantes,
			 \item renseignez l'adresse (facultatif),
			 \item  validez par le bouton \menu{Suivant} ;
		\end{enumerate}
		
	\item Sélection de la \indexword{devise}\index{devise} de base (étape 3/6):
		\begin{enumerate} 
		 	\item cliquez sur la devise choisie dans la liste,
			\item cochez la case \menu{Afficher les devises obsolètes} si vous voulez aussi afficher d'anciennes devises,
			\item validez par le bouton \menu{Suivant};
		\end{enumerate}

	\item sélection des \indexword{types de catégories}\index{catégories !types} utilisées (étape 4/6):
	
	\begin{figure}[htbp]
	\begin{center}
		\includegraphics[width=0.9\textwidth]{image/screenshot/start_category_select}
	\end{center}
	\caption{Sélection des catégories à utiliser}
	\label{start_category_select}
	\end{figure}
		
		\begin{enumerate} 
		 	\item cliquez sur la catégorie choisie dans la liste ci-dessus:
%		 	\begin{itemize}
%		 		\item\menu{Catégories générales},
%		 		\item \menu{Liste vide},
%		 		\item\menu{Comptabilité libérale},
%		 		\item\menu{Plan comptable associatif simplifié}
%		 		\item\menu{Plan comptable associatif},
%		 	\end{itemize}
			\item cochez la case \menu{Afficher l'ensemble des catégories} si vous voulez aussi afficher d'autres catégories libellées en anglais,
			\item validez par le bouton \menu{Suivant} ;
		\end{enumerate}		

	\item Définition des \indexword{banques}\index{banques !définition} détenant vos comptes (étape 5/6):
		\begin{enumerate} 
		 	\item cliquez sur  \menu{Ajouter} pour définir une banque ; renseignez les détails de la banque (nom, code banque, etc.), puis validez par le bouton \menu{Ajouter} pour valider la banque,
			\item sélectionnez une banque dans la liste et validez par le bouton \menu{Enlever} pour supprimer une banque, puis confirmez dans la fenêtre qui s'ouvre,
			\item répétez les actions a et b autant de fois que nécessaire,
			\item  validez par le bouton \menu{Suivant} pour passer à l'étape suivante, \menu{Création d'un nouveau compte} ;
		\end{enumerate}		 	

	\item Configuration terminée (étape 6/6):\par
	La configuration du fichier de comptabilité est terminée, et la fenêtre ci-dessous vous propose de choisir l'une des deux méthodes de création de votre premier
%\ifIllustration
compte:%\refimage{start-account-choice-img} :
%\else compte :
%\fi
\vspace{5mm}
%\ifIllustration
% image centrée
\begin{figure}[htbp]
\begin{center}
\includegraphics[width=0.9\textwidth]{image/screenshot/start_account_choice}
\end{center}
\caption{Choix du premier compte}
\label{start_account_choice}
\end{figure}
% image centrée
%\fi

		\begin{itemize}
			\item \menu{Créer un nouveau compte vide} : si vous cochez cette ligne, puis si vous validez par le bouton \menu{Fermer}, cette fenêtre se ferme et l'assistant de création de nouveau compte démarre. Reportez-vous à la section \vref{accounts-new}, \menu{Création d'un nouveau compte}, qui décrit entièrement cette procédure, puis revenez à cette page ;

			\item \menu{À partir de données provenant d'un fichier bancaire ou d'un autre logiciel} : si vous cochez cette ligne, puis si vous validez par le bouton \menu{Fermer}, cette fenêtre se ferme et l'assistant d'importation des données d'un fichier de compte par Grisbi démarre. Reportez-vous à la section \vref{move-import-importinit}, \menu{Import de fichiers de compte d'un autre logiciel dans Grisbi}, qui décrit entièrement cette procédure, puis revenez à cette page.
		\end{itemize}
\end{enumerate}

% étiquette du paragraphe suivant, pour que les liens hypertexte dans account.tex et QIF.tex  arrivent bien dessus
\label{start-newfile-end}

\textit{\textbf{D'une manière ou d'une autre}}, vous venez donc de créer votre fichier de comptabilité, ainsi que le premier compte de ce fichier. 

%espace pour changement de thème
\vspacepdf{5mm}
Si vous voulez créer maintenant d'autres comptes, sélectionnez le menu \menu{Édition - Nouveau compte} pour créer un autre compte (voir la section \vref{accounts-new}, \menu{Création d'un nouveau compte}).

%espace pour changement de thème
\vspacepdf{5mm}
Sinon, vous pouvez commencer à utiliser le compte que vous venez de créer ou celui dont vous venez d'importer les données.

% espace avant Attention ou Note  : 5 mm
\vspacepdf{5mm}
\textcolor{red}{\strong{Attention}} : d'une manière générale, il est déconseillé d'avoir des accents ou des espaces dans les noms des répertoires et fichiers utilisés par Grisbi. Si c'est le cas, renommez-les maintenant. Par exemple, les espaces peuvent être remplacées par des tirets bas (\_).

% saut de page pour titre solidaire
%\newpage


\section{Enregistrement de votre fichier de comptabilité\label{start-save}}


Vos opérations ne sont pas écrites au fur et à mesure de leur saisie comme 
elles peuvent l'être dans d'autres logiciels; vous devez donc enregistrer votre fichier de comptabilité avant de quitter. N'ayez crainte, Grisbi vous prévient si vous ne l'avez pas fait. 

Vous pouvez configurer les options d'enregistrement du fichier de comptabilité dans le menu \menu{Édition - Préférences}, voir le paragraphe \vref{setup-general-files-manage}, \menu{Gestion des fichiers de comptes}.


\section{Import à partir d'un autre logiciel de comptabilité personnelle}

Voir la section \vref{move-import-importinit} pour importer des fichiers de compte d'un autre logiciel dans Grisbi.  Pour le moment, Grisbi supporte les formats \gls{Gnucash}, \gls{OFX}, \GLS{CSV} et \GLS{QIF}.


