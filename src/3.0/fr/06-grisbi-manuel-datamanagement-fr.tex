%%%%%%%%%%%%%%%%%%%%%%%%%%%%%%%%%%%%%%%%%%%%%%%%%%%%%%%%%%%%%%%
% Contents: The data management chapter
% $Id: grisbi-manuel-datamanagement.tex, v 0.8.9 2012/04/27 Jean-Luc Duflot
% $Id: grisbi-manuel-datamanagement.tex, v 1.0 2014/02/12 Jean-Luc Duflot
% $Id: 06-grisbi-manuel-datamanagement-fr.tex, v 3.0 2025/11 Dominique Brochard
%%%%%%%%%%%%%%%%%%%%%%%%%%%%%%%%%%%%%%%%%%%%%%%%%%%%%%%%%%%%%%%%%

\chapter{Gestion des données\label{datamanagement}}


Les données que vous avez entrées dans Grisbi et les traitements que vous en avez faits sont en nombre important; de ce fait ils ne doivent en aucun cas être perdus, et leur quantité ne doit pas être un obstacle à leur bonne gestion. Grisbi propose donc trois outils pour faire face à ces problématiques:
	\begin{itemize}
		\item la \menus{Gestion des fichiers de comptes} créés par Grisbi,
		\item les \menus{Sauvegardes} de ces mêmes fichiers,
		\item l'\menus{Archivage} des opérations stockées dans ces fichiers.
	\end{itemize}
	
\section{Gestion des fichiers de comptes\label{datamanagement-files}}

Vous pouvez définir les options de gestion suivantes dans le menu \menus{Éditer - Préférences}:

\begin{itemize}
	\item le chargement automatique du dernier fichier consulté;
	\item l'enregistrement automatique lors de la fermeture;
	\item le \indexword{forçage de l'enregistrement}\index{enregistrement!forçage} des fichiers verrouillés;
	\item le \indexword{chiffrement du fichier}\index{fichier!chiffrement}\index{chiffrement!fichier};
	\item la \indexword{Compression} du fichier\index{fichier!compression}\index{compression!fichier} (voir \gls{compression});
	\item la mémorisation des derniers fichiers ouverts.
\end{itemize}

Toutes ces options sont explicitées en détail et peuvent être configurées (voir le paragraphe \vref{setup-general-files-manage}, \menus{Gestion des fichiers de comptes}).


\section{Sauvegardes\label{datamanagement-backups}}


D'une manière générale, quelles que soient les données que vous possédez dans le disque dur de votre ordinateur, vous devez impérativement en faire des sauvegardes, pour la simple raison que \emph{tout système de stockage de données a une durée de vie limitée}. Faire des sauvegardes a pour but de limiter les risques de pertes de données.

Grisbi vous permet de faire des sauvegardes automatiques de votre fichier de comptes. Ces sauvegardes devraient être stockées dans un répertoire spécial ou une \gls{partition} spéciale du disque de votre ordinateur, avec les sauvegardes de toutes vos autres données, ce qui vous permettrait alors de sauvegarder facilement ce répertoire ou cette partition, de préférence sur des supports de type différents, indépendants de l'ordinateur, et mis en lieu sûr.

% espace avant Attention ou Note : 5 mm
\vspacepdf{5mm}
\Note{}: vous êtes maintenant prévenu(e), prenez ces conseils au sérieux: ne prenez pas de risques avec vos données, cela peut vous éviter bien des déboires\ldots
% espace après Attention ou Note : 5 mm
\vspacepdf{5mm}

Grisbi peut enregistrer automatiquement, dans un répertoire à définir:
\begin{itemize} 
 \item soit un fichier de sauvegarde unique avec un nom de la forme \file{nom\_du\_fichier\_backup.gsb} et qui est mis à jour régulièrement,
 \item soit des fichiers de sauvegarde qui s'accumulent dans ce répertoire; le nom de ces fichiers de sauvegarde est de la forme \file{nom\_du\_fichier\_AAAAMMJJTHHMMSS.gsb}, où:
 	\begin{itemize}
 		\item \frquote{nom\_du\_fichier} est le nom de votre fichier Grisbi,
 		\item \frquote{AAAAMMJJ} est la date en année-mois-jours,
 		\item \frquote{T} (pour \emph{time}) sépare les indications de date (à gauche) et d'heure (à droite),
 		\item \frquote{HHMMSS} est l'heure en heures-minutes-secondes.
 	\end{itemize}
 Ce format est basé sur le format international de date ISO 8601, ce qui permet, entre autres, le classement automatique par ordre alphanumérique et chronologique dans votre répertoire de sauvegarde.
\end{itemize}
%\newpage

%espace pour changement de thème
\vspacepdf{5mm}
Grisbi vous propose les options de sauvegarde suivantes:

\begin{itemize}
	\item la création d'un fichier de sauvegarde unique, sinon les fichiers de sauvegarde s'ajoutent dans leur répertoire;
	\item la \indexword{\gls{compression} du fichier de sauvegarde}\index{fichier de sauvegarde!compression}\index{compression!fichier de sauvegarde}, pour occuper moins d'espace disque;
	\item la sauvegarde après l'ouverture du fichier Grisbi; 
	\item la sauvegarde avant l'enregistrement du fichier Grisbi; 
	\item le réglage de l'intervalle entre deux sauvegardes, en minutes;
	\item le réglage de la suppression des sauvegardes en fonction de leurs ancienneté, en mois;
	\item la définition du \indexword{répertoire de sauvegarde}\index{répertoire de sauvegarde}\index{fichier de sauvegarde!répertoire}.
\end{itemize}

%espace pour changement de thème
\vspacepdf{5mm}
Toutes ces options sont décrites en détail et peuvent être configurées dans le menu \menus{Éditer - Préférences} (voir le paragraphe \vref{setup-general-files-backup}, \menus{Sauvegardes}).


\section{Archivage\label{datamanagement-archive}}


Une archive est une sorte de \frquote{mise entre parenthèses} d'une partie des opérations de \textit{\textbf{tous}} les comptes de votre fichier créé par Grisbi. Les opérations à l'intérieur d'une archive ne sont plus affichées et ne peuvent plus faire l'objet de traitements, mais elles sont toujours conservées dans ce fichier. Vous pouvez toujours et à tout moment dés-archiver une archive existante pour en afficher les opérations et l'inclure dans un traitement.

\vspace{.2em}
Lorsque vous utilisez Grisbi, vous entrez des opérations dans vos différents comptes. Ces opérations sont toutes enregistrées dans la mémoire et sur le disque dur de l'ordinateur, et une petite partie est affichée sur l'écran. L'affichage et le traitement des opérations consomme donc de la mémoire et du temps de microprocesseur.

\vspace{.2em}
Au fur et à mesure que le temps passe, il y a de plus en plus d'opérations enregistrées, donc leur affichage et leurs traitements demandent de plus en plus d'espace mémoire et de temps de microprocesseur. Votre ordinateur devient donc de plus en plus lent, mais évidemment, cela dépend toujours de ses propres caractéristiques.

\vspace{.2em}
Pour limiter cette perte de performances dans l'affichage et le traitement, en particulier dans l'établissement d'états ou dans la recherche d'informations, Grisbi vous propose de choisir une partie des opérations et de les mettre dans une archive, c'est-à-dire de les mettre à part pour leur éviter d'être concernées par de futurs affichages ou traitements.


\subsection{Création d'une archive\label{datamanagement-archive-new}}

La création d'une archive peut être déclenchée automatiquement ou manuellement:

\subsubsection{Déclenchement automatique de la création d'une archive\label{datamanagement-archive-auto}}

Lorsqu'un certain nombre d'opérations enregistrées est atteint, Grisbi peut vous avertir que cette quantité d'opérations n'a pas été encore archivée. Pour cela, il vous faudra activer la \menus{création automatique} des archives dans les préférences de Grisbi (voir le paragraphe \vref{setup-general-archives-create}. 

\vspace{.5em}
Cocher l'option \menus{Créer automatiquement une archive si nécessaire}:
\begin{itemize}
	 \item lancera l'assistant d'archivage à l'ouverture de votre fichier Grisbi si le seuil de déclenchement est atteint, 
	 \item active le seuil d'avertissement de trois mille opérations (minimum par défaut), libellé \menus{Avertir si plus de \ldots{} opérations ne sont pas archivées}.
\end{itemize}

La première fenêtre de l'assistant \menus{Créer une archive} vous informera du nombre total d'opérations enregistrées dans votre fichier.

A l'issue de l'archivage, le processus de comptage est remis à zéro et Grisbi vous proposera de nouveau le même avertissement après trois mille opérations supplémentaires.


\subsubsection{Création manuelle d'une archive\label{datamanagement-archive-manu}}

La création manuelle peut se faire en plus ou à la place du lancement automatique. L'avertissement du nombre d'opérations non archivées n'est pas actif.

\begin{enumerate}
	\item dans la barre de menus, sélectionnez \menus{Fichier - Créer une archive}: la fenêtre de l'assistant de création d'archive s'affiche; validez par le bouton \menus{Suivant};
	\item dans la fenêtre suivante, vous pouvez choisir parmi les trois modes de sélection des opérations à archiver \refimage{datamanagement-archive-create-img}:
	% image centrée
	\begin{figure}[htbp]
	\begin{center}
	%\vspace{-5pt}
	\includegraphics[width=0.95\textwidth]{image/screenshot/datamanagement_archive_create}
	\end{center}
	%\vspace{-10pt}
	\caption{Création d'une archive}
	%\vspace{-10pt}
	\label{datamanagement-archive-create-img}
	\end{figure}
	% image centrée
		\begin{itemize}
			\item \menus{Tri par date}: saisissez la \menus{Date initiale} et la \menus{Date finale} dans les champs adéquats,
			\item \menus{Archiver les opérations par exercice}: sélectionnez un exercice disponible dans la liste déroulante,
			\item \menus{Archiver les opérations appartenant à l'état}: sélectionnez un état disponible dans la liste déroulante;
% saut de ligne pour indentation correcte de la note dans la liste

			\Note{}: la dernière ligne dans la fenêtre indique soit une erreur de saisie de ces paramètres (en rouge), soit le nombre d'opérations qui seront archivées (tous comptes confondus) sur le nombre total d'opérations de votre fichier Grisbi. 
		\end{itemize}
	\item validez par le bouton \menus{Suivant};
	\item dans la fenêtre suivante, saisissez le nom que vous voulez donner à cette archive; validez avec le bouton \menus{Appliquer};
	\item la dernière fenêtre vous informe que l'archive a été créée, et affiche le nom de l'archive ainsi que le \indexword{nombre d'opérations archivées}\index{archive!nombre d'opérations} \emph{tous comptes confondus} sur le nombre total d'opérations de votre fichier Grisbi; validez avec le bouton \menus{Fermer} ou cliquer sur le bouton \menus{Précédent} pour créer une autre archive.
\end{enumerate}

\vspacepdf{5mm}
\Note{}: au cas où Grisbi serait devenu plus lent après avoir créé une archive, vous pouvez le configurer pour ne pas charger les opérations rapprochées (R) au démarrage, afin d'augmenter sa rapidité (voir la section \ref{transactions-functions}, \menus{Barre d'outils}).
\vspacepdf{5mm}

\subsection{Affichage des archives\label{datamanagement-archive-display}}

L'\indexword{affichage d'une archive}\index{archive!affichage} apparaît tout en haut de la liste des opérations \emph{de chaque compte}, sous la forme d'une ligne d'opération sur \colorbox[RGB]{60,120,40}{fond vert}, indiquant la date de début de l'archive (pour une archive par dates), son nom et ses paramètres de création (dates, exercice ou état), ainsi que le \indexword{nombre d'opérations archivées}\index{archive!nombre d'opérations} \emph{pour le compte affiché} \refimage{datamanagement-archive-line-img}.

% image centrée
\begin{figure}[htbp]
	\begin{center}
		\includegraphics[width=0.95\textwidth]{image/screenshot/datamanagement_archive_line}
	\end{center}
	\caption{Ligne d'une archive}
	\label{datamanagement-archive-line-img}
\end{figure}
% image centrée

Vous pouvez afficher (ou masquer) la ligne d'archive pour tous les comptes en cochant (ou décochant) la case du menu \menus{Affichage - Montrer les lignes d'archives}, ou en cliquant sur \menus{Affichage} de la barre de menu, puis en cochant (ou décochant) \menus{Montrer les lignes d'archives} (\keys{Alt+L}) dans la liste déroulante.

Si vous voulez consulter les opérations à l'intérieur d'une archive, vous pouvez ouvrir cette archive en double-cliquant sur sa ligne: après validation dans la fenêtre qui apparaît, les opérations sont affichées dans la liste. 

\vspacepdf{5mm}
\Note{}: il ne s'agit que d'une ouverture de l'archive pour affichage, et en aucun cas cette archive n'est supprimée. À la prochaine utilisation de Grisbi, la ligne verte de l'archive réapparaîtra en haut de la liste de chaque compte. Pour une véritable suppression de l'archive, voir la section \vref{datamanagement-archive-remove}, \menus{Suppression d'une archive}.
%\vspacepdf{5mm}

\subsection{Paramètres d'une archive\label{datamanagement-archive-parameters}}

Vous pouvez consulter les paramètres qui ont été définis pendant la création d'une archive, dans le menu
\menus{Éditer - Préférences - Généralités - Archives}. Pour cela, voir la section \vref{setup-general-archives-existing}, \menus{Archives existantes}.


\subsection{Modification d'une archive\label{datamanagement-archive-modify}}

Vous pouvez uniquement \indexword{modifier le nom d'une archive}\index{archive!modification} dans le menu \menus{Édition - Préférences}. Pour cela, voir le paragraphe \vref{setup-general-archives-modify}, \menus{Modifier l'archive}.


\subsection{Suppression d'une archive\label{datamanagement-archive-remove}}

Vous pouvez \indexword{supprimer une archive}\index{archive!suppression} existante, dans le menu \menus{Édition - Préférences}. Il y a deux fonctions de suppression distinctes: la suppression d'une archive tout en \emph{conservant} ses opérations, et la suppression d'une archive tout en \emph{supprimant} toutes ses opérations. Pour cela, voir le paragraphe \vref{setup-general-archives-modify}, \menus{Modifier l'archive}. 


\subsection{Export d'une archive\label{datamanagement-archive-export}}

Exporter une archive permet de créer un fichier contenant une copie de l'archive, afin de la stocker, ou de l'utiliser dans un autre fichier de comptes de Grisbi ou dans une autre application de comptabilité. L'exportation ne peut se faire qu'à travers les formats de fichiers \indexword{\gls{GSB}}\index{gsb}, \indexword{\gls{QIF}}\index{qif} ou \indexword{\gls{CSV}}\index{csv}.

% espace avant Attention ou Note : 5 mm
\vspacepdf{5mm}
\Attention{}: les formats de fichiers QIF et CSV ne supportent pas les devises, et toutes les opérations seront converties dans la devise de leur compte respectif.
% espace après Attention ou Note : 5 mm
\vspacepdf{5mm}

Pour exporter une archive, procédez comme suit:

\begin{enumerate}
	\item dans la barre de menus, sélectionnez \menus{Fichier - Exporter une archive vers un fichier GSB/QIF/CSV}: la fenêtre de l'assistant d'exportation d'archive s'affiche; validez par le bouton \menus{Suivant};
	\item un tableau affiche la liste des archives existantes avec leur nom et, selon le cas, leurs dates initiale et finale, leur exercice ou le nom de l'état; sélectionnez l'archive à exporter en cochant la case dans sa ligne; validez par le bouton \menus{Suivant};
	\item une fenêtre de gestionnaire de fichiers s'affiche; vous pourrez modifier:
		\begin{itemize}
			\item le nom du fichier sous lequel l'archive sera exportée,
			\item le dossier où elle sera enregistrée,
			\item le format du fichier d'exportation entre les formats Grisbi(GSB),QIF ou CSV.
			
			\Note{}: le format QIF n'accepte qu'un fichier par compte; Grisbi créera donc autant de fichiers QIF que de comptes.
		\end{itemize}   
	 Validez par le bouton \menus{Suivant};
	\item la dernière fenêtre vous informe que l'archive a été exportée; validez par le bouton \menus{Fermer}.
\end{enumerate}
