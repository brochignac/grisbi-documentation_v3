%%%%%%%%%%%%%%%%%%%%%%%%%%%%%%%%%%%%%%%%%%%%%%%%%%%%%%%%%%%%%%%%%
% Contents: The importexport chapter
% $Id: grisbi-manuel-QIF.tex, v 0.4 2002/10/27 Daniel Cartron
% $Id: grisbi-manuel-QIF.tex, v 0.5.0 2004/06/01 Loic Breilloux
% $Id: grisbi-manuel-QIF.tex, v 0.6.0 2011/11/17 Jean-Luc Duflot
% $Id: grisbi-manuel-QIF.tex, v 0.8.9 2012/04/27 Jean-Luc Duflot
% $Id: grisbi-manuel-QIF.tex, v 1.0 2014/02/12 Jean-Luc Duflot
% $Id: grisbi-manuel-importexport.tex, v 3.0 2025/06 Dominique Brochard
%%%%%%%%%%%%%%%%%%%%%%%%%%%%%%%%%%%%%%%%%%%%%%%%%%%%%%%%%%%%%%%%%


\chapter{Import et export de comptes\label{importexport}}

Vous ne pouvez pas utiliser directement dans Grisbi des données qui ont été créées par d'autres applications de comptabilité personnelle, et réciproquement. Comme ces applications fonctionnent différemment, leurs données sont structurées différemment: il faut donc convertir leur structure de données avant de pouvoir les utiliser. 

Cette conversion ne peut pas se faire d'un seul coup sur l'ensemble des données, mais doit se faire indépendamment pour chaque compte géré par l'application. Pour convertir chacun de ces comptes, il faut donc d'une part les \frquote{exporter} de l'application d'origine, puis les \frquote{importer} dans l'application de destination.

% espace avant Attention ou Note : 5 mm
\vspacepdf{5mm}

\Note{}: ne pas confondre le fichier Grisbi, d'\gls{extension} \file{.gsb} et qui contient tous les comptes avec leurs données, et les fichiers de ces mêmes comptes, qui sont des fichiers ne contenant que les données d'un seul compte à la fois, et créés uniquement pour importer ou exporter ces données d'une application de comptabilité à une autre. Ces fichiers de compte doivent avoir un \gls{format de fichier} (ou une \gls{extension}) obligatoirement compatible avec l'application d'origine ET l'application de destination.


 %ne pas confondre LE \frquote{fichier de comptes} qui contient toutes les données de tous les comptes créés pour la gestion d'une entité comptable (dans Grisbi, ce fichier porte l'\gls{extension} \file{.gsb}), et LES \frquote{fichiers de compte}, qui sont des fichiers ne contenant que les données d'un seul compte à la fois, et créés uniquement pour exporter ou importer ces données d'une application de comptabilité à une autre. Ces \frquote{fichiers de compte} doivent avoir un \gls{format de fichier} (ou une extension) obligatoirement compatible avec l'application d'origine ET l'application de destination.
% espace après Attention ou Note : 5 mm
\vspacepdf{5mm}

Grisbi supporte actuellement les formats de données de compte de comptabilité personnelle \gls{Gnucash}, \gls{OFX}, \gls{CSV} et \gls{QIF}.


\section{Import de comptes d'un autre logiciel\label{importexport-import}}


Si vous voulez utiliser dans Grisbi des données de comptes qui ont été créés dans une autre application de comptabilité, vous devez d'abord exporter individuellement chacun des comptes de cette application dans un fichier, puis les importer dans Grisbi grâce à ces fichiers.


\subsection{Export d'un compte d'un autre logiciel\label{importexport-import-exportinit}}

La première étape consiste, dans l'application de comptabilité personnelle d'origine, à exporter chaque compte dans un fichier au format choisi. Le format choisi doit être compatible à l'exportation par l'application d'origine \emph{et} compatible à l'importation par Grisbi.

La procédure d'exportation est bien évidemment différente pour chaque logiciel, donc référez-vous à sa documentation. Si vous voulez exporter tous les comptes, vous devrez obtenir autant de fichiers que vous avez de comptes gérés par l'application.


\subsection{Import de fichiers de compte d'un autre logiciel dans Grisbi\label{importexport-import-importinit}}

\Note{}: Grisbi permet d'importer un ou plusieurs fichiers de compte au cours de la même procédure. Bien que l'on puisse importer les fichiers de compte un par un, il est important de bien importer tous les fichiers de compte simultanément, afin que Grisbi puisse recréer les liens entre les comptes, particulièrement en ce qui concerne les opérations de virement.
% espace après Attention ou Note : 5 mm
\vspacepdf{5mm}

Pour plus de renseignements sur les \indexword{types de compte}\index{types de compte} que Grisbi sait gérer, voir la section \vref{accounts-type}, \menus{Types de comptes de Grisbi}.

L'import est paramétrable dans le menu \menus{Éditer - Préférences} (\keys{Ctrl+Maj+P}), menu \menus{Généralités - Importer}, onglet \menus{Importation des fichiers} (voir la section \vref{setup-general-import-files}).

Vous pouvez définir quelle date sera utilisée pour l'attribution d'un exercice à chaque opération importée, voir le paragraphe \vref{setup-general-import-financialyear}, \menus{Définition de l'exercice}.

Grisbi vous permet aussi d'établir une association entre une chaîne de caractères de ce fichier et un tiers. Par exemple, tous les libellés contenant \frquote{loyer} peuvent être associés à un tiers qui représente votre propriétaire. Cela doit être configuré dans le menu \menus{Éditer - Préférences} (\keys{Ctrl+Maj+P}), menu \menus{Généralités - Importer}, onglet \menus{Associations pour l'importation} (voir la section \vref{setup-general-importlinks}).

% espace pour changement de thème
\vspacepdf{5mm}
Dans le menu \menus{Fichier} de Grisbi, choisissez l'option \menus{Importer un fichier} (ou utilisez le raccourci-clavier \keys{Ctrl+I}), ce qui ouvre la fenêtre de l'assistant d'importation. L'importation d'un seul fichier de compte se déroule en cinq étapes, auxquelles il faudra rajouter une étape par compte supplémentaire:

\begin{enumerate}
	\item Accueil de l'assistant d'importation (étape 1/4): validez avec le bouton \menus{Suivant};
	\item Sélection des fichiers de compte à importer (étape 2/4) \refimage{importexport-import-files-select-img}:	
		\begin{enumerate}
			\item cliquez sur le bouton \menus{Ajouter un ou des fichiers...}: une fenêtre de gestionnaire de fichiers s'ouvre;
			\begin{figure}[htbp]
				\raggedleft
				%\begin{flushleft}
					\includegraphics[width=.95\textwidth]{image/screenshot/importexport_import_files_select}
				%\end{flushleft}
				\caption{Sélection des comptes à importer}
				\label{importexport-import-files-select-img}
			\end{figure}
			\item cherchez le répertoire où se trouvent ce ou ces fichiers de compte;
			\item sélectionnez un ou plusieurs fichiers de compte (avec la combinaison \keys{Ctrl+Clic Gauche} et \keys{Maj \shift+Clic gauche}); vous pouvez aussi changer la \gls{locale} (\gls{encodage des caracteres}) des fichiers à importer dans le menu déroulant \menus{Codage};
			\item validez avec le bouton \menus{Ouvrir} pour revenir à la fenêtre de sélection des fichiers de compte;
			\item vous pouvez choisir de ne pas importer les catégories en cochant l'option adéquate. Dans le cas d'import d'un fichier \gls{CSV}, une nouvelle fenêtre vous permet de choisir les paramètres d'import \refimage{importexport-import-CSV-setup-img}:
				\begin{itemize}[label=\textopenbullet]
					\begin{figure}[htbp]
						\raggedleft
						\includegraphics[width=.95\textwidth]{image/screenshot/importexport_import_CSV_setup}
						\caption{Paramétrage de l'import d'un fichier CSV}
						\label{importexport-import-CSV-setup-img}
					\end{figure}
					\item \textbf{Choisissez le séparateur CSV}: le séparateur entre les données peut être sélectionné dans la liste déroulante de la fenêtre de gauche et s'affiche dans la fenêtre de droite, où vous pourrez aussi le modifier;
					\item \textbf{Forcer le format de la date}: le format de date peut être forcé en cochant la case idoine et en le sélectionnant dans la liste déroulante;
					\item \textbf{Sélectionnez les champs}: vous pourrez cocher les lignes de données qui \textit{ne seront pas} importées;
					\item le bouton \frquote{Créer une règle pour cet import.} (en bas) permet de créer une règle d'import que vous devrez nommer pour la valider. Vous la retrouverez dans la barre d'outils du compte  (voir \vref{transactions-functions}).
				\end{itemize}
			\item si les comptes choisis sont bien cochés, vous pouvez valider par le bouton \menus{Suivant};
		\end{enumerate}
	\item Fin de la préparation de l'importation des fichiers de compte: si tout s'est bien passé, cette fenêtre donne la liste des fichiers de compte qui seront importés; continuez l'importation en validant par le bouton \menus{Suivant};
	\item Création et paramétrage de chaque compte importé dans Grisbi (étape 4): vous pouvez passer en revue chaque compte et y choisir les actions suivantes \refimage{importexport-import-files-setup-img}:
	\begin{figure}[htbp]
		\begin{center}
		\includegraphics[width=0.95\textwidth]{image/screenshot/importexport_import_files_setup}
		\end{center}
		\caption{Paramétrage de chaque compte importé}
		\label{importexport-import-files-setup-img}
	\end{figure}
		\begin{itemize}
			\item \menus{Créer un nouveau compte}: cela ajoutera le fichier sélectionné comme nouveau compte dans votre fichier Grisbi. Le menu déroulant \menus{Type de compte}, en dessous, vous permettra de modifier le type de compte;
			\item \menus{Ajouter des opérations à un compte}: si des opérations planifiées sont trouvées dans l'intervalle de temps spécifié, une fenêtre spécifique s'ouvre pour savoir ce que vous voulez en faire: soit fusionner ces opérations planifiées avec les opérations importées correspondantes, soit ajouter les opérations importées en sus de celles-là (voir la section \vref{setup-general-import-parameters}, \menus{Paramètres pour l'import}). Le menu déroulant \menus{Nom du compte}, en dessous, vous permettra de sélectionner le compte auquel seront ajoutées les opérations;
			\item \menus{Marquer les opérations d'un compte}: cela marquera les opérations avec un \frquote{T} dans la colonne \frquote{P/R} (\vref{transactions-list-fields}) du compte concerné. Si des \indexword{opérations orphelines}\index{opération!orpheline} sont trouvées, une fenêtre s'ouvrira en fin d'import pour savoir ce que vous voulez en faire: soit les ajouter, soit les ignorer. Le menu déroulant \menus{Nom du compte}, en dessous, vous permettra de sélectionner le compte dans lequel les opérations seront marquées;
			\item définir la devise du compte (ou bien en créer une nouvelle);
			\item \menus{Inverser le montant de l'opération importée}: utile pour les comptes de carte bancaire de la Banque Postale, par exemple;
			\item \menus{Créer une règle pour cet import}: permet de définir une règle d'import rapide si le fichier est au format \gls{QIF}, \gls{Gnucash} ou \gls{OFX} et uniquement si vous ajoutez ou marquez des opérations à un compte. Cette règle est spécifique à chaque compte et devra être nommée pour être validée. Vous pourrez la retrouver dans la barre d'outils du compte (voir \vref{transactions-functions});
			\item quand tout est correct, validez l'importation par le bouton \menus{Suivant};
		\end{itemize}
	\item Validation de la fin de l'importation: valider par le bouton \menus{Fermer}.
\end{enumerate}

Si, et seulement si, vous venez de créer votre fichier Grisbi juste avant cette importation de données de comptes, revenez à la fin de la section \vref{start-newfile-end}, \menus{Création d'un nouveau fichier de comptes}. Allez juste après la fin de la procédure de création du fichier de comptes, au paragraphe commençant par \textbf{\emph{D'une manière ou d'une autre\ldots{ }}}, ce qui vous proposera de créer tout de suite d'autres comptes.

%espace pour changement de thème
\vspacepdf{5mm}
Sinon, vous pouvez commencer à utiliser le compte que vous venez de créer.


\section{Export de comptes à partir de Grisbi\label{importexport-export}}


Si vous voulez utiliser, dans une autre application de comptabilité, des données de compte qui ont été créées par Grisbi, vous devez d'abord exporter ces données dans des fichiers, puis les importer dans l'autre application grâce à ces fichiers. Le format de fichier choisi doit être compatible à l'exportation par Grisbi \emph{et} compatible à l'importation par l'application de destination.
 
Dans le menu \menus{Fichier}, choisissez l'option \menus{Exporter vers un fichier QIF/CSV} (ou utilisez le raccourci-clavier \keys{Ctrl+E}) qui ouvre l'assistant Export des comptes. L'exportation des comptes comporte à minima quatre étapes:

\begin{enumerate}
	\item Accueil de l'assistant (étape 1/3): cette fenêtre indique que, comme les formats de fichier \gls{QIF} et \gls{CSV} ne supportent pas les devises, toutes les opérations seront converties dans la devise de leur compte respectif; validez par le bouton \menus{Suivant};
	\item Sélection des comptes et des options \refimage{importexport-export-img}:
		\begin{figure}[htbp]
			\begin{center}
				\includegraphics[width=0.95\textwidth]{image/screenshot/importexport_export}
			\end{center}
			\caption{Export des comptes}
			\label{importexport-export-img}
		\end{figure}
		\begin{itemize}
			\item \textbf{Sélectionner les comptes à exporter} (étape 2/3): cliquez sur la ou les cases correspondantes à chaque compte à exporter ou sur le bouton \menus{Sélectionner tout};
			\item \textbf{Sélectionner les options pour l'export}:
			\begin{itemize}
					\item \menus{Format QIF}: exporte le ou les comptes cochés au format \gls{QIF}; en plus l'option:
						\begin{itemize}
							\item \menus{Force les dates au format US}: enregistre la date au format \frquote{mois/jour/année} (mm/dd/yyyy),
							\item \menus{Force les nombres au format US}: utilise le point \frquote{.} comme séparateur de décimale et la virgule \frquote{,} comme séparateur des milliers;
						\end{itemize}
					\item \menus{Format CSV}: exporte le ou les comptes cochés au format \gls{CSV}; en plus des options disponibles au format QIF (ci-dessus):
						\begin{itemize}
							\item le séparateur entre les données peut être sélectionné dans la liste déroulante de la fenêtre de gauche et s'affiche dans la fenêtre de droite, où vous pourrez aussi le modifier;
							\item \menus{Citer les dates}: si cochée (par défaut), les dates seront mises entre guillemets, comme les autres données. Le survol du menu avec la souris ouvre une fenêtre d'information \frquote{Décocher pour ne pas mettre les dates entre guillemets};
						\end{itemize}
			\end{itemize}
			\item \menus{Traiter automatiquement le(s) compte(s) sélectionné(s)}: cette option exporte le ou les comptes sélectionnés automatiquement:
				\begin{itemize}
					\item avec un nom de fichier généré comme suit:\newline
					[nom de l'entité comptable]-[nom du compte].[extension sélectionnée], 
					\item et dont la destination est le dossier où a été ouvert le fichier Grisbi; si le ou les fichiers exportés existent déjà, Grisbi vous demandera confirmation pour le ou les écraser; le survol du menu avec la souris ouvre la fenêtre d'information suivante:\newline
					\frquote{Le nom du(es) fichier(s) de compte sera(ont) nommé(s) comme suit:\newline
					[nom de l'entité comptable]-[nom du compte].[extension]\newline
					Le dossier d'enregistrement sera le dossier du fichier Grisbi}.
				\end{itemize}	
			\Note{}: il est recommandé de ne pas utiliser le caractère barre oblique \frquote{\slash{}} dans le nom du compte sous peine d'avoir une erreur lors de l'export sur les systèmes \gls{Linux}.
			\item validez par le bouton \menus{Suivant};
		\end{itemize}
	\item pour chaque compte, définissez le nom du fichier, le répertoire de destination et le format d'exportation; l'option \menus{Traiter automatiquement le(s) compte(s) sélectionné(s)} shunte cette étape. Puis validez par le bouton \menus{Suivant}
	\item la fenêtre de fin de l'exportation s'affiche; validez par le bouton \menus{Fermer}.
\end{enumerate}

\Attention{}: d'une manière générale, il est déconseillé d'avoir des accents ou des espaces dans les noms des répertoires et fichiers utilisés par Grisbi. Si c'est le cas, renommez-les maintenant. Par exemple, les espaces peuvent être remplacées par des tirets bas (\_).