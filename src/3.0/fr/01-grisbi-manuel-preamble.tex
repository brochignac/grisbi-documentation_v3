%%%%%%%%%%%%%%%%%%%%%%%%%%%%%%%%%%%%%%%%%%%%%%%%%%%%%%%%%%%%%%%%%
% Contents: The preamble chapter
% $Id: grisbi-manuel-preamble.tex, v 0.4 2002/10/27 Daniel Cartron
% $Id: grisbi-manuel-preamble.tex, v 0.6.0 2011/11/17 Jean-Luc Duflot (no change)
% $Id: grisbi-manuel-preamble.tex, v 0.8.9 2012/04/27 Jean-Luc Duflot (typo changes)
% $Id: grisbi-manuel-preamble.tex, v 1.0 2014/02/12 Jean-Luc Duflot
% $Id: grisbi-manuel-preamble.tex, v 3.0 2024/04/10 Dominique Brochard: writing the Préambule in the 3rd person (Daniel Cartron). Add Blu-ray zone 2 and DVD/Blu-ray zone 1.
%%%%%%%%%%%%%%%%%%%%%%%%%%%%%%%%%%%%%%%%%%%%%%%%%%%%%%%%%%%%%%%%%

% supprimé car inopérant en pdf et crée un problème en html
%\begin{small}

%commande étoilée : étoile supprimée pour entrées d'index avec affichage du lien correct en html
\chapter{Préambule\label{preamble}}

%% commenté pour entrées d'index avec affichage du lien correct en html
%% voir aussi le chapitre préambule déplacé dans le \mainmatter dans manuel.tex
%\markboth{}{Préambule}
%\ifpdf
%\addcontentsline{toc}{chapter}{Préambule}
%\else
%\fi

Plusieurs utilisateurs de \gls{Grisbi} ont demandé à Daniel \familyname{Cartron} d'insérer dans le manuel un bref rappel de la signification de ce mot, qui, à son grand dam, est retombé en désuétude.

Ses premières (brèves) recherches n'ayant pas ramené de résultats valant la peine d'être publiés, il avait laissé tomber jusqu'à ce qu'un jour il prenne le temps de passer quelques moments dans une bibliothèque bien fournie en dictionnaires de toutes sortes. Et là, la moisson fut abondante. Tellement abondante que Daniel \familyname{Cartron} hésité longuement pour savoir ce qu'il allait garder et ce qu'il allait éliminer\ldots

Finalement il décida de tout garder, même si on passe ainsi à un paragraphe de quatre pages. Il a juste opéré un classement des articles les plus courts vers les plus longs.

En effet il trouve intéressant d'étudier les divergences entre les différents dictionnaires, mais plus encore de constater les ressemblances, à ce point frappantes que l'on pourrait intituler ce chapitre non pas \emph{Le jeu des sept erreurs} mais \emph{Qui a copié qui?} À vous de trouver\dots

Et il a aussi rajouté un passage sur le film puisque pour ceux qui savent encore ce que \indexword{grisbi}\index{Grisbi} veut dire cela tient essentiellement à la renommée méritée de cette \oe uvre.


\section{Étymologie\label{preamble-etymology}}


Voici donc quelques sources sur l'étymologie\index{etymologie@étymologie} du mot grisbi :


\subsection*{Dictionnaire de la langue française --- Hachette}

[grizbi] n. m. Arg. Argent --- De gris (monnaie grise; cd. rouchi griset [1834], \frquote{liard}), et suff. pop. -bi; 1895, répandu en 1953.


\subsection*{Grand Larousse de la langue française}

[grizbi] n. m. (de \emph{gris[et]}, pièce de six liards [1834, Esnault] --- dér. de \emph{gris}, à cause de la couleur [cf. aussi \emph{grisette}, \frquote{monnaie} --- XVIIe s. ---, et \emph{monnaie blanche et grise}, 1784, Esnault] --- avec le suff.arg. \emph{-bi}; 1896, Delesalle).

\emph{Arg.} Argent : \emph{Touchez pas au grisbi} (titre d'un roman d'Albert Simonin [1953]).


\subsection*{Dictionnaire historique de la langue française --- Robert}

n.m. apparu en 1895 (\emph{grisbis}) et répandu à partir de 1953 par le roman \emph{Touchez pas au grisbi} de A. Simonin, serait composé de \emph{gris} \frquote{ monnaie grise} (1784 : cf. le rouchi \emph{griset} \frquote{pièce de six liards}, 1834; et \emph{grisette} \frquote{monnaie}, v. 1634) et de l'élément \emph{bi} d'origine obscure : \emph{grisbi}, \frquote{argent} en argot, pourrait être un composé tautologique de \emph{gris} et \emph{bis}.


\subsection*{Dictionnaire de l'argot français et de ses origines --- Larousse}

Origine très controversée : soit de griset, \frquote{pièce de monnaie}, et d'un mystérieux suffixe -bi, ou du pain à la fois gris et bis, ou du slang anglais crispy, argent; nous proposons d'y voir un emploi métonymique de gripis 1628 [Cheneau], grispin, grisbis 1849 [Halbert], \frquote{meunier}, c'est-à-dire \frquote{celui qui a chez lui du blé} 1895 [Delsalle] mais remis en circulation par \frquote{Touchez pas au grisbi}, célèbre roman de A. Simonin, paru en 1953.

VARIANTES ---  grijbi : 1902 [Esnault] --- grèzbi : vers 1926 [id.]

DÉRIVÉS --- grisbinette n.f. Pièce de cent anciens francs : 1957
[Sandry-Carrère].


\subsection*{Trésors de la langue française}

\emph{Arg.} Argent. Synon. pop. \emph{fric, galette, pèze, pognon. Le grisbi je suis assez grand pour aller le chercher moi-même ! (\ldots) Riton qu'avait même pas su se tenir en homme (\ldots) dès qu'il s'était senti assez de grisbi} (\familyname{Simonin}, \emph{Touchez pas au grisbi}, 1953, p. 231).

\strong{Prononc.} : [grizbi].  \strong{Étymol. et Hist.} 1896 \emph{grisbis}
arg. \frquote{argent} (\familyname{Delesalle}, \emph{Dict. arg.-fr. et fr.-arg.}). Mot composé du rad. de \emph{griset}, au sens de \frquote{pièce de six liards} (1834 ds \familyname{Esn.}), dér. de \emph{gris}, à cause de la couleur (\emph{cf.} aussi \emph{ca} 1634 \emph{grisette} \frquote{monnaie}, \emph{La Muse Normande} de D. Ferrand, éd. A. Héron, II, 91; 1784, Brest, \emph{monnaie blanche et grise} ds \familyname{Esn.}), et d'une seconde partie d'orig. obsc. qui représente peut-être le suff. pop. -bi, à rapprocher de \emph{nerbi} \frquote{très noir} (d'apr. \familyname{Esn.}). Il n'est pas impossible que \emph{grisbi} (anciennement \emph{grisbis}) soit un composé tautologique de \emph{gris} et de \emph{bis}.
\strong{Bbg.} \familyname{Rigaud} (A.). L'arg. litt. \emph{Vie lang.} 1972, pp. 114-117.


\subsection*{Grand Robert de la langue française}

[grizbi] n. m. --- 1895 : répandu 1953 par le roman de Simonin \emph{Touchez pas au grisbi}; le mot était rare ou archaïque v. 1950 : de \emph{gris} \frquote{monnaie grise} (cf. rouchi \emph{griset} \frquote{liard}, 1834), et suff. pop. \strong{Argot.} Argent. \emph{T'as du grisbi ?}

1 --- Cette expression : \frquote{Ne touchez pas au grisbi} devient une variante de \frquote{Ne chahutez pas avec les nippes}. C'est le maître mot qui dirige la chronique de ces chevaliers de fortune mal acquise qui donnèrent de la mobilité aux romans de cape et de mitraillette de Peter Cheyney.

			\familyname{P. Mac Orlan}, \emph{in} Albert \familyname{Simonin}, Touchez pas au grisbi, Préface, p. 6.

2 --- \frquote{Te casse pas la tête pour les politesses\ldots D'abord on a pas le temps si tu veux que je te trouve Ali. Tout dépend de ce qu'il a de grisbi en fouille; s'il est armé, on a une chance de le trouver au flambe, à la partie du Carillon}

			Albert \familyname{Simonin}, Touchez pas au grisbi, p. 147.


\subsection*{Dictionnaire du français non conventionnel --- Hachette}

n.m. (grisby)

Argent (intrinsèquement).

\emph{Au petit caïd de l'équipe, un mouflet à casquette torpédo, bleu de chauffe et pompes vernies, la môme venait d'affirmer qu'elle me frimait seulement pour le bon motif, pour me soulager de mes cent sacs. Il avait répliqué, le vilain jalmince : --- Le \emph{grisbi}, je suis assez grand pour aller le chercher moi-même !
Ils disaient vrai tous les deux, l'un et l'autre également prêts à tout pour le \emph{grisbi}. Eux et leurs petits potes. Pareil Angelo-la-Tante et Josy-la-Peau-de-Vache; pareil Ali-le-Fumier et ses ordures d'espingos; pareil Riton qu'avait même pas su se tenir en homme avec sa môme, dès qu'il s'était senti assez de \emph{grisbi}; pareil Marco et sa petite Wanda, si honnête, mais qu'hésitait pas à se faire enjamber par le bonhomme \emph{grisbi} ! pareil aussi la môme Lulu sans doute, qu'attendait patiemment chez moi que je rabatte, avec mon \emph{grisbi}!}

			A. \familyname{Simonin}, Touchez pas au grisbi, p. 233

HIST. ---1895, mais sans doute peu usité : A. Bruant et L. Blédort, qui accumulent à l'occasion les synonymes (\emph{pèse, os,} etc.), n'emploient pas \emph{grisbi}, bien que Bruant l'enregistre en 1901 (\emph{grisbis}). Le succès mérité du roman d'A. \familyname{Simonin} en 1953 a rendu une jeunesse au mot, qui ne paraît pas pour autant véritablement intégré à la série des désignants de l'argent, comme \emph{blé, oseille, flouze} ou \emph{fric}.

Du rouchi \emph{griset}, \frquote{pièce de six liards} (1834), ainsi dénommée à cause de sa couleur. Mais l'explication donnée par Esnault, actuellement seule disponible, n'est pas satisfaisante; d'une part, l'élément \emph{bi} reste inexpliqué, sinon par un \frquote{suffixe} inconnu; d'autre part, Bruant écrit \emph{grisbis}, et il est possible (sinon probable) que le \emph{s} central ne soit prononcé que depuis 1953; ce qui amènerait à une explication: \emph{gris-bis}, dans la série des désignants issus d'un nom du pain, \emph{blé, carme, biscuit, galette}, etc.

Enfin, si la métonymie de la couleur est effectivement utilisée pour dénommer l'argent, il s'agit toujours d'une catégorie d'argent précise : des \frquote{espèces}. Ainsi \emph{jaunet, blanc, blanche, cuivre}, ne sont pas interchangeables, ni utilisables pour \frquote{de l'argent} abstrait.

On rappellera par ailleurs le sens de \emph{gris} : \frquote{cher} (V. \emph{grisol}) et la possibilité du pseudo-suffixe augmentatif \emph{bi}, \frquote{très}, même rare. On aurait alors : \emph{gris-bi}, \frquote{très cher} ? Mais l'hypothèse est aventurée.


\section{Bibliographie\label{preamble-biblio}}


\emph{Touchez pas au Grisbi !} par Albert \familyname{Simonin}

\begin{itemize}
	\item Gallimard, Folio Policier \no 183, 1ère édition en 1953, réédition en 2014 (toujours édité)
	\item Gallimard, Folio \no 2068, première édition en 1989
	\item Gallimard, Carré Noir \no 94, première édition en 1972
	\item Le Livre de Poche \no 1152, première édition en 1953
	\item Gallimard, Série Noire \no 148, première édition en 1953
\end{itemize}


\section{Filmographie\label{preamble-filmography}}


\strong{Touchez pas au grisbi !}

Film italien, français (1953). Policier. Durée : 1h 34 min

Titre Original : Grisbi

Distribution :

\begin{itemize}
	\item Jean \familyname{Gabin} : Max le menteur	
	\item René \familyname{Dary} : Riton;
	\item Dora \familyname{Doll} : Lola	
	\item Vittorio \familyname{Sanipoli} : Ramon	
	\item Marilyn \familyname{Buferd} : Betty	
	\item Gaby \familyname{Basset} : Marinette	
	\item Paul \familyname{Barge} : Eugène
\end{itemize}

Réalisateur : Jacques \familyname{Becker}


\subsection*{Synopsis}

Max-le-menteur et Riton viennent de réussir le coup de leur vie : voler 50 millions de francs en lingots d'or à Orly. Avec ce \frquote{grisbi}, les deux gangsters comptent bien profiter d'une retraite paisible. Mais Riton ne peut s'empêcher de parler du magot à sa maîtresse Josy. L'entraîneuse transmet la précieuse information à Angelo, un trafiquant de drogue avec lequel elle trompe Riton. Angelo kidnappe le vieux truand et demande le \frquote{grisbi} à Max comme rançon\dots


\subsection*{Anecdotes}

\paragraph{Un tandem bien huilé :} Jean \familyname{Gabin} et René \familyname{Dary} sont considérés comme deux monstres sacrés du cinéma de l'avant-guerre.

\paragraph{\familyname{Becker} père et fils :} Le fils de Jacques \familyname{Becker}, Jean, fait ici ses débuts au cinéma en tant qu'assistant réalisateur. Il n'a pourtant que quinze ans !

\paragraph{Albert \familyname{Simonin}:} L'écrivain et scénariste Albert \familyname{Simonin}, qui adapte ici son propre roman, fera quatre autres films avec \familyname{Gabin}, tous dialogués par \familyname{Audiard}: \emph{Le cave se rebiffe} (1961) et \emph{Le gentleman d'Epsom} (1962) de Gilles \familyname{Grangier}, \emph{Mélodie en sous-sol} (1963) d'Henri \familyname{Verneuil} et \emph{Le pacha} (1967) de Georges \familyname{Lautner}. Après avoir adapté son \oe uvre \emph{Les Tontons flingueurs} pour Georges \familyname{Lautner} (1963), il devient son scénariste pour \emph{Les Barbouzes} (1964).


\subsection*{Version DVD zone 2}

\begin{itemize}
	\item Interactivité : Menu d'accueil, accès aux scènes, filmographies déroulantes du réalisateur, de Lino \familyname{Ventura} et Jean \familyname{Gabin}
	\item Format cinéma : plein écran
	\item Version sonore : VF en mono
	\item Sous-titres : aucun
	\item France --- 1953 --- Noir \& blanc --- 92 min --- Film Office --- 1~disque --- 1~face --- 1~couche
	\item Date de parution : 19 septembre 2001
	\item Éditeur : Studio Canal
\end{itemize}

\subsection*{Version Blu-ray zone 2}

\begin{itemize}
	\item Format cinéma : 4/3 format respecté 1.33
	\item Version sonore : VF en stéréo
	\item France --- 1953 --- Noir \& blanc --- 94 min
	\item Sous-titres : aucun
	\item Date de parution : 10 mars 2017
	\item Éditeur : Studio Canal
\end{itemize}

\subsection*{Versions DVD / Blu-ray zone 1}

Les versions zone 1 contiennent des sous-titres en anglais, avec en plus, des interviews des acteurs.
% supprimé car inopérant en pdf et crée un problème en html
%\end{small}
