%%%%%%%%%%%%%%%%%%%%%%%%%%%%%%%%%%%%%%%%%%%%%%%%%%%%%%%%%%%%%%%
% Contents: The glossary entries
% $Id: grisbi-manuel-glossary.tex, v 1.0 2014/02/12 Jean-Luc Duflot
% $Id: grisbi-manuel-glossary.tex, v 3.0 2024/04 Dominique Brochard : update
% $Id: grisbi-manuel-glossary.tex, v 3.0 2024/11 Dominique Brochard :
% - rename file to 30-xxx
% - add Windows and MacOS X entries
%%%%%%%%%%%%%%%%%%%%%%%%%%%%%%%%%%%%%%%%%%%%%%%%%%%%%%%%%%%%%%%% glossary entries file

%example
%%\newglossaryentry{example}{name=example, description={trying a Glossary Entry}}

\newglossaryentry{Unicode control characters}{name=Unicode control characters, description={Many Unicode characters are used to control the interpretation or display of text, but these characters themselves have no visual or spatial representation.  For example, the null character (U+0000 NULL) is used in C-programming application environments to indicate the end of a string of characters}}
\newglossaryentry{encryption}{name=encryption, description={see "to encrypt"}}
\newglossaryentry{to encrypt}{name=to encrypt, description={In cryptography, encrypt a document is the process of encoding information. This process converts the original representation of the information, known as plaintext, into an alternative form known as ciphertext. Ideally, only authorized parties can decipher a ciphertext back to plaintext and access the original information}}
\newglossaryentry{compression}{name=compression, description={Data compression is the process of encoding information using fewer bits than the original representation. Typically, a device that performs data compression is referred to as an encoder, and one that performs the reversal of the process (decompression) as a decoder}}
\newglossaryentry{CSV}{name=CSV, description={(\lang{Comma-Separated Values}) is a text file format that uses commas to separate values, and newlines to separate records}}
\newglossaryentry{CVS}{name=CVS, description={(\lang{Concurrent Versions System}) is a version control system originally developed by Dick \familyname{Grune} in July 1986. CVS operates as a front end to Revision Control System (RCS), an earlier system which operates on single files. It expands upon RCS by adding support for repository-level change tracking, and a client-server model}}
\newglossaryentry{Debian}{name=Debian, description={also known as Debian GNU/Linux, is a Linux distribution composed of free and open-source software and optionally non-free firmware or software developed by the community-supported Debian Project, which was established by Ian \familyname{Murdock}. Debian is also the basis for many other distributions that have different purposes, like Proxmox for servers, Ubuntu or Linux Mint for desktops, Kali for penetration testing, and Pardus and Astra for government use. The word "Debian" was formed as a portmanteau of the first name of his then-girlfriend (later ex-wife) Debra \familyname{Lynn} and his own first name}}
\newglossaryentry{Linux distributions}{name=Linux distributions, description={A Linux distribution, sometimes called a GNU/Linux distribution (often abbreviated as distro) is an operating system made from a software collection that includes the Linux kernel and often a package management system. Linux users usually obtain their operating system by downloading one of the Linux distributions, which are available for a wide variety of systems}}
\newglossaryentry{text editor}{name=text editor, description={A text editor is a type of computer program that edits plain text. Text editors are provided with operating systems and software development packages, and can be used to change files such as configuration files, documentation files and programming language source code. There are important differences between plain text (created and edited by text editors) and rich text (such as that created by word processors or desktop publishing software). Plain text exclusively consists of character representation. Plain text contains no other information about the text itself, not even the character encoding convention employed. Rich text, on the other hand, may contain metadata, character formatting data (e.g. typeface, size, weight and style), paragraph formatting data (e.g. indentation, alignment, letter and word distribution, and space between lines or other paragraphs), and page specification data}}
\newglossaryentry{character encoding}{name=character encoding, description={Character encoding is the process of assigning numbers to graphical characters, especially the written characters of human language, allowing them to be stored, transmitted, and transformed using digital computers. Common examples of character encoding systems include ASCII and Unicode UTF-8}}
\newglossaryentry{extension}{name=extension, description={This is a suffix to the name of a computer file. The extension indicates a characteristic of the file contents or its intended use. A filename extension is typically delimited from the rest of the filename with a period. For example, the readme.txt file is an plain text file}}
\newglossaryentry{file format}{name=file format, description={A file format is a standard way that information is encoded for storage in a computer file. It specifies how bits are used to encode information in a digital storage medium. File formats may be either proprietary or free. One popular method used by many operating systems is to determine the format of a file based on the end of its name, more specifically the letters following the final period. This portion of the filename is known as the filename extension}}
\newglossaryentry{typesetting system}{name=typesetting system, description={is the name given to software intended to lay down the text, leaving the editor focused on his text, and without the display of the result disturbing his creative activity. The result will be obtained later by compilation of the source document in the desired format, for example PDF or PostScript. For example, \LaTeX{} is a software system for typesetting documents, opposed to the formatted text found in WYSIWYG (What You See Is What You Get) word processors like Microsoft Word, LibreOffice Writer and Apple Pages}}
\newglossaryentry{Free and Open Source Software}{name=\lang{Free and Open Source Software}, description={or \lang{FOSS}, is software that is available under a license that grants the right to use, modify, and distribute the software, modified or not, to everyone free of charge. FOSS is in contrast to proprietary software, where the software is under restrictive copyright or licensing and the source code is hidden from the users}}
\newglossaryentry{Free Software Foundation}{name=\lang{Free Software Foundation}, description={or \lang{FSF}, is a non-profit organization founded by Richard \familyname{Stallman} on October 4, 1985, to support the free software movement, with the organization's preference for software being distributed under copyleft ("share alike") terms, such as with its own GNU General Public License}}
\newglossaryentry{Git}{name=Git, description={is a distributed version control system that tracks changes in any set of computer files, usually used for coordinating work among programmers who are collaboratively developing source code during software development. Git was originally authored by Linus \familyname{Torvalds} in 2005 for development of the Linux kernel. Git is a free and open-source software shared under the GPL-2.0-only license}}
\newglossaryentry{GitHub}{name=GitHub, description={is a proprietary developer platform that allows developers to create, store, manage, and share their code. It uses Git to provide distributed version control. It is commonly used to host open source software development projects}}
\newglossaryentry{Gnome}{name=Gnome, description={originally an acronym for \lang{GNU Network Object Model Environment}, is a free and open-source desktop environment for Linux and other Unix-like operating systems}}
\newglossaryentry{Gnucash}{name=Gnucash, description={is an accounting program that implements a double-entry bookkeeping system. It was initially aimed at developing capabilities similar to Intuit, Inc.'s Quicken application, but also has features for small business accounting. Recent development has been focused on adapting to modern desktop support-library requirements}}
\newglossaryentry{GNU Free Documentation License}{name=\lang{GNU Free Documentation License}, description={(\lang{GNU FDL} or simply \lang{GFDL}) is a copyleft license for free documentation, designed by the \lang{Free Software Foundation} (\lang{FSF}) for the GNU Project. It is similar to the GNU General Public License, giving readers the rights to copy, redistribute, and modify (except for "invariant sections") a work and requires all copies and derivatives to be available under the same license. Copies may also be sold commercially, but, if produced in larger quantities (greater than 100), the original document or source code must be made available to the work's recipient}}
\newglossaryentry{GNU General Public License}{name=\lang{GNU General Public License}, description={(\lang{GNU GPL} or simply \lang{GPL}) is a series of widely used free software licenses, or copyleft, that guarantee end users the four freedoms to run, study, share, and modify the software. The license was the first copyleft for general use, and was originally written by Richard \familyname{Stallman}, the founder of the \lang{Free Software Foundation} (\lang{FSF}), for the GNU Project. The last official GPLv3 was released by the FSF on 29 June 2007}}
\newglossaryentry{GNU/Linux}{name=GNU/Linux, description={\lang{GNU}, initialy developed by Richard \familyname{Stallman} in 1984, is an extensive collection of free software, which can be used as an operating system or can be used in parts with other operating systems. GNU is a recursive acronym for ``\lang{GNU's Not UNIX}''. The use of the completed GNU tools led to the family of operating systems popularly known as Linux. Linux is a family of open-source Unix-like operating systems based on the Linux kernel, an operating system kernel first released on September 17, 1991, by Linus \familyname{Torvalds}. Many Linux distributions use the word "Linux" in their name, but the \lang{Free Software Foundation} uses and recommends the name "GNU/Linux" to emphasize the use and importance of GNU software in many distributions}}
\newglossaryentry{Grisbi}{name=Grisbi, description={Grisbi is a free personal accounting program}}
\newglossaryentry{GSB}{name=.gsb, description={is the extension given to Grisbi account files}}
\newglossaryentry{GTK}{name=GTK, description={(\lang{formerly The GIMP Toolkit and GTK+}) is a free software cross-platform widget toolkit for creating Graphical User Interfaces (GUIs)}}
\newglossaryentry{GZ}{name=.gz, description={is the usual extension for files compressed by the free gzip compression software (acronym of GNU zip). Compressed archives are typically created by assembling collections of files into a single tar archive (also called \lang{tarball}), and then compressing that archive with gzip. The final compressed file usually has the extension .tar.gz or .tgz. Since the late 1990s, bzip2, a file compression utility based on a block-sorting algorithm, has gained some popularity as a gzip replacement. It produces considerably smaller files (especially for source code and other structured text), but at the cost of memory and processing time (up to a factor of 4)}}
\newglossaryentry{HTML}{name=HTML, description={\lang{HyperText Markup Language} or \lang{HTML} is the standard markup language for documents designed to be displayed in a web browser.  HTML provides a means to create structured documents by denoting structural semantics for text such as headings, paragraphs, lists, links, quotes, and other items}}
\newglossaryentry{IBAN}{name=IBAN, description={\lang{International Bank Account Number} uniquely identifies the account of a customer at a financial institution}}
\newglossaryentry{KDE}{name=KDE, description={Since 2009, the name KDE no longer stands for \lang{K Desktop Environment}, but for the community that produces the software. KDE is an international free software community that develops free and open-source software. As a central development hub, it provides tools and resources that allow collaborative work on this kind of software. Well-known products include the Plasma Desktop, KDE Frameworks, and a range of cross-platform applications such as Amarok, digiKam, and Krita that are designed to run on Unix and Unix-like operating systems, Microsoft Windows, and Android}}
\newglossaryentry{C}{name=C, description={C is a general-purpose computer programming language which  was originally developed by Dennis \familyname{Ritchie} between 1972 and 1973 to construct utilities running on Unix. Many later languages have borrowed directly or indirectly from C, including C++, C\#, Unix's C shell, Java, Perl, PHP, and many others}}
\newglossaryentry{LaTeX}{name=\LaTeX, description={is a software system for typesetting documents. \LaTeX{} uses the TeX typesetting program for formatting its output, and is itself written in the TeX macro language. Its name comes from its author Leslie \familyname{Lamport}}}
\newglossaryentry{hyperlinks}{name=hyperlinks, description={In computing, a hyperlink, or simply a link, is a digital reference to data that the user can follow or be guided to by clicking or tapping. For instance, on the \lang{World Wide Web} most hyperlinks cause the target document to replace the document being displayed, but some are marked to cause the target document to open in a new window (or, perhaps, in a new tab)}}
\newglossaryentry{locale}{name=locale, description={In computing, a locale is a set of parameters that defines the user's language, region and any special variant preferences that the user wants to see in their user interface. Usually a locale identifier consists of at least a language code and a country/region code}}
\newglossaryentry{free software}{name=free software, description={\lang{Free software}, \lang{libre software}, or \lang{libreware} is computer software distributed under terms that allow users to run the software for any purpose as well as to study, change, and distribute it and any adapted versions.Un logiciel libre est un logiciel dont l'utilisation, l'étude, la modification et la duplication en vue de sa diffusion sont possibles techniquement et permises légalement. These rights are usually defined by a license. For instance, Grisbi is licensed under the \lang{GPL} (\lang{General Public License})}}
\newglossaryentry{input method}{name=input method, description={is an operating system component or program that enables users to generate characters not natively available on their input devices by using sequences of characters (or mouse operations) that are available to them}}
\newglossaryentry{OFX}{name=OFX, description={\lang{Open Financial Exchange} is a data-stream format for exchanging financial information (XML based) (that evolved from Microsoft's Open Financial Connectivity (OFC)}}
\newglossaryentry{partition}{name=partition, description={Disk partitioning is the creation of one or more regions on secondary storage, so that each region can be managed separately. These regions are called partitions. Each partition then appears to the operating system as a distinct "logical" disk that uses part of the actual disk. Partitioning allows the use of different filesystems to be installed for different kinds of files. The best known filesystems are FAT32, NTFS, Ext2, Ext3 and Ext4}}
\newglossaryentry{PDF}{name=PDF, description={\lang{Portable Document Format} is a file format developed by Adobe in 1992 to present documents, including text formatting and images, in a manner independent of application software, hardware, and operating systems}}
\newglossaryentry{PNG}{name=PNG, description={\lang{Portable Network Graphics} is a raster-graphics file format that supports lossless data compression. PNG was developed as an improved, non-patented replacement for Graphics Interchange Format (GIF)—unofficially, the initials PNG stood for the recursive acronym "PNG's not GIF"}}
\newglossaryentry{porting}{name=porting, description={In software engineering, porting is the process of adapting software for the purpose of achieving some form of execution in a computing environment that is different from the one that a given program (meant for such execution) was originally designed for}}
\newglossaryentry{PostScript}{name=PostScript, description={PostScript (often abbreviated as PS) is a page description language and dynamically typed, stack-based programming language. It is most commonly used in the electronic publishing and desktop publishing realm, but as a Turing complete programming language, it can be used for many other purposes as well. PostScript was created by Adobe Systems}}
\newglossaryentry{QIF}{name=QIF, description={\lang{Quicken Interchange Format} is an open specification for reading and writing financial data to media}}
\newglossaryentry{RedHat}{name=Red Hat, description={Red Hat, Inc. (formerly Red Hat Software, Inc.) is an American software company that provides open source software products to enterprises and is a subsidiary of IBM. Red Hat creates, maintains, and contributes to many free software projects}}
\newglossaryentry{Slackware}{name=Slackware, description={is a Linux distribution created by Patrick \familyname{Volkerding} in 1993. Originally based on Softlanding Linux System (SLS), Slackware has been the basis for many other Linux distributions, most notably the first versions of SUSE Linux distributions, and is the oldest distribution that is still maintained. Slackware aims for design stability and simplicity and to be the most "Unix-like" Linux distribution}}
\newglossaryentry{SVG}{name=SVG, description={\lang{Scalable Vector Graphics} is an XML-based vector image format for defining two-dimensional graphics, having support for interactivity and animation}}
\newglossaryentry{sorting}{name=sorting, description={In computer science, a sorting algorithm is an algorithm that puts elements of a list into an order}}
\newglossaryentry{primary sorting}{name=primary sorting, description={is the term used in Grisbi to designate a sorting based on a primary sorting key, that is to say a first-level sorting. If there is only primary sorting without secondary sorting, it is a simple sorting, based on a single sorting criterion}}
\newglossaryentry{secondary sorting}{name=secondary sorting, description={is the term used in Grisbi to designate a sort based on a secondary sort key, i.e. a second-level sort. For example, you can do a primary sort on the value date, and a secondary sort on the payee}}
\newglossaryentry{URL}{name=URL, description={A \lang{Uniform Resource Locator}, colloquially known as an "address" on the Web, is a reference to a resource that specifies its location on a computer network and a mechanism for retrieving it. A URL is a specific type of Uniform Resource Identifier (URI) which is a unique sequence of characters that identifies an abstract or physical resource, such as resources on a webpage, mail address, phone number, books, real-world objects such as people and places, concepts}}
\newglossaryentry{UTF-8}{name=UTF-8, description={is a variable-length character encoding standard used for electronic communication. Defined by the Unicode Standard, the name is derived from Unicode Transformation Format – 8-bit. It was designed for backward compatibility with ASCII, limited to 95 printable characters}}
\newglossaryentry{XML}{name=XML, description={\lang{eXtensible Markup Language} is a markup language and file format for storing, transmitting, and reconstructing arbitrary data, inherited from SGML}}
\newglossaryentry{macOS}{name=macOS, description={originally Mac OS X, previously shortened as OS X, is a partially proprietary operating system developed and marketed by Apple since 1998, whose macOS Sequoia version (version 15) was launched on September 16, 2024 for the general public}}
\newglossaryentry{Windows}{name=Windows, description={(littéralement « Fenêtres » en anglais) is a range of proprietary operating systems developed by Microsoft. The first version of Windows, in 1985, was simply a graphical interface for MS-DOS used on IBM computers. Versions 2, 3, 95 (released in 1995 and sold pre-installed on almost all personal computers, due to the many exclusive agreements signed with computer manufacturers prohibiting them from installing another system on pain of financial penalties), XP, Vista, 7, 8 and 10 followed. Version 11 is the current version in 2024}}

%% end of glossary entries
