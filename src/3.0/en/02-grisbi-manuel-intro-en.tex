%%%%%%%%%%%%%%%%%%%%%%%%%%%%%%%%%%%%%%%%%%%%%%%%%% %%%%%%%%%%%%%%%

% Contents: The introduction chapter

% $ Id: graybi-manual-intro.tex, v 0.4 2002/10/27 Daniel Cartron

% $ Id: gray-manual-intro.tex, v 0.5.0 2004/06/01 Loic Breilloux

% $ Id: gray-manual-intro.tex, v 0.6.0 2011/11/17 Jean-Luc Duflot

% $ Id: gray-manual-intro.tex, v 0.8.9 2012/04/27 Jean-Luc Duflot

% $ Id: gray-manual-intro.tex, v 1.0 2014/02/12 Jean-Luc Duflot

%%%%%%%%%%%%%%%%%%%%%%%%%%%%%%%%%%%%%%%%%%%%%%%%%% %%%%%%%%%%%%%%%

\chapter{Introduction\label{introduction}}

\section{General \label{introduction-General}}

Grisbi is a free accounting program, developed in the \gls{C} language with support for \gls{GTK}+ 3, originally for the \gls{GNU/Linux} platform. There is now a \indexword{\gls{porting}}\index{porting} on \gls{Windows}, \gls{macOS}, FreeBSD, packages for different  \gls{Linux distributions}, and other possibilities to discover on \lang{Grisbi}\footnote{\urlGrisbi{}} or \lang{Sourceforge}\footnote{\urlSourceForge{}}.

The basic principle is to allow you to classify in a simple and intuitive way your financial operations, whatever they are, in order to be able to exploit them in the best way according to your needs.

Grisbi emphasises simplicity and efficiency, without excluding the sophistication needed by more advanced users. Future features will always try to meet these criteria.

\section{Features\label{introduction-features}}

\strong{Translators Note:} In this translation, the French word ``Tiers'' in general has been replaced by the English word "Payees", however the French version of Grisbi also caters for small business use. In the French version any third party i.e. a creditor or debtor could be implied by the word ``Tiers''.\\
\textbf{Note:} French business accounting rules may be incompatible with the rules of any English speaking territories, Grisbi should be used \strong{only} for personal financial record keeping outside of France.



\subsection{What Grisbi knows how to do}

\begin{itemize}
	\item Software developed by French programmers, so in full compliance with the logic of French accounting
	\item Simple and intuitive interface, with full-screen display control
	\item Multi-account and multi-user management
	\item Local Settings Management (Dates, Decimal Separators and Thousands)
	\item Bank accounts, cash, assets and liabilities
	\item Multi-currency management, with support for exchange rates and exchange fees
	\item Credit card management (immediate or deferred debit)
	\item Description of transactions with: date, valuation date, fiscal year, Payee amount, currency, category and sub-category, budget allocation and sub-allocation (for analysis of expenditure), note, entry number (assigned by Grisbi), part number, account number, bank reference
	\item Budget classifications and attribution with automatic recall of operations and sub-operations for a given Payee
	\item Clear Credit and Flow fields option for auto-complete
	\item Calculation of the balance according to the date of the transaction or its valuation date
	\item Transfer between accounts, including different currencies, with automatic cross posting
	\item Scheduling operations with automatic or manual validation
	\item Deadline monitoring
	\item Moving and Cloning Operations
	\item Cloning scheduled operations
	\item Shading of different periods in the scheduler
	\item Analysis and financial reports thanks to the powerful report templates module
	\item Several pre-formatted report templates available and customizable
	\item Virtuals Payees created by reports
	\item Printing of reports
	\item Simulation of loans and depreciation tables with printing and data export
	\item Estimated budget with charts on forecasts and historical data
	\item Business Accounting with Chart of Accounts (for French Accounting)
	\item Account order by drag and drop in the account list
	\item Import files in the following formats: \gls{QIF}, \gls{OFX}, \gls{Gnucash} or \gls{CSV}
	\item Importing category files into budget allocations
	\item Export files in the \gls{QIF} or \gls{CSV} formats
	\item Choice of custom icons for your different accounts
	\item Icons in the \indexword{\gls{SVG}}\index{SVG} formats
	\item Even more context menus on the right mouse button (navigation pane)
	\item Numerous keyboard shortcuts for good ergonomics
\end{itemize}

\subsection{What Grisbi does not know yet}

\begin{itemize}
	\item Automatic breakdown of loan repayments
	\item Internet reconciliation
\end{itemize}

\section{Software Evolution}

\subsection{Development and versions}

Grisbi is software in active development, any feedback (idea, bug, documentation\dots{}) is welcome. You can send it to the relevant lists referenced in the  \vref{introduction-contacts} \menu{Contacts} section or on the  \lang{Grisbi}\footnote{\urlGrisbi{}} site.


%You can, if you have the taste of adventure, download the versions in development on the Grisbi content management system based on \gls{Git}.
If you're feeling adventurous, you can download and compile the latest version being under development on the content management system \gls{GitHub}\footnote{\urlGitHubGrisbi{}\label{siteGitHubGrisbi}} using \gls{Git}.

Indeed, the code of a new versions of Grisbi is typically frozen several weeks before its final release, in order to allow the development team time to check and eradicate the latest bugs. During this period, the format of the account file no longer changes, and you can, with a minimum of precautions (frequent backups, etc.), benefit from the latest improvements several weeks in advance, and also participate in debugging.

%Finally you also have access to all the evolutions of the code after version 0.3.2 (the version from which we set up a \gls{CVS} on the Grisbi site.
Finally, you can access all changes to the code since version 0.3.2 (the version from which the code is available on the \gls{GitHub}\footref{siteGitHubGrisbi} website).

Note that as of version 0.6, even numbered minor number versions (e.g. 0. \underline8) are stable versions, whereas odd number minor number versions are unstable and should not be used under normal conditions; therefore only stable versions are mentioned here.

% page break for solidarity title

%\newpage
\subsection{Version 0.1 \textnormal{(10/04/2000)}}

\begin{itemize}
	\item Management of several accounts%Gestion de plusieurs comptes
	\item Creation, modification and deletion of accounts and transactions (the minimum required to operate  \dots{})%Création, modification et suppression de comptes et d'opérations (enfin le minimum pour fonctionner \dots{})
	\item Possibility of pointing operations%Possibilité de pointer les opérations
	\item And backup, without which none of this would be very useful!%Et la sauvegarde, sans laquelle tout cela n'aurait pas beaucoup d'intérêt !
\end{itemize}

\subsection{Version 0.2 \textnormal{(19/06/2000)}}

\begin{itemize}
	\item Automatic import of account files from previous versions%Importation automatique des fichiers de comptes de versions précédentes
	\item Use of lists for third parties and categories \dots{} (more practical than typing everything)%Utilisation de listes pour les tiers, catégories \dots{} (plus pratique que de tout taper)
	\item Automatic entry when typing and filling in of the end of the transaction like the previous version%Saisie automatique lors de la frappe et remplissage de la fin de l'opération comme la précédente semblable
	\item Management of transfers between accounts%Gestion des virements entre comptes
	\item Simplified or full display of transactions%Affichage des opération simplifié ou complet
	\item Account balancing%Équilibrage de comptes
\end{itemize}

\subsection{From version 0.3.0 \textnormal{(10/12/2001)} to version 0.3.3 \textnormal{(15/11/2002)}}%De la version 0.3.0 à la version 0.3.3}

\begin{itemize}
	\item Management of planned or cyclical operations (schedule)%Gestion des opérations prévues ou cycliques (échéancier) 
	\item Support for currencies and the changeover to the euro (from 1 January 1999 to 1 January 2002)%Prise en charge des devises et du passage à l'euro (du 1er janvier 1999 au 1er janvier 2002)
	\item Import/export added and file import improved \gls{QIF}%Ajout de l'import/export et amélioration de l'import de fichiers \gls{QIF}
	\item Added breakdown%Ajout de la ventilation
	\item Multi-user support%Support multi-utilisateur
	\item For each bank and account, added details such as:%Pour chaque banque et compte, ajout des détails comme:
		\begin{itemize}
			\item[\textopenbullet] full numbers of the associated account, branch and bank code%numéros complets du compte associé, du guichet et du code banque
			\item[\textopenbullet] different holder for each account with home address%titulaire différent pour chaque compte avec adresse personnelle
			\item[\textopenbullet] contact details for a correspondent%coordonnées d'un correspondant
		\end{itemize}
	\item Categories tab added%Ajout de l'onglet catégories
	\item Ability to change currency names and codes%Possibilité de changer le nom et le code des devises
	\item Alphabetical sorting now (finally) takes accents into account%Le classement alphabétique prend (enfin) en compte les accents
	\item Memorises the last working directory and the last files opened% Mémorise le dernier répertoire de travail et les derniers fichiers ouverts
	\item A single number per transaction, whatever the account%Un numéro unique par opération quel que soit le compte
	\item The \key{+} (\key{-}) keys in a date field increment (decrement) the date%Les touches \key{+} (\key{-}) dans un champ de date incrémentent (décrémentent) la date
	\item New fields can be displayed in the transaction entry form%Affichage possible de nouveaux champs dans le formulaire de saisie des opérations 
	\item Deadlines can be set for a defined period of time%Échéances paramétrables pour un laps de temps défini
	\item Display of the current balance%Affichage du solde courant pointé
	\item Possibility of entering the date in the form ddmm, ddmmyyy or ddmmyyyy%Possibilité d'entrer la date sous la forme jjmm, jjmmaa ou jjmmaaaa
	\item Transaction totals in the Payees, Categories and Budget Allocations tabs%Totaux des opérations dans les onglets Tiers, Catégories et Imputations Budgétaires (IB)
	\item The logo and display font can be customised%Personnalisation possible du logo et de la police d'affichage.
	\item Automake and Autoconf tools to simplify compilation of software source files%Mise en place des outils Automake et Autoconf permettant de simplifier la compilation des fichiers sources du logiciel
\end{itemize}

\subsection{From version 0.4.0 \textnormal{(15/02/2003)} to version 0.4.4 \textnormal{(25/03/2004)}}

\begin{itemize}
	\item Customisable interface layout%Personnalisation de l'agencement de l'interface
	\item Import/export capability:%Possibilité d'import/export:
		\begin{itemize}
			\item[\textopenbullet] lists of categories%de listes de catégories
			\item[\textopenbullet] lists of budget items%de listes d'imputations budgétaires (IB)
			\item[\textopenbullet] created reports%d'états créés
		\end{itemize}	
	\item Financial reports added%Ajout des rapports financiers
	\item Grisbi is now internationalised and translations have been improved%Grisbi est désormais internationalisé et les traductions sont améliorées
	\item Display of multi-currency balances in the start-up screen%Affichage des soldes multi-devises dans l'écran de démarrage
	\item Remarks can now be displayed in the schedule%Les remarques peuvent être affichées dans l'échéancier
	\item New tool to allow contributors to anonymise Grisbi files to maintain confidentiality before submission%Nouvel outil permettant aux contributeurs d'anonymiser les fichiers Grisbi afin de garder la confidentialité avant de les envoyer
	\item The width of the columns in the schedule can be modified%La largeur des colonnes de l'échéancier est modifiable
	\item Harmonisation of date fields:%Harmonisation dans les champs de date:
		\begin{itemize}
			\item[\textopenbullet] \key{Ctrl}\key{Enter} in a date field opens a calendar%\key{Ctrl}\key{Entrée} dans un champ de date ouvre un calendrier
			\item[\textopenbullet] the arrow keys are active in a calendar%les touches fléchées sont actives dans un calendrier
			\item[\textopenbullet] \key{Ctrl}\key{+} (\key{-}) in a date field increases (decreases) the date by approximately one week%\key{Ctrl}\key{+} (\key{-}) dans un champ de date augmente (diminue) la date d'environ une semaine
			\item[\textopenbullet] \key{Pg.Up} (\key{Pg.Dn}) increases (decreases) the date by approximately one month%\key{Pg.Préc} (\key{Pg.Suiv}) augmente (diminue) la date d'environ un mois
			\item[\textopenbullet] \key{Ctrl}\key{Pg.Up} (\key{Pg.Dn}) increases (decreases) the date by approximately one year%\key{Ctrl}\key{Pg.Préc} (\key{Pg.Suiv}) augmente (diminue) la date d'environ un an
		\end{itemize}	
\end{itemize}

\subsection{From version 0.4.5 \textnormal{(07/04/2004)} to version 0.6 \textnormal{(05/05/2010)}}

\begin{itemize}
	\item Revamped GUI
	\item \gls{GTK}+ 2 library support for a nicer environment and simplified porting on Windows
	\item No more dependencies on \gls{Gnome}
	\item Native Windows version (thanks to François \familyname{Terrot})
	\item Native \gls{UTF-8} support 
	\item Printing reports by \gls{LaTeX}
	\item Export reports in \gls{HTML}
	\item Improved user interface:
		\begin{itemize}
			\item[\textopenbullet] message enhancement, users can select ``ignore''
			\item[\textopenbullet] improved management of segmentation errors
			\item[\textopenbullet] improvement of the preferences window
			\item[\textopenbullet] pop-up menu on the list of operations
			\item[\textopenbullet] improving items
		\end{itemize}
	\item Completely clickable list of accounts with summaries
	\item Global configuration in \gls{XML} (thanks to Axel \familyname{Rousseau})
	\item Rewrite of the file import:
		\begin{itemize}
			\item[\textopenbullet] support for the \gls{QIF}, \gls{OFX}, \gls{Gnucash} or \gls{CSV}  formats
			\item[\textopenbullet] incremental import
			\item[\textopenbullet] automatic reconciliation
		\end{itemize}
	\item Beginning of the Italian translation by Giorgio \familyname{Mandolfo}
	\item Hidden exchange rates on a session to avoid re-entering them
	\item Support for text attributes (bold, italic, large, small) for states
	\item New logo with mascot Grisbi on a Euro sign (€) (thanks to André \familyname{Pascual})
	\item Editable animated waiting logo
	\item Completely case-sensitive lists of characters
	\item Improved entry of bank details
	\item Keyboard support in the tree structure of payees, categories and budget charges
	\item Automatic clickable term maturities
	\item Breakdown of maturities
	\item Transactions convertible into maturities
	\item Transactions that can be moved to another account
\end{itemize}

\subsection{What's new in version 0.8 \textnormal{(21/02/2011)}}

\begin{itemize}
	\item Budget module in the basic version
	\item Credit simulator and amortization tables with the possibility to print and export the data in a spreadsheet
	\item Amortization table for liability accounts with the ability to print and export data in a spreadsheet
	\item Local Settings Management (Date Format, Decimal Separators and Thousands)
	\item Embedding custom icons in the accounts file
	\item Shading of different periods in the scheduler
	\item Cloning scheduled operations
\end{itemize}

\subsection{What's new in version 1.0 \textnormal{(17/03/2014)}}

\begin{itemize}%\itemsep=-3pt
	\item Graphics on forecasts
	\item Credit card management (immediate or deferred debit)
	\item Business accounting with a Chart of Accounts
	\item Grisbi icon in the \indexword{ \gls{SVG}}\index{SVG} formats 
	\item Even more context menus on the right mouse button
	\item Importing category files into budget allocations
	\item Changing the account display order by drag and drop in the account list
	\item Calculation of the balance according to the date of the transaction or its valuation date
	\item Option to delete credit and debit fields for auto-completion
	\item View unused payees
	\item Full-screen display command by function key \key{F11}
	\item Keyboard shortcut \key{Ctrl} \key{T} for the call of a new operation
	\item Direct access to the user manual through the menu \menu{Help} or the keyboard shortcut \key{Ctrl} \key{H}
\end{itemize}

\subsection{What's new in version 2.0 \textnormal{(11/01/2021)}}

\begin{itemize}
	\item Ported to \gls{GTK}+ 3 (since version 1.2.0)%Portage sous \gls{GTK}+ 3 (depuis la version 1.2.0)
	\item Addition of a search module in the list of accessible operations, in the operations context menu (right-click on an operation)%Ajout d'un module de recherche dans la liste des opérations accessibles, dans le menu contextuel des opérations (clic droit sur une opération)
	\item Automatic detection/integration of dark schematics%Détection automatique/intégration de schéma de couleur foncée
	\item New colour settings%Nouveaux réglages pour les couleurs
	\item General font size configuration%Configuration générale de la taille de police
	\item Improved display on low-resolution devices%Affichage amélioré sur les dispositifs à faible résolution
	\item Revised credit module%Révision du module de crédit
	\item Import rules for \gls{CVS} files%Règles d'importation pour les fichiers \gls{CSV}
	\item Search functionality%Fonctionnalité de recherche
	\item Setting for deleting old backups%Réglage pour la suppression d'anciennes sauvegardes
	\item Bug fixes%Résolution de bogues
\end{itemize}

\subsection{What's new in version 3.0 \textnormal{(13/11/2023)}}

\begin{itemize}
	\item Modification of the beneficiary search%Modification de la recherche du bénéficiaire
	\item Addition of a new type of consumer loan%Ajout d'un nouveau type de prêt à la consommation
	\item All transactions added to archives when account is closed%Ajout de toutes les transactions dans les archives lorsque le compte est clôturé
	\item Code clean-up%Nettoyage du code
	\item Preparing the transition to \gls{GTK} 4%Préparation de la transition vers \gls{GTK} 4
	\item Bug fixes%Résolution de bogues
\end{itemize}

\subsection{And for the future?}

Full port under \gls{GTK} 4

\section{Contacts\label{introduction-contacts}}

In addition to emailing the authors, you have several mailing lists you can contact us or obtain information through.

To keep abreast of developments in Grisbi, you can register on the \lang{Information list}\footnote{\urlListInfoEmail{}} provided for this purpose.  You will then receive an email at the release of each new version.

%If you want to participate in the development of Grisbi, there is a \lang{development list}\footnote{\urlListDevelEmail{}}.
If you'd like to get involved in Grisbi's development, there's a \lang{development list}\footnote{\urlListDevelEmail{}}.

%If you want to regularly compile the latest version of Grisbi's code from \gls{Git}, you will find it advantageous to subscribe to the \lang{Git information list}\footnote{\urlListCVSEmail{}} to be notified of the latest \cmd{commit}.

%In addition, we decided to undertake the internationalization of Grisbi and, if you wish to help us, you can first contact us on the development list.
We've also decided to take Grisbi international, and if you'd like to help us, you can get in touch with us on the development list.

%Specialized lists will be created for each language as needed.

%For now, only the \lang{English translation list}\footnote{\urlListAnglaiseEmail{}} exists.

%To subscribe to one of these lists, simply go to the page \urlListSF{}\emph{list} replacing \emph{list} with the name of the list you are interested in. Or go to \lang{Grisbi}\footnote{\urlGrisbi{}} and look for \cmd{Contacts} there.
To subscribe to one of these lists, simply go to the \urlListDiffGrisbi{} page and click on the list(s) you are interested in.

You can also use the discussion forums (or newsgroups) with a program called ``news reader'' (like Thunderbird for example) by entering \cmd{listes.grisbi.org} as the group server name.

In addition do not hesitate to regularly consult Grisbi's official website.

% page break for solidarity title

%\newpage

\section{Authors and contributors\label{introduction-authors}} %TODO section to be updated/completed

Cédric \familyname{Auger} is the basis of the project.

Daniel \familyname{Cartron} wrote the documentation up to version 0.4.0, provided accounting advice and ergonomics, and created the first Grisbi site. His passion for ultra-compliant accounts files brings an undeniable bonus to the discovery of unpublished bugs.

André \familyname{Pascual}, from \lang{Linuxgraphic}\footnote{\urlLinuxGraphic{}}, is the author of the first logo (Grisbi mascot on the euro symbol €).

Sébastien \familyname{Blondeel} wrote the scripts used to produce the various documentation formats and those for converting images to the appropriate formats. He is also responsible for the adoption of \gls{LaTeX} for writing the documentation. In addition, his experience of electronic publishing makes him a valuable adviser and source of many suggestions.

Benjamin \familyname{Drieu}, developer for Grisbi and official packager for \gls{Debian}.

Alain \familyname{Portal}, who was starting to get bored in \gls{RedHat} packaging. His love of a job well done and his obstinacy make him, for the moment, a bug fixer. He also participates in the compilation of the documentation. He wants to start coding in the unstable version.

Loic \familyname{Breilloux} has updated the documentation for version 0.5.1 and will try to update the documentation for future releases.

Gerald \familyname{Niel} replaced Daniel \familyname{Cartron} in the role of webmaster and is therefore the creator of the new version of \lang{Grisbi}\footnote{\urlGrisbi{}}. He is also responsible for \gls{Slackware} packages.

Juliette \familyname{Martin} has the thankless task of proofreading of the documentation. If there are any mistakes, it's certainly they were well hidden for escaping his attentive eyes

François \familyname{Terrot}\footnote{\urlFrancoisTerrotEmail{}} joined the team to create Grisbi's \gls{porting} for Windows.

{Pierre \familyname{Biava}}\footnote{\urlPierreBiavaEmail{}} joined the development team in 2008.

Didier\familyname{Chevalier}\footnote{\urlDidierChevalierEmail{}},William \familyname{Ollivier}\footnote{\urlWilliamOllivierEmail{}} and Mickael \familyname{Remars} have also participated in development.

{Jean-Luc \familyname{Duflot}}\footnote{\urlJeanLucDuflotEmail{}} made a big update of the manual for the 0.6 version, which was needed since 2004, and continued on with the 0.8 and the 1.0 versions too.

Alain \familyname{Letient}\footnote{\urlAlainLetientEmail{}} tenaciously re-read the 0.6 manual and created its iconography, and also continued with versions 0.8 and 1.0.

Guy \familyname{Lebègue}, first for version 0.8, then with {Michèle\familyname{Bondil}}\footnote{\urlMicheleBondilEmail{}} for 1.0, created the business accounting option, which requires many specialist accounting skills.

\section{Acknowledgments \label{introduction-thanks}}

Thanks to \lang{TuxFamily}\footnote{\urlTuxFamily{}} who has long made available to us all the tools we needed to develop Grisbi 
(website, ftp, CVS, mailing lists, etc.). Alas, the attacks inflicted by hackers in late 2003 - early 2004 on \lang{TuxFamily} have forced us to seek a new web home. So today we thank \lang{SourceForge}\footnote{\urlSourceForge{}}, the platform to which we migrated.  We wish a quick and quick recovery to \lang{TuxFamily} that is sorely lacking hundreds of free projects.

A big thank you also to all the contributors on the development list who helped Grisbi's evolution through their suggestions, remarks and bug reports, as well as to the many readers of the \menu{User Manual}, which contribute to make it a better tool.

\section{Licenses \label{introduction-licenses}}

The program is subject to the terms of the \gls{GNU General Public License}.  Bug fixes and updates are not guaranteed.

The manual is subject to the terms of the \gls{GNU Free Documentation License}.

Permission is granted to copy, distribute and / or modify this document under the terms of the GNU Free Documentation License Version 1.1 or any later version published by the Free Software Foundation.

\section{About this manual \label{introduction-manual}}

%This is version \actuality{} 1.0 of the manual, dated \actuality{} February 12, 2014, which corresponds to the version 1.0 of Grisbi software.  This manual was written with the \gls{LaTeX} \index{latex @ \LaTeX} \gls{text formatter}.
% and is available in PDF or HTM \gls{file formats}}, with or without illustrations ( screenshots) in these two formats.
You are looking at version \actuality{}3.0 of the manual, dated \actuality{} early 2025, which corresponds to version 3.0 of the Grisbi software.

\vspacepdf{5mm}

\textbf{Note}: this version 3.0 of the manual is a mixture of an update of the English version 1.0 and a translation of the French version 3.0.

\vspacepdf{5mm}

This manual was written using the \gls{text editor} \gls{LaTeX}, and is available in \gls{file format} \gls{PDF}\index{PDF} or \gls{HTML}\index{HTML}, with illustrations (screenshots) in both formats.

\vspacepdf{5mm}

It can be accessed directly in the Grisbi software via the \menu{Help - Manual} menu on the menu bar, in \gls{HTML} format with illustrations.

\vspacepdf{5mm}

However, all these different formats can be downloaded from the \lang{Sourceforge}\footnote{\urlSourceForgeDocumentation{}} site, as well as the corresponding versions of the software on \lang{Sourceforge}\footnote{\urlSourceForge{}} in the ``\textsf{grisbi stable}'' or ``\textsf{grisbi unstable}'' folders.

\vspacepdf{5mm}

The tools needed to read the various manual formats are presented in the  \vref{introduction-manual-readers} section \menu{Reading software}.


\subsection{Introduction \label{introduction-manual-presentation}}

Although Grisbi is designed to be as intuitive as possible and most functions are immediately understandable, a reference manual is needed. This manual has been designed according to the following principles:

\begin{itemize}
	\item the most comprehensive possible, covering all the program features;
	\item chapters organised according to a framework that is as standardised as possible:
		\begin{itemize}
			\item[\textopenbullet] presentation of the chapter
			\item[\textopenbullet] description of the display
			\item[\textopenbullet] description of the functions
		\end{itemize}
		to help you find your way around the document	 
	\item paragraphs that recur from one chapter to the next are written in as identical a manner as possible, to make them easier to read quickly
	\item easy to find information thanks to numerous \gls{hyperlinks}, an index and a glossary.
\end{itemize}

% space for theme change: 5 mm

\vspacepdf{5mm}

Here is a brief description of the different chapters:

\begin{itemize}
	\item \menu{Preamble} Introduces the English translation and explains the origin of the name given to this software;
	\item \menu{Introduction} Introduces the software, the manual, their authors and contacts;
	\item \menu{First start Grisbi} is the \emph{essential} chapter to help you to start using the software;
	\item \menu{Home} describes the main elements of the GUI and their manipulation with the mouse and the keyboard (shortcuts);
	\item \menu{Export and Import Accounts} describes how to exchange data with other software;
	\item \menu{Data Management} presents the options of the account files, backups and archives and their management;
	\item \menu{Accounts Management} describes the properties of accounts, their management and the different types of accounts with their use;
	\item \menu{Account Operations}  describes how transactions are handled, the information and input fields used and how they are managed, and how transactions are created and managed;
	\item \menu{Bank Reconciliation} details the procedure for reconciling an account and managing reconciliations;
	\item \menu{Schedule} describes the planning of future entries and their manipulation;
	\item \menu{Searches} takes stock of data search possibilities;
	\item \menu{Payees}, \menu{Categories}, \menu{budgetary allocations} and \menu{Exercises} describe the management of this data;
	\item \menu{Credit Simulation} describes methods and simulation management;
	\item \menu{Forecast Budgets} describes the tools and procedures for budget creation and depreciation tables, as well as their management;
	\item \menu{Credit card management and their prediction} describes the management of these cards, including deferred debit cards, and forecasting methods;
	\item \menu{Business Accounting} presents two introductions for business managers (for French only)
	\item \menu{Reports} and \menu{Report creation} describe the management and creation of reports;
	\item \menu{Configuration of Grisbi} details all the possibilities of setting the software;
	\item \menu{Maintenance Tools} gives some tips to use in case of errors or bugs.
\end{itemize}

\subsection{Typographical conventions of this manual \label{introduction-manual-conventions}}

The following list defines and illustrates the typographic conventions used as visual aids for identifying particular elements of the text of the document:

\begin{itemize}
	\item the interface components are window titles, icon and button names, menu names, and other options that appear on the screen; they are presented as follows:
		\newline
		\hspace*{1.5cm} click \menu{Back}
	\item the keyboard key label represents what is written on the keyboard keys; it is presented as follows:
		\newline
		\hspace*{1.5cm} press the \key{Enter} key;
	\item key combinations are a series of keys to be pressed simultaneously (unless otherwise specified) to perform a single function; they are presented as follows:
		\newline
		\hspace*{1.5cm} press the key combination \key{Ctrl} \key{R}
	\item The commands that are part of an instruction and that must be entered are presented as follows:
		\newline
		\hspace*{1.5cm} type \cmd{grisbi} to start the program
	\item file and directory names are shown as:
		\newline
		\hspace*{1.5cm} \file{grisbi-n.n.n.rpm} and \file{/usr/local/bin}
	\item the command lines consist of a command and can include one or more possible parameters of the command; they are presented as follows:
		\newline
		\hspace*{1.5cm} \cmd{rpm -Uvh grisbi-n.n.n.rpm}
	\item any sequence of alphanumeric characters in blue, in the document in PDF or HTML format, is a hypertext link, referring to either an image, another part of the document, an indexed word or to the glossary (for PDF only);
	\item The words or groups of words referenced in the index are highlighted in the chapters as follows:
		\newline
% TODO: to be updated
		\hspace*{1.5cm} \textopenbullet{} \textsf{referenced term} for \gls{PDF} format
		\newline
		\hspace*{1.5cm} \textopenbullet{} in brown for \gls{HTML}
% END_TODO
	\item \textbf{Note}: underlines a particular point to be taken into account
	\item \textcolor{red}{\strong{Warning}}: indicates either a point that is very important for understanding, or an error that must not be made on pain of serious damage to your data; a \textcolor{red}{\strong{Warning}} is \emph{imperatively to be respected}.

\end{itemize}

% space before Attention or Note: 5 mm

\vspacepdf{5mm}

%In addition, a \textbf{Note} underlines a particular point to take into account, while a \strong{Attention} indicates either a very important point for the understanding, or an error not to do under pain of important damage for your data ; a \strong{Warning} is \emph{to be respected}.

%In addition, a \textbf{Note} underlines a particular point to be taken into account, while a \textcolor{red}{\strong{Warning}} indicates either a point that is very important for understanding, or an error that must not be made on pain of serious damage to your data; a \textcolor{red}{\strong{Warning}} is \emph{imperatively to be respected}.

\subsection{Reading software \label{introduction-manual-readers}}

To read this document, we recommend the use of free software, which respects all your privacy and the confidentiality of your data; the following software has the features of \gls{hyperlinks}:

\begin{itemize}
	\item for \gls{PDF}\index{PDF} format:
		\begin{itemize}
			\item[\textopenbullet] Linux: Evince, Firefox, Xpdf, Ghostscript\textsuperscript{\textcolor{red}{\textbf{*}}}, MuPDF\textsuperscript{\textcolor{red}{\textbf{*}}}, Okular
			\item[\textopenbullet] Mac: Okular, Skim, Xpdf
			\item[\textopenbullet] Windows: Evince, Firefox, MuPDF\textsuperscript{\textcolor{red}{\textbf{*}}}, Okular, SumatraPdf
		\end{itemize}
		\textsuperscript{\textcolor{red}{\textbf{*}}} Software that does not display the table of contents in a side panel.
	\item for the \gls{HTML}\index{HTML} format:
		\begin{itemize}
			\item[\textopenbullet] Linux: Firefox, Falcon, Links2, Midori, Dillo, SeaMonkey, NetSurf, Min
			\item[\textopenbullet] Mac: Firefox, Falcon, Midori, SeaMonkey, NetSurf
			\item[\textopenbullet] Windows: Firefox, Falcon, Midori, SeaMonkey, NetSurf
		\end{itemize}
\end{itemize}

% space for theme change: 5 mm

%\vspacepdf{5mm}

In short, you have the choice!

% space for theme change: 5 mm

\vspacepdf{5mm}

%These programs are all downloadable on their own website and are all under a license of \gls{free software}, and you can also find them on the \lang{Framasoft}\footnote{\urlFramasoftLogiciels{}} site.
These software packages can all be downloaded from their own websites and are all licensed as \gls{free software}, and you can read about some of them on the \lang{Free Software Directory}\footnote{\urlFreeSoftwareDirectory{}}or French \lang{Framasoft}\footnote{\urlFramasoftLogiciels{}} websites

