%%%%%%%%%%%%%%%%%%%%%%%%%%%%%%%%%%%%%%%%%%%
%THIS FILE CONTAINS ALL COMMON COMMANDS NEEDED FOR COMPILATION INTO BOTH PDF AND HTML.
%THE EXTENSION file src/macros.sty CONTAINS ONLY COMMANDS NEEDED EXCLUSIVELY FOR COMPILATION INTO PDF
%THE DIRECTORY src/hva/ CONTAINS ONLY COMMANDS NEEDED EXCLUSIVELY FOR COMPILATION INTO HTML WITH HEVEA, WITH THREE EXTENSION FILES, imakeidx.hva, macros.hva AND picins.hva.
%%%%%%%%%%%%%%%%%%%%%%%%%%%%%%%%%%%%%%%%%%%
% 2024/04:
% - \documentclass[] : change {book} to {scrbook} from KOMA-script -> rewritting the manual.
% - change fancyhdr to scrlayer-fancyhdr (compatibility with scrbook) to scrlayer-scrpage (contained in koma-script) -> modify macros.sty too.
% - rewriting title page for \maketitle (use with pandoc [produce html file], \begin(title)...\end(title) doesn't work with pandoc, "microtype" package too)
% 2024/10
% - renaming chapter files with an order number
% - clean documentclass / packages / add comments
% 2024/10
% - comment \ifIllustration to always have pictures in manuel
% - texlive-extra-utils -> make4ht
% 2024/11
% - simplification of \usepackage{graphicx} when pdf

%\ifx\pdfoutput\undefined		% if no pdfLaTeX outputting
% test KOMA-script with "scrbook" instead of "book"
%\documentclass[%
%	paper=a4,			% default=a4 and therefore not needed
%	12pt,				% default=11pt
%	headsepline,		% line under the header
%	footsepline,		% line above the footer
%	twoside=semi		% same left/right margins, default=twoside
%]{scrbook} 			% 
%}{
%\else							% else pdfLaTeX outputting
\documentclass[%
	pdflatex,
	a4paper,			% default=a4 and therefore not needed
	12pt,				% default=11pt
	english,			% use in babel, translator and varioref packages options %TODO: change with ngerman/english option in de/en manuals
	toc=listof,			% add a "lof" (list of figures) line at the begining of the toc (table of contents)
	twoside=semi,		% same recto/verso margins similar to simple verso margins, default=twoside
	headsepline,		% draw rule below header
	footsepline,		% draw rule above footer
	plainfootsepline 	% draw footsepline on plain pages (here for heading pages, may be due to twoside=semi in scrbook options
]{scrbook} 				% KOMA-Script class "book"		
%}
%\fi

%\usepackage{ifpdf}					% detect pdfTeX, replaced by \Ifpdfoutput (KOMA-script)
\usepackage[T1]{fontenc}			% the font encoding
%\usepackage[utf8]{inputenc}			% no longer need to specify, included in LaTex since April 2018
\usepackage{lmodern}				% Latin Modern police
\usepackage{babel}					% use the language define in \documentclass
%\def\frenchcontentsname{Sommaire}	% rename "Table des matières" (at end of document) from [french]{babel} to "Sommaire" (at start of document)
%\frenchsetup{ItemLabeli=\textbullet}	% change 1st level list marker of french babel
%\frenchsetup{ItemLabelii=-}			% change 2nd level list marker of french babel
\usepackage{translator}		% for translating the Fixed Names
\usepackage{textcomp}				% for special characters like \textcurrency
%\usepackage{txfonts}
%\usepackage{ae}
%\usepackage{cmbright}
%\usepackage{times}					% changed from txfonts
\usepackage{newtxtext}				% install texlive-fonts-extra, replacing times which replaces txfonts
%\usepackage{lwarp}					% produce html file with lwarp (install texlive-extra-utils)

\usepackage{microtype}				% Font expansion (use for title) with "\textls[x]" - doesn't work with pandoc / espace entre caractères (utilisé pour le titre)
% Web addresses
%\usepackage{url}					% loaded by hyperref


% --------------------------------------------
% Load graphicx package with pdf if needed
% --------------------------------------------
%\ifpdf
%%\Ifpdfoutput{%
	%%\usepackage[pdftex]{graphicx}		% enhanced support for importing graphics (x=enhanced) and then using the \includegraphics command to insert the file
	%%}{%			
	% else								% Commented out to avoid errors in html, and put in the macros.sty file for pdf
\usepackage{graphicx}				% enhanced support for importing graphics (x=enhanced) and then using the \includegraphics command to insert the file
\newcommand{\refimage}[1]{ (fig. \vref{#1})}	% for the hypertext link display on images
\newcommand{\vspacepdf}[1]{\vspace{#1}}			% for spaces after images bordered with text
	%%}
%% Graphics Extensions
%\ifpdf
%%\Ifpdfoutput{%
	%%\DeclareGraphicsExtensions{.pdf,.png,.jpg}
	%%}{%
	%\else
	%%\DeclareGraphicsExtensions{.eps}
	%%}
%\fi


% For illustrations
%\usepackage{picins}				% for text around image TODO change to wrapfig2 in the manuel
\usepackage{wrapfig2}				% TODO change in chapters, for text around image (replace picins)
\usepackage{caption}				% for good references in table of figures


%\usepackage{scrlayer-fancyhdr}		% facilities for constructing headers and footers, replace fancyhdr
% change to scrlayer-scrpage here and in macros.sty


%\pagestyle{fancyplain}
%\usepackage{tabularx}
%\usepackage{hhline}				% produce single or double line
%\usepackage{layout}				% TODO: Only if DE mode
\usepackage{varioref}		% defines the commands \vref, \vpageref, \vrefrange, and \vpagerefrange in french
%\usepackage{lastpage}
%\usepackage{longtable}
\usepackage{color}					% for a colored title page
%\usepackage{vmargin}				% for changing the title page margins
\usepackage{thumbpdf}				% create thumbnails (vignettes) in pdf


%\usepackage{emptypage}		% change by using \cleardoubleemptypage (in koma-scripts)
% Latex loads macros-3.0.sty for pdf output and Hevea loads /hva/macros.hva for html output
%\usepackage{macros-3.0}

%% Page layout %%
\usepackage[%
%	showframe,					% to see frame of geometry package
	top=1in,					% marge haute
	headheight=7mm,				% hauteur en-tête
	headsep=6mm,				% distance entre entête et corps de texte
	textheight=252mm,			% textheight=paperheight-topmargin-headheight-headsep-footskip
	footskip=11mm,				% distance entre bas de pied de page et bas du corps de texte
	hmargin=25mm				% horizontal margins (left and right), textwidth = paperwidt - margins
]{geometry}
\usepackage{scrlayer-scrpage}	% define and manage page styles by controlling page headers and footers			
\clearpairofpagestyles			% remove the default marks of the headings and the plain pages
\lehead{\leftmark}				% leftmark at left of even page
\lohead{\leftmark}				% leftmark at left of odd page
\rehead{\rightmark}				% rightmark at right of even page
\rohead{\rightmark}				% rightmark at right of odd page
\cfoot*{\pagemark}				% pagemark in the center of the footer
\ModifyLayer[addvoffset=-1ex]{scrheadings.foot.above.line}		% shift line up to increase distance between footer text and footerline in normal stylepages
\ModifyLayer[addvoffset=-1ex]{plain.scrheadings.foot.above.line}	% shift line up to increase distance between footer text and footerline in plain style page (chapter page)


%% items list environment %%
\renewenvironment{itemize}{%
	\begin{list}{\textbullet}{%
		\setlength{\itemsep}{-3pt}		% modify space between items, default=0 ?, to avoid enumitem package
		}
	}{%
	\end{list}%
}

%\frenchsetup{ItemLabeli=\textbullet}	% change 1st level list marker of french babel
%\frenchsetup{ItemLabelii=-}			% change 2nd level list marker of french babel

%\renewenvironment{itemize}%
%{\begin{itemize}%
%		\setlength{\itemsep}{-3pt}%
%		\setlength{\parsep}{0pt}%
%		\setlength{\parskip}{0pt}%
%	}%
%	{\end{itemize}}

%\newenvironment{maliste}{
%	\begin{list}{--}{
%		\setlength{\leftmargin}{0cm}
%		\setlength{\itemindent}{\widthof{--}+\labelsep}
%		}
%	}{
%	\end{list}
%}


%% Footnotes %%
\usepackage[%
	perpage						% resets footnote numbering for each page of the document.
]{footmisc}						% provides several different customizations of footnotes

\renewcommand\footnoterule{%	% redefine rule above footnote
	\kern 5pt 					% above footnoterule, space between text and footnoterule  
	\hrule width 2.5in			% define rule's width to 2.5 in
	\kern 6pt					% space between footrule and footnotes below
}

%% Index %%
\usepackage[%
	xindy						% sort index
]{imakeidx}						% for creating an index
\usepackage[%
	columns=2,					% default value is 2
	rule=1pt,					% thickness of a vertical rule between index columns. Default value is 0 pt, i. e. no rule.
	totoc						% add index in toc
]{idxlayout}					% key-value interface to configure index layout parameters

\usepackage[unicode]{hyperref}		% replace \usepackage{url}, used to add an URL and rewrite the "grisbi-manuel-urldef.tex" file (must be the last package)
\hypersetup{%							% to create metadata to insert in pdf
	pdftitle={Manuel de Grisbi},		% sets the document information Title field
	pdfauthor={The Grisbi Team},		% sets the document information Author field
	pdfcreator={Alain PORTAL}			% sets the document information Creator field
	pdfpagemode=UseOutlines,			% set default mode of PDF display, UseOutlines=show bookmarks
	pdfstartview=XYZ null null 1.0,		% set the startup page view, XYZ=left top zoom
	pdffitwindow=true,					% resize document window to fit document size, default=false
	pdfcenterwindow=true,				% position the document window in the center of the screen, default=false
	bookmarksnumbered=true,				% put section numbers in bookmarks, default=false
	bookmarksopen=true,					% open up bookmark tree, default=false
	colorlinks=true,					% color links, default=false
	citebordercolor=1 1 1,				% the color of the box around citations, rgb color
	linkbordercolor=1 1 1,				% the color of the box around normal links, rgb color
	linkcolor=blue,						% color for normal internal links, color=blue
	menubordercolor=1 1 1,				% color of border around menu links, rgb color
	urlbordercolor=1 1 1,				% color of border around URL links, rgb color
	urlcolor=blue,						% color of URL links, color=blue
	plainpages=true					% do page number anchors as plain Arabic, default=false
%	pdfpagelabels=true					% set PDF page labels - Commenté car élimine  "Package hyperref Warning: Option `pdfpagelabels' has already been used,"
}
%%%%%%%%%%%%%%%%%%%%%%%%%%%%%%%%%%%%%%%%%%%%%%%%%%%%%%%%%%%%%%%
% Contents: The url chapter
% $Id: grisbi-manuel-urldef.tex, modified from previous file :
% $Id: grisbi-manuel-urldef.tex, v 0.8.8 2011/XX/XX Jean-Luc Duflot
% $Id: grisbi-manuel-urldef.tex, v 1.0 2014/02/12 Jean-Luc Duflot
% $Id: grisbi-manuel-urldef.tex, v 3.0 2024/04/07 Dominique Brochard: create
% $Id: grisbi-manuel-urldef.tex, v 3.0 2024/11 Dominique Brochard:
% - rename file to 31-xxx
% - modify \urldef{\urlListSF} to {\urlListDiffGrisbi}
% - update Martin Stromberger mail
%%%%%%%%%%%%%%%%%%%%%%%%%%%%%%%%%%%%%%%%%%%%%%%%%%%%%%%%%%%%%%%


\urldef{\urlGrisbi}%
\url{https://en.grisbi.org}

\urldef{\urlGrisbiTelechargement}%
\url{https://en.grisbi.org/post/Download}

\urldef{\urlBugTracker}%
\url{https://www.grisbi.org/bugsreports/}

\urldef{\urlGrisbiWiki}%
\url{https://github.com/grisbi/grisbi/wiki}

\urldef{\urlTuxFamily}%         % French website
\url{https://www.tuxfamily.org/en/main}

\urldef{\urlSourceForge}%
\url{https://sourceforge.net/projects/grisbi/files/}

\urldef{\urlSourceForgeDocumentation}%
\url{https://sourceforge.net/projects/grisbi/files/Documentation/}

\urldef{\urlGitHubGrisbi}%
\url{https://github.com/grisbi/grisbi/}

\urldef{\urlLinuxGraphic}%         % French website
\url{https://www.linuxgraphic.org}

\urldef{\urlFramasoftLogiciels}%         % French website
\url{https://framalibre.org/}

\urldef{\urlFreeSoftwareDirectory}%
\url{https://directory.fsf.org/wiki/Main_Page}

%\urldef{\urlAssociationsGouv}%
%\url{https://www.associations.gouv.fr/la-comptabilite-associative.html}

%\urldef{\urlPlanComptable}%
%\url{https://www.plancomptable.com/index.htm}

%\urldef{\urlPlanDeComptes}%
%\url{https://www.plancomptable.com/titre-IV/titre-IV_chapitre-III_section-1.htm#431-1}

%\urldef{\urlListeComptes}%
%\url{https://www.plancomptable.com/titre-IV/liste_des_comptes_sa.htm}

%\urldef{\urlMaisonAssociations}%
%\url{https://www.loi1901.com/regle_comptable.php}

%\urldef{\urlComptaOnLine}%
%\url{https://www.compta-online.com/plan-comptable-general-pdf-ao2428}

\urldef{\urlWikipedia}%
\url{https://en.wikipedia.org/}

\urldef{\urlMetonymyDef}%
\url{https://literarydevices.net/metonymy/}

%\urldef{\urlAndrePascualEmail}%
%\url{andre@linuxgraphic.org}% without the leading "mailto:" for the DVI version

%\urldef{\urlDanielCartronEmail}%
%\url{daniel@cartron.org}    % without the leading "mailto:" for the DVI version

%\urldef{\urlCedricAugerEmail}%
%\url{cedric@grisbi.org}     % without the leading "mailto:" for the DVI version

%\urldef{\urlSebastienBlondeelEmail}%
%\url{sbi@april.org}% without the leading "mailto:" for the DVI version

%\urldef{\urlGeraldNielEmail}%
%\url{gerald@grisbi.org}% without the leading "mailto:" for the DVI version

\urldef{\urlBenjaminDrieuEmail}%
\url{benjamin@drieu.org}     % without the leading "mailto:" for the DVI version

%\urldef{\urlDionysosEmail}%
%\url{dionysos@grisbi.org}     % without the leading "mailto:" for the DVI version

%\urldef{\urlJulietteEmail}%
%\url{juliette@grisbi.org}     % without the leading "mailto:" for the DVI version

\urldef{\urlFrancoisTerrotEmail}%
\url{grisbi@terrot.net}     % without the leading "mailto:" for the DVI version

%\urldef{\urlLoicBreillouxEmail}%
%\url{lbreilloux@users.sourceforge.net}    % without the leading "mailto:" for the DVI version

\urldef{\urlPierreBiavaEmail}%
\url{pierre.biava@orange.fr}     % without the leading "mailto:" for the DVI version

\urldef{\urlDidierChevalierEmail}%
\url{didier.chevalier35@gmail.com}     % without the leading "mailto:" for the DVI version

\urldef{\urlWilliamOllivierEmail}%
\url{guneeyoufix@gmail.com}     % without the leading "mailto:" for the DVI version

%\urldef{\urlMickaelRemarsEmail}%
%\url{grisbi@remars.com}     % without the leading "mailto:" for the DVI version

\urldef{\urlJeanLucDuflotEmail}%
\url{jielbil@mailo.com}     % without the leading "mailto:" for the DVI version

\urldef{\urlAlainLetientEmail}%
\url{al1.letient@free.fr}     % without the leading "mailto:" for the DVI version

%\urldef{\urlGuyLebegueEmail}%
%\url{guy@guy-lebegue.fr}     % without the leading "mailto:" for the DVI version

\urldef{\urlMicheleBondilEmail}%
\url{ciboulette05@club-internet.fr}     % without the leading "mailto:" for the DVI version

\urldef{\urlListInfoEmail}%
\url{info@listes.grisbi.org}     % without the leading "mailto:" for the DVI version

\urldef{\urlListDevelEmail}%
\url{devel@listes.grisbi.org}     % without the leading "mailto:" for the DVI version

\urldef{\urlLudovicRousseauEmail}%
\url{ludovic.rousseau@gmail.com}     % without the leading "mailto:" for the DVI version

\urldef{\urlDominiqueBrochardEmail}%
\url{dbro17@free.fr}     % without the leading "mailto:" for the DVI version

\urldef{\urlBobAndersonEmail}%
\url{www23@scilutions.co.uk}     % without the leading "mailto:" for the DVI version

\urldef{\urlMartinStrombergerEmail}%
\url{mstromberger@mailbox.org}     % without the leading "mailto:" for the DVI version

\urldef{\urlListTraductionEmail}%
\url{traduction@grisbi.org}  % without the leading "mailto:" for the DVI version

\urldef{\urlListBugsreport}%
\url{bugsreports@listes.grisbi.org}     % without the leading "mailto:" for the DVI version

\urldef{\urlListDiffGrisbi}%
\url{https://listes.grisbi.org/mailman/listinfo}     % without the leading "mailto:" for the DVI version


		% include the "grisbi-manuel-urldef" file during the compilation

%% Glossary %%
\usepackage[%
	xindy,              % sort glossaries
	toc                 % add the glossary reference to the toc (table of contents)
]{glossaries}       % create a glossary, must be loaded AFTER hyperref
\usepackage[%
	automake
]{glossaries-extra} 	% check why no glossaries and give solutions

%% Index %%
\makeindex  %[intoc]	% creates the index with its reference in the toc

\definecolor{jaunegrisbi}{rgb}{1,1,.6}			% creating your own colors {yourcolorname}{model=rgb[0to1],RGB[0to255],cmyk or grey} rgb= 3 comma-separated values between 0 and 1 define the components of the color.
\definecolor{bleugrisbi}{rgb}{.1,.1,.4}
\definecolor{vertgrisbi}{rgb}{0,.6,.4}
\definecolor{ocregrisbi}{rgb}{1,.7,0}


% Virtualization of fonts
\newcommand{\lang}[1]{\emph{#1}}				% new command -> \lang = emph (e.g. italic)
\newcommand{\familyname}[1]{\textsc{#1}}		% new command -> \familynamelang = small caps
\newcommand{\menu}[1]{\textsl{#1}}				% new command -> \menu = slanted is oblique version of the roman font (e.g. italic)
\newcommand{\strong}[1]{\textsc{\textbf{#1}}}	% new command -> \strong = small caps + bold
\newcommand{\key}[1]{\texttt{<#1>}}				% new command -> \key = teletype font
\newcommand{\cmd}[1]{\texttt{#1}}				% new command -> \cmd = teletype font
\newcommand{\file}[1]{\textbf{#1}}				% new command -> \file = bold
\newcommand{\xml}[1]{\texttt{#1}}				% new command -> \xml = teletype font
\newcommand{\indexword}[1]{\textsf{#1}}			% new command -> \indexword = sans serif, for easy search of each indexed word in the page

%\renewcommand*{\footnote}{\centering}

\newcommand{\actuality}{}	% to see places to watch when updating the doc, using  the command grep actuality *.tex


%% Commented out because it doesn't work in html, and put in the macros.sty file for pdf
%% NE FONCTIONNE PAS POUR HTML
% redefine command listoffigures
%\ifpdf
%\makeatletter
%\renewcommand\listoffigures{%
%	% increases space between number and figure name for number >9 (see book.cls file)
%	\renewcommand\l@figure{\@dottedtocline{1}{1.5em}{2.8em}}%
%    \if@twocolumn
%     \@restonecoltrue\onecolumn
%    \else
%      \@restonecolfalse
%    \fi
%    \chapter*{\listfigurename}
%      \@mkboth{\MakeUppercase\listfigurename}{}
%    \@starttoc{lof}
%    \if@restonecol\twocolumn\fi
%    }
%\makeatother
%\else
%\fi

% For pdf only; for html, redefined  by an empty command in hva/macros.hva
% Glossary
\makeglossaries									% creates the glossary, entries of the glossary are in "src/{version}/{lang}/30-grisbi-manuel-glossary-en.tex"

%\input{grisbi-manuel-glossary}					% include "grisbi-manuel-glossary.tex" file in the document
\loadglsentries{30-grisbi-manuel-glossary-en}		% load and include glossary's entries in and from "30-grisbi-manuel-glossary-en"

%\input{grisbi-manuel-boolean-illustration}		% include "grisbi-manuel-boolean-illustration.tex" file in the document

%% -------------------
%% Begin of title page
%% -------------------

%\ifIllustration
\title{%
	\fontsize{40}{20}\selectfont					% increase size of title=40pt / line spacing=20pt
	\scshape										% small caps
	%	M\:a\:n\:u\:e\:l\; d\:e\; G\:r\:i\:s\:b\:i\\	% \: = medium space, \; = large space, \\ = line break
	\textls[250]{%									% increase space between characters, default=100 (active "microtype" package) [[ !!! doesn't work with pandoc !!! ]]
	Grisbi\; Manual}
	\\												% new line
	\vspace{1.5cm}									% vertical space between title and logo
	\includegraphics[width=6cm]{image/grisbi-logo.png}		% insert {file_name} graphic with width=6cm
	\vspace{1.5cm}									% % vertical space between logo and subtitle
}
\subtitle{%											% subtitle font size, default=large
	\sffamily\bfseries								% sans serif, bold
	\scshape										% small caps
	\fontsize{19}{20}\selectfont					% increase size of subtitle=20pt / line spacing=22pt
	Personal accounting software
	\vspace{1cm}\\									% vertical space
	\rule{4.5cm}{0.4pt}								% horizontal line, first argument width, second thickness
	\vspace{0.6cm}\\								% vertical space
}
\addtokomafont{author}{\large}						% modify author font size, default=Large
\author{%
	Copyright © 2001-2003 Daniel \familyname{Cartron}\\
	Copyright © 2004 Loïc \familyname{Breilloux}\\
	Copyright © 2004 Benjamin \familyname{Drieu}\\
	Copyright © 2011-2014 Jean-Luc \familyname{Duflot}\\
	Copyright © 2018 Bob \familyname{Anderson} (en)\\
	Copyright © 2018-2020 Martin \familyname{Stromberger} (de)\\
	Copyright © 2024-2025 Dominique \familyname{Brochard}\\
	\rule{2.5cm}{0.4pt}							% insert horizontal line, first argument width, second thickness
}
\addtokomafont{date}{\large}					% modify date font size, default=Large
\date{%
	Version 3.0 of 2025 (provisional)
}
%\else
%\title{Manuel de Grisbi}
%\fi

%% -----------------
%% End of title page
%% -----------------

\begin{document}

%\pagestyle{scrheadings}			% titles of chapters and sections are repeated in the header, number of page in the footer


\maketitle							% create title page
%\cleardoubleoddemptypage			% force a break to an even (left, verso) page -> produce an additional odd page with empty style
% For pdf only; for html, redefined  by an empty command in hva/macros.hva
\frontmatter						% renders numbered pages in toc/tof in lower-case Roman letters

% not used
%\include{grisbi-manuel-title}

%\maketitle				% title v3.0.1 -> %\maketitle
%\myclearemptydoublepage

%\sommaire
\tableofcontents					% table of contents (toc)	

%\myclearemptydoublepage


% List of figures for pdf only ; for html, the addcontentsline is redefined by an empty command in hva/macros.hva
% Adds reference to lof in the toc
%\ifIllustration
%	\listoffigures
%		\ifpdf
%		\addcontentsline{toc}{chapter}{Table des figures}
%%		\myclearemptydoublepage
%		\cleardoubleemptypage % replace \myclearemptydoublepage
%		\else
%		\fi
%	\else
%\fi

\listoffigures				% list of figures (lof)

%% mis dans le \mainmatter pour lien correct des entrées d'index dans le préambule en html
%% voir aussi le chapitre Préambule
%\include{grisbi-manuel-preamble}
%\myclearemptydoublepage
% For pdf only; for html, redefined  by an empty command in hva/macros.hva
\mainmatter							% pages are numbered in Arabic numerals and the page counter is reset to 1.
%\KOMAScriptVersion						% print the used version number of Koma-Script

%%%%%%%%%%%%%%%%%%%%%%%%%%%%%%%%%%%%%%%%%%%%%%%%%% %%%%%%%%%%%%%%%
% Contents: The preamble chapter
% $ Id: gray-manual-preamble.tex, v 0.4 2002/10/27 Daniel Cartron
% $ Id: gray-manual-preamble.tex, v 0.6.0 2011/11/17 Jean-Luc Duflot (no change)
% $ Id: gray-manual-preamble.tex, v 0.8.9 2012/04/27 Jean-Luc Duflot (typo changes)
% $ Id: gray-manual-preamble.tex, v 1.0 2014/02/12 Jean-Luc Duflot
%%%%%%%%%%%%%%%%%%%%%%%%%%%%%%%%%%%%%%%%%%%%%%%%%% %%%%%%%%%%%%%%%
% deleted because it does not work in pdf and creates a problem in html
% \Begin{small}
% star command: deleted star for index entries with correct link display in html
\chapter{Preamble\label{preamble}}

%% commented for index entries with correct link display in html
%% see also the preamble chapter moved in \mainmatter in manuel.tex
% \markboth{}{Preamble}
% \ifpdf
% \addcontentsline{toc}{chapter}{Preamble}
% \else
% \fi

\section{Forward to the English Translation\label{preamble-foreword}}

%This English translation is a work in progress and constitutes a revision of version 1.0 of this manual.
%This manual is currently being translated into English version 3.0, which corresponds to version 3.0 of the software. It is a combination of the revision of English version 1.0 and the translation of the French version 3.0.
Manual version 3.0 corresponds to software version 3.0.\\
The English version is a combination of the revision of English version 1.0 and the translation of French version 3.0.

It was started by Bob \familyname{Anderson}\footnote{\urlBobAndersonEmail{}} with the version 1.0 in order to give a little back to the \gls{Free and Open Source Software} (FOSS) Community.  He had no particular financial or linguistic skills, but as a long-standing user of the software and managing without the help of the manual, he knew how necessary this translation was.

%This translation is a strict paragraph by paragraph  translation, it is important to keep to this translation format in order to simplify the addition to the English translation of any changes made to the original French Manual.  In some places you will also come across a short paragraph or footnote introduced by the phrase \\
%\strong{Translators Note:} \\
%where an extra paragraph or footnote has had to be added to explain a problem with the translation, something specific to the English language interface or flag a section which needs revision in both the French and English manual versions.  However in order to ensure the strict one to one correspondence between the French and English translations; it is essential that the French Manual remains the authoritative document and thus revisions to the English Manual will only be made \strong{AFTER} the French revision is accepted.

This translation is word for word. It is important to keep to this format to make it easier to add changes to the English translation of the original French manual.  You may also see a short paragraph or footnote with the note \strong{Translators Note:}. If a new paragraph or footnote is added to explain a problem with the translation, or to flag a section that needs revision in both the French and English versions, it must be done in a way that ensures the French and English versions are exactly the same. This means that any changes to the English version will only be made \strong{after} the French version is accepted.

%\subsection*{How to help make an English manual}

%There are many ways you can help to complete this English translation.  As yet only a handful of chapters have been translated.  The figures are currently showing the French interface version and all the example account entries in the illustrations are also in French.  Those chapters that have been translated could probably benefit from better phrasing in places.

%For detailed guidance on how you can help please see the document HelpTranslate-en.md on the manual\lang{Github}\footnote{\urlGitDoc{}} site.

%\subsection*{Acknowledgements and a dedication}

%Most of my translations were done with the aid of \lang{Google Translate}\footnote{\urlGtrans{}} using \lang{Translate Shell}\footnote{\urlGtransShell{}}

%The English translation of this manual is dedicated to the memory of my father John Anderson FCA\footnote{FCA = Fellow of the Institute of Chartered Accountants.  John Anderson (AJ)  studied accountancy while serving in the Royal Air-force as a Radar Technician during the second World War.  On being demobbed at the end of the war he then joined the accountancy firm started by my Grandfather: H J Anderson and Co.  Any accountancy skills I may have picked up are thanks to him.}

\subsection*{The translation from the French Manual starts on the next page}

%The translation from the French Manual starts with a the next section heading \strong{Why call it Grisbi?}. I have changed the section heading from the original \strong{Préambule}. This section is a commentary by the original author on the etymology of the word Grisbi.
%The translation from the French manual begins with the title of the subsequent section \strong{Why refer to it as Grisbi?}, modifying the original title \strong{Préambule}. This section comprises the Daniel \familyname{Cartron} original author's commentary on the etymology of the word \lang{grisbi}.

The translation from the French manual begins with the title of the next section, \strong{Why refer to it as Grisbi?}, which has been modified by Bob \familyname{Anderson} from the original title, \strong{Etymology}. This section includes the original author Daniel \familyname{Cartron}'s commentary on the etymology of the word \lang{grisbi}.

%I should give a word of warning about my translation efforts on this first section about the origin of the word Grisbi.  I have tried to retain the sense of the various French dictionary entries,  although most of the words are translated to English the accuracy of the translation should not be relied upon for more than giving the reader an overall feel of the original French text.  If you are very interested in the etymology then you had better read the original French version of this section of the manual.  

Bob \familyname{Anderson} provides a cautionary note regarding his translation efforts. He has endeavoured to maintain the denotation and connotation of the French dictionary entries. While the majority of the words have been translated into English, the accuracy of these translations should not be relied upon, except to provide a general impression of the original French text. For those with a keen interest in etymology, it would be beneficial to consult the original French version of this manual section.

\newpage 
\section{Why refer to it as Grisbi?\label{preamble-etymology}}

Several \gls{Grisbi} users have asked Daniel \familyname{Cartron}, the editor of the first version of the manual, to insert a brief reminder of the meaning of this word, which, to his chagrin, fell back into disuse.

His first (brief) research did not bring him any results worth publishing, He had then dropped it until one day he had the opportunity to spend some time in a well-stocked library containing dictionaries of all kinds, there the harvest was abundant. It was so abundant that Daniel \familyname{Cartron} hesitated a long time to know what he was going to keep and what he was going to eliminate \dots{}
Finally he decided to keep everything, even if we go from a paragraph to four pages. He just made a compilation of the shortest items to the longest ones.
Indeed he find it interesting to study the differences between the different dictionaries, but even more to note the similarities, at this point striking that one could title this chapter not \emph{The game of the seven errors} but \emph{Who copied who?} It's up to you to find\dots{}
And he also added a passage on the film because for those who still know what \indexword{grisbi}\index{Grisbi} means is essentially due to the deserved reputation of this work.

Here are some sources on the etymology \index{etymology} of the word grisbi:

\subsection*{Dictionary of French language --- Hachette}

[grizbi] n. m. Slang. Silver --- Of grey (grey money, cd rouchi griset [1834], ``liard''), and suff. pop. -bi; 1895, spread in 1953.

\subsection*{Grand Larousse of the French language}

[grizbi] n. m. (of \emph{grey [and]}, piece of six liards [1834, Esnault] --- der.
from \emph{grey}, because of the color [cf. also \emph{grisette}, ``coin'' --- 17th century. ---, and \emph{white and grey coin}, 1784, Esnault] --- with the suffix.slang. \emph{-bi}; 1896, Delesalle).
\emph{Slang.} Money: \emph{Do not touch the grisbi} (title of a novel by Albert Simonin [1953]).

\subsection*{Historical Dictionary of the French language --- Robert}

n.m. appeared in 1895 (\emph{grisbis}) and spread from 1953 by the novel \emph{Do not touch the grisbi} by A. Simonin, would be composed of \emph{gris} ``grey money'' (1784: see le rouchi \emph{griset} ``six-liard coin'', 1834, and \emph{grisette} ``coin'', v. 1634 ) and the \emph{bi} element of obscure origin: \emph{grisbi}, ``silver'' in slang, could be a tautological compound of \emph{grey} and \emph{bis}.

\subsection*{Dictionary of French slang and its origins --- Larousse}

Very controversial origin: either of griset, ``coin'', and a mysterious suffix -bi, or bread both grey and bis, or English slang crispy, silver; we propose to see a metonymic use of gripis 1628 [Cheneau], grispin, grisbis 1849 [Halbert], ``meunier'', that is to say, ``one who has at his home wheat'' 1895 [Delsalle] but reintroduced by ``Touch the grisbi'', famous novel of A. Simonin, published in 1953.

VARIANTS --- grijbi: 1902 [Esnault] --- grèzbi: around 1926 [id.]

DERIVATIVES --- grisbinette n.f. One hundred old franc coins: 1957 [Sandry-Carrère].

\subsection*{Treasures of the French language}

\emph{Slang.} Silver. Synon. pop. \emph{money, cake, pèze, cash. The grisbi I'm big enough to pick it myself! (\dots{}) Riton who had not even known how to behave like a man (\dots{}) as soon as he felt enough grisbi} (\familyname{Simonin}, \emph{Touch not to the grisbi}, 1953, p 231).
\strong{Pronunciation}: [grizbi]. \strong{Étymol. and Hist.} 1896 \emph{grisbis}
slang. ``money'' (\familyname{Delesalle}, \emph{French-Slang and Slang-French Dict.}). Word composed of rad. of \emph{griset}, in the sense of this six-liard piece'' (1834 ds \familyname{Esn.}), der. of \emph{grey}, because of the color (\emph{cf.} also \emph{ca} 1634 \emph{grisette} ``coin'', \emph{The Norman Muse} by D. Ferrand , Ed A. Heron, II, 91, 1784, Brest, \emph{white and grey coin} in \familyname{Esn.}, and a second part of obscure origin which represents maybe the suff. pop. -bi, to be close to \emph{nerbi} ``very black'' (from \familyname{Esn.}). It is not impossible that \emph{grisbi} (formerly \emph{grisbis}) is a tautological compound of \emph{grey} and \emph{bis}.

\strong{Bbg.} \familyname{Rigaud} (A.). L'arg. litt. \emph{Life lang.} 1972, pp. 114-117.

\subsection*{Grand Robert of the French language}

[grizbi] n. m. --- 1895: spread 1953 by Simonin's novel \emph{Touch not to grisbi}; the word was rare or archaic v. 1950: from \emph{grey} ``grey money'' (see rouchi \emph{griset} ``liard'', 1834), and suff. pop. \strong{Slang} Money. \emph{T'as du grisbi?}

1 --- This expression: ``Do not touch the grisbi'' becomes a variation of ``Do not mess with the nippes''. This is the keyword that leads the chronicle of these knights of ill-gotten fortune who gave mobility to Peter \familyname{Cheyney}'s cape and machine gun novels.
\familyname{P. Mac Orlan}, \emph{in} Albert \familyname{Simonin}, Do not touch the grisbi, Preface, p. 6.

%2 --- ``Don't sweat the small talk \dots{} First we have no time if you want me to find you Ali. It all depends on what he has of grisbi digging; if he is armed, one has a chance to find him in the flames, in the Carillon part.''
2 --- ``Forget bein' nice, fam\dots{} We ain't got time if you need me to track down Ali. It all boils down to how much grisbi (crispy) he's stacking in his pockets. If he’s packing, there’s a shot at spotting him at the casino, at the Carillon joint''

Albert \familyname{Simonin}, do not touch the grisbi, p. 147.

\subsection*{Dictionary of Unconventional French --- Hachette}

n.m. (Grisby)

Silver (intrinsically).

%??\emph{To the little boss of the team, a muffle with a torpedo cap, blue heaters and pumps varnished, the kid had just said she was frying only for the right motive, to relieve me of my hundred bags. He had replicated, the naughty jealousy: --- The \emph{grisbi}, I'm big enough to pick it up myself! They both said they were both, and they were equally ready for anything emph{grisbi} (loot). They and their little friends. Like Angelo-la-Tante and Josy-la-Peau-de-Vache; like Ali-le-Fumier and his garbage of espingos; like Riton, who had not even known how to be a man with his child, as soon as he had felt enough of \emph{grisbi}; like Marco and his little Wanda, if honest, but not hesitating to be stepped over by the man \emph{grisbi}! like the same Lulu, no doubt, waiting patiently for me to turn down with my \emph{grisbi}!}??

``To the little kingpin of the team, a kid in a torpedo cap, blue overalls and patent leather pumps, the little girl had just said she was watching only for the right motive, to relieve me of my hundred notes. He had replicated, the naughty jealousy: ``The \emph{grisbi}, I'm big enough to pick it up myself!'' They were both right, both equally ready to do anything for the \emph{grisbi} (loot). They and their little friends. Like Angelo-la-Tante and Josy-la-Peau-de-Vache; like Ali-le-Fumier and his garbage of espingos; like Riton, who hadn't even known how to behave like a man with his brat, as soon as he felt he had enough \emph{grisbi}; like Marco and his little Wanda, so honest, but who didn't hesitate to be stepped over by the \emph{grisbi}! like also the young Lulu, no doubt, who waited patiently at home for me to come back, with my \emph{grisbi}''

A. \familyname{Simonin}, Do not touch the grisbi, p. 233

HIST. --- 1895, but probably little used: A. \familyname{Bruant} and L. \familyname{Blédort}, who occasionally accumulate synonyms (\emph{weigh, bone,}, etc.), do not use \emph{grisbi}, although \familyname{Bruant} records it in 1901 (\emph{grisbis}).
% The deserved success of A. Simonin's novel in 1953 ??made first use of?? the word, which does not seem really integrated in the series of alternative words for money, like \emph{wheat, sorrel, flouze} or \emph{fric}.
The deserved success of A. \familyname{Simonin}'s novel in 1953 gave a new lease of life to the word, which nevertheless does not seem to be truly integrated into the series of designations of money, such as \emph{wheat, sorrel, flouze} or \emph{fric}.

Rouchi \emph{griset}, ``a six-liard piece'' (1834), so called because of its color. But the only explanation currently available given by \familyname{Esnault},  is not satisfactory; on the one hand, the element \emph{bi} remains unexplained, if not by a ``suffix'' unknown; on the other hand, \familyname{Bruant} writes \emph{grisbis}, and it is possible (if not probable) that the central \emph{s} pronounced only since 1953; which would lead to an explanation: \emph{gris-bis}, in the series of alternative words for bread, \emph{wheat, carmine, biscuit, pancake}, etc.

Finally, if the \lang{metonymy}\footnote{\urlMetonymyDef{}} of colour is actually used to denominate money, it is always a precise category of money: ``cash''. Thus \emph{jaunet, white, white, copper}, are not interchangeable nor usable for ``money'' abstract.

%We will also recall the meaning of \emph{grey}: ``expensive'' (V. \emph{grisol}) and the possibility of the pseudo-suffix augmentative \emph{bi}, ``very'' , even rare. We would then have: \emph{grey-bi}, ``very expensive''? But the hypothesis is speculative.

We would also point out the meaning of \emph{gris}: ``expensive'' (V. \emph{grisol}) and the possibility of the augmentative pseudo-suffix \emph{bi}, ``très'', even rare. We would then have: \emph{gris-bi}, ``very expensive''? But the hypothesis is adventurous.


\section{Bibliography\label{preamble-biblio}}

\emph{Do not touch the Grisbi!} by Albert \familyname{Simonin}, published in 1953

\begin{itemize}\itemsep=-3pt
	\item Gallimard, Collection Série Noire No. 148, first edition in 1953
	\item Le Livre de Poche No. 1152, first edition in 1953
	\item Gallimard, Collection Carré Noir No. 94, first edition in 1972
	\item Gallimard, Collection Folio No. 2068, first edition in 1989
	\item Gallimard, Collection Folio Policier No. 183, first edition in 1953, reissued in 2014 (always edited)
\end{itemize}

\section{Filmography\label{preamble-filmography}}

\strong{Do not touch the grisbi} (French for \strong{Don't touch the loot}), released as \strong{Honour Among Thieves} in the United Kingdom and \strong{Grisbi} in the United States

French-Italian film (1954). Gangster-film genre. Duration: 1h 34 min

Original title: Grisbi\\

Distribution:

\begin{itemize}\itemsep=-3pt
    \item Jean \familyname{Gabin}: Max the liar
    \item Rene \familyname{Dary}: Riton
    \item Dora \familyname{Doll}: Lola
    \item Vittorio \familyname{Sanipoli}: Ramon
    \item Marilyn \familyname{Buferd}: Betty
    \item Gaby \familyname{Basset}: Marinette
    \item Paul \familyname{Barge}: Eugene
\end{itemize}

Director: Jacques \familyname{Becker}

\subsection*{Synopsis}

Max-the-liar and Riton have just pulled off the greatest heist of their lives: stealing 50 million francs worth of gold bars at Orly. With this ``grisbi'', both gangsters expect to enjoy a peaceful retirement. But Riton can not resist telling his mistress Josy about money. The younger burlesque-dancer girlfriend passes the valuable information to Angelo, a drug dealer with whom she is cheating on Riton with. Angelo kidnaps the old mobster and demands ``grisbi'' from Max as ransom \dots{}

\subsection*{Trivia} 

\paragraph{A well-oiled tandem:} Jean \familyname{Gabin} and René \familyname{Dary} are considered two sacred figures of cinema before the war.
\paragraph{\familyname{Becker} father and son:} Jacques' son \familyname{Becker}, Jean, makes his film debut here as an assistant director. He is only fifteen years old!
\paragraph{Albert \familyname{Simonin}:} Writer and screenplay by Albert \familyname{Simonin}, who here adapts  his own novel, will make four more movies with \familyname{Gabin}, all dialogues by \familyname{Audiard}: \emph{Le cave se rebiffe} (1961) and \emph{The gentleman of Epsom} (1962) by Gilles \familyname{Grangier}, \emph{Mélodie en sous-sol} (1963) by Henri \familyname{Verneuil} and \emph{The Pasha} (1967) by Georges \familyname{Lautner}. After adapting his \emph{Les Tontons flingueurs} for Georges \familyname{Lautner} (1963), he became his screenwriter for \emph{Les Barbouzes} (1964).

\subsection*{Zone 2 DVD version}

\begin{itemize}\itemsep=-3pt
    \item Interactivity: Home menu, access to scenes, filmographies drop-downs from the director, from Lino \familyname{Ventura} and John \familyname Gabin{}
    \item Cinema format: full screen
    \item Sound version: French in mono
    \item Subtitles: none
    \item France --- 1954 --- Black \& white
    \item Length: 92 min --- 1 ~ disc --- 1 ~ side --- 1 ~ layer
    \item Release date: September 19, 2001
    \item Publisher: Studio Canal
\end{itemize}

\subsection*{Zone 2 Blu-ray version}

\begin{itemize}\itemsep=-3pt
	\item Cinema format: 4/3 respected format 1.33
	\item Sound version: French in stéréo
	\item France --- 1954 --- Black \& white --- 94 min
	\item Subtitles: none
	\item Release date: March 10, 2017
	\item Publisher: Studio Canal
\end{itemize}

\subsection*{Zone 1 DVD / Blu-ray versions}

\begin{itemize}\itemsep=-3pt
	\item Cinema format: 4/3 respected format 1.33
	\item Sound version: French in stéréo
	\item France --- 1954 --- Black \& white --- 94 min
	\item Subtitles: English
	\item Release date: 21 Aug. 2017
	\item Publisher: Studio Canal
\end{itemize}

The zone 1 versions have English subtitles and include interviews with filmmaker Jean \familyname{Becker} and actress Jeanne \familyname{Moreau}.

% deleted because it does not work in pdf and creates a problem in html
% \End{small}
 	 % TODO update screenshots and text "preamble", uncomment when finished
%\cleardoubleemptypage

%%%%%%%%%%%%%%%%%%%%%%%%%%%%%%%%%%%%%%%%%%%%%%%%%% %%%%%%%%%%%%%%%

% Contents: The introduction chapter

% $ Id: graybi-manual-intro.tex, v 0.4 2002/10/27 Daniel Cartron

% $ Id: gray-manual-intro.tex, v 0.5.0 2004/06/01 Loic Breilloux

% $ Id: gray-manual-intro.tex, v 0.6.0 2011/11/17 Jean-Luc Duflot

% $ Id: gray-manual-intro.tex, v 0.8.9 2012/04/27 Jean-Luc Duflot

% $ Id: gray-manual-intro.tex, v 1.0 2014/02/12 Jean-Luc Duflot

%%%%%%%%%%%%%%%%%%%%%%%%%%%%%%%%%%%%%%%%%%%%%%%%%% %%%%%%%%%%%%%%%

\chapter{Introduction\label{introduction}}

\section{General \label{introduction-General}}

Grisbi is a free accounting program, developed in the \gls{C} language with support for \gls{GTK}+ 3, originally for the \gls{GNU/Linux} platform. There is now a \indexword{\gls{porting}}\index{porting} on \gls{Windows}, \gls{macOS}, FreeBSD, packages for different  \gls{Linux distributions}, and other possibilities to discover on \lang{Grisbi}\footnote{\urlGrisbi{}} or \lang{Sourceforge}\footnote{\urlSourceForge{}}.

The basic principle is to allow you to classify in a simple and intuitive way your financial operations, whatever they are, in order to be able to exploit them in the best way according to your needs.

Grisbi emphasises simplicity and efficiency, without excluding the sophistication needed by more advanced users. Future features will always try to meet these criteria.

\section{Features\label{introduction-features}}

\strong{Translators Note:} In this translation, the French word ``Tiers'' in general has been replaced by the English word "Payees", however the French version of Grisbi also caters for small business use. In the French version any third party i.e. a creditor or debtor could be implied by the word ``Tiers''.\\
\textbf{Note:} French business accounting rules may be incompatible with the rules of any English speaking territories, Grisbi should be used \strong{only} for personal financial record keeping outside of France.



\subsection{What Grisbi knows how to do}

\begin{itemize}
	\item Software developed by French programmers, so in full compliance with the logic of French accounting
	\item Simple and intuitive interface, with full-screen display control
	\item Multi-account and multi-user management
	\item Local Settings Management (Dates, Decimal Separators and Thousands)
	\item Bank accounts, cash, assets and liabilities
	\item Multi-currency management, with support for exchange rates and exchange fees
	\item Credit card management (immediate or deferred debit)
	\item Description of transactions with: date, valuation date, fiscal year, Payee amount, currency, category and sub-category, budget allocation and sub-allocation (for analysis of expenditure), note, entry number (assigned by Grisbi), part number, account number, bank reference
	\item Budget classifications and attribution with automatic recall of operations and sub-operations for a given Payee
	\item Clear Credit and Flow fields option for auto-complete
	\item Calculation of the balance according to the date of the transaction or its valuation date
	\item Transfer between accounts, including different currencies, with automatic cross posting
	\item Scheduling operations with automatic or manual validation
	\item Deadline monitoring
	\item Moving and Cloning Operations
	\item Cloning scheduled operations
	\item Shading of different periods in the scheduler
	\item Analysis and financial reports thanks to the powerful report templates module
	\item Several pre-formatted report templates available and customizable
	\item Virtuals Payees created by reports
	\item Printing of reports
	\item Simulation of loans and depreciation tables with printing and data export
	\item Estimated budget with charts on forecasts and historical data
	\item Business Accounting with Chart of Accounts (for French Accounting)
	\item Account order by drag and drop in the account list
	\item Import files in the following formats: \gls{QIF}, \gls{OFX}, \gls{Gnucash} or \gls{CSV}
	\item Importing category files into budget allocations
	\item Export files in the \gls{QIF} or \gls{CSV} formats
	\item Choice of custom icons for your different accounts
	\item Icons in the \indexword{\gls{SVG}}\index{svg} formats
	\item Even more context menus on the right mouse button (navigation pane)
	\item Numerous keyboard shortcuts for good ergonomics
\end{itemize}

\subsection{What Grisbi does not know yet}

\begin{itemize}
	\item Automatic breakdown of loan repayments
	\item Internet reconciliation
\end{itemize}

\section{Software Evolution}

\subsection{Development and versions}

Grisbi is software in active development, any feedback (idea, bug, documentation\dots{}) is welcome. You can send it to the relevant lists referenced in the  \vref{introduction-contacts} \menu{Contacts} section or on the  \lang{Grisbi}\footnote{\urlGrisbi{}} site.


%You can, if you have the taste of adventure, download the versions in development on the Grisbi content management system based on \gls{Git}.
If you're feeling adventurous, you can download and compile the latest version being under development on the content management system \gls{GitHub}\footnote{\urlGitHubGrisbi{}\label{siteGitHubGrisbi}} using \gls{Git}.

Indeed, the code of a new versions of Grisbi is typically frozen several weeks before its final release, in order to allow the development team time to check and eradicate the latest bugs. During this period, the format of the account file no longer changes, and you can, with a minimum of precautions (frequent backups, etc.), benefit from the latest improvements several weeks in advance, and also participate in debugging.

%Finally you also have access to all the evolutions of the code after version 0.3.2 (the version from which we set up a \gls{CVS} on the Grisbi site.
Finally, you can access all changes to the code since version 0.3.2 (the version from which the code is available on the \gls{GitHub}\footref{siteGitHubGrisbi} website).

Note that as of version 0.6, even numbered minor number versions (e.g. 0. \underline8) are stable versions, whereas odd number minor number versions are unstable and should not be used under normal conditions; therefore only stable versions are mentioned here.

% page break for solidarity title

%\newpage
\subsection{Version 0.1}

\begin{itemize}
	\item Management of several accounts%Gestion de plusieurs comptes
	\item Creation, modification and deletion of accounts and transactions (the minimum required to operate  \dots{})%Création, modification et suppression de comptes et d'opérations (enfin le minimum pour fonctionner \dots{})
	\item Possibility of pointing operations%Possibilité de pointer les opérations
	\item And backup, without which none of this would be very useful!%Et la sauvegarde, sans laquelle tout cela n'aurait pas beaucoup d'intérêt !
\end{itemize}

\subsection{Version 0.2}

\begin{itemize}
	\item Automatic import of account files from previous versions%Importation automatique des fichiers de comptes de versions précédentes
	\item Use of lists for third parties and categories \dots{} (more practical than typing everything)%Utilisation de listes pour les tiers, catégories \dots{} (plus pratique que de tout taper)
	\item Automatic entry when typing and filling in of the end of the transaction like the previous version%Saisie automatique lors de la frappe et remplissage de la fin de l'opération comme la précédente semblable
	\item Management of transfers between accounts%Gestion des virements entre comptes
	\item Simplified or full display of transactions%Affichage des opération simplifié ou complet
	\item Account balancing%Équilibrage de comptes
\end{itemize}

\subsection{From version 0.3.0 to version 0.3.3}%De la version 0.3.0 à la version 0.3.3}

\begin{itemize}
	\item Management of planned or cyclical operations (schedule)%Gestion des opérations prévues ou cycliques (échéancier) 
	\item Support for currencies and the changeover to the euro (from 1 January 1999 to 1 January 2002)%Prise en charge des devises et du passage à l'euro (du 1er janvier 1999 au 1er janvier 2002)
	\item Import/export added and file import improved \gls{QIF}%Ajout de l'import/export et amélioration de l'import de fichiers \gls{QIF}
	\item Added breakdown%Ajout de la ventilation
	\item Multi-user support%Support multi-utilisateur
	\item For each bank and account, added details such as:%Pour chaque banque et compte, ajout des détails comme:
	\begin{itemize}
		\item full numbers of the associated account, branch and bank code%numéros complets du compte associé, du guichet et du code banque
		\item different holder for each account with home address%titulaire différent pour chaque compte avec adresse personnelle
		\item contact details for a correspondent%coordonnées d'un correspondant
	\end{itemize}
	\item Categories tab added%Ajout de l'onglet catégories
	\item Ability to change currency names and codes%Possibilité de changer le nom et le code des devises
	\item Alphabetical sorting now (finally) takes accents into account%Le classement alphabétique prend (enfin) en compte les accents
	\item Memorises the last working directory and the last files opened% Mémorise le dernier répertoire de travail et les derniers fichiers ouverts
	\item A single number per transaction, whatever the account%Un numéro unique par opération quel que soit le compte
	\item The \key{+} (\key{-}) keys in a date field increment (decrement) the date%Les touches \key{+} (\key{-}) dans un champ de date incrémentent (décrémentent) la date
	\item New fields can be displayed in the transaction entry form%Affichage possible de nouveaux champs dans le formulaire de saisie des opérations 
	\item Deadlines can be set for a defined period of time%Échéances paramétrables pour un laps de temps défini
	\item Display of the current balance%Affichage du solde courant pointé
	\item Possibility of entering the date in the form ddmm, ddmmyyy or ddmmyyyy%Possibilité d'entrer la date sous la forme jjmm, jjmmaa ou jjmmaaaa
	\item Transaction totals in the Payees, Categories and Budget Allocations tabs%Totaux des opérations dans les onglets Tiers, Catégories et Imputations Budgétaires (IB)
	\item The logo and display font can be customised%Personnalisation possible du logo et de la police d'affichage.
	\item Automake and Autoconf tools to simplify compilation of software source files%Mise en place des outils Automake et Autoconf permettant de simplifier la compilation des fichiers sources du logiciel
\end{itemize}

\subsection{From version 0.4.0 to version 0.4.4}

\begin{itemize}
	\item Customisable interface layout%Personnalisation de l'agencement de l'interface
	\item Import/export capability:%Possibilité d'import/export:
	\begin{itemize}
		\item lists of categories%de listes de catégories
		\item lists of budget items%de listes d'imputations budgétaires (IB)
		\item created reports%d'états créés
	\end{itemize}	
	\item Financial reports added%Ajout des rapports financiers
	\item Grisbi is now internationalised and translations have been improved%Grisbi est désormais internationalisé et les traductions sont améliorées
	\item Display of multi-currency balances in the start-up screen%Affichage des soldes multi-devises dans l'écran de démarrage
	\item Remarks can now be displayed in the schedule%Les remarques peuvent être affichées dans l'échéancier
	\item New tool to allow contributors to anonymise Grisbi files to maintain confidentiality before submission%Nouvel outil permettant aux contributeurs d'anonymiser les fichiers Grisbi afin de garder la confidentialité avant de les envoyer
	\item The width of the columns in the schedule can be modified%La largeur des colonnes de l'échéancier est modifiable
	\item Harmonisation of date fields:%Harmonisation dans les champs de date:
	\begin{itemize}
		\item \key{Ctrl}\key{Enter} in a date field opens a calendar%\key{Ctrl}\key{Entrée} dans un champ de date ouvre un calendrier
		\item the arrow keys are active in a calendar%les touches fléchées sont actives dans un calendrier
		\item \key{Ctrl}\key{+} (\key{-}) in a date field increases (decreases) the date by approximately one week%\key{Ctrl}\key{+} (\key{-}) dans un champ de date augmente (diminue) la date d'environ une semaine
		\item \key{Pg.Up} (\key{Pg.Dn}) increases (decreases) the date by approximately one month%\key{Pg.Préc} (\key{Pg.Suiv}) augmente (diminue) la date d'environ un mois
		\item \key{Ctrl}\key{Pg.Up} (\key{Pg.Dn}) increases (decreases) the date by approximately one year%\key{Ctrl}\key{Pg.Préc} (\key{Pg.Suiv}) augmente (diminue) la date d'environ un an
	\end{itemize}	
\end{itemize}

\subsection{From version 0.4.5 to version 0.6}

\begin{itemize}
	\item Revamped GUI
	\item \gls{GTK}+ 2 library support for a nicer environment and simplified porting on Windows
	\item No more dependencies on \gls{Gnome}
	\item Native Windows version (thanks to François \familyname{Terrot})
	\item Native \gls{UTF-8} support 
	\item Printing reports by \gls{LaTeX}
	\item Export reports in \gls{HTML}
	\item Improved user interface:
		\begin{itemize}
		\item message enhancement, users can select ``ignore''
		\item improved management of segmentation errors
		\item improvement of the preferences window
		\item pop-up menu on the list of operations
		\item improving items
		\end{itemize}
	\item Completely clickable list of accounts with summaries
	\item Global configuration in \gls{XML} (thanks to Axel \familyname{Rousseau})
	\item Rewrite of the file import:
		\begin{itemize}
		\item support for the \gls{QIF}, \gls{OFX}, \gls{Gnucash} or \gls{CSV}  formats
		\item incremental import
		\item automatic reconciliation
		\end{itemize}
	\item Beginning of the Italian translation by Giorgio \familyname{Mandolfo}
	\item Hidden exchange rates on a session to avoid re-entering them
	\item Support for text attributes (bold, italic, large, small) for states
	\item New logo with mascot Grisbi on a Euro sign (€) (thanks to André \familyname{Pascual})
	\item Editable animated waiting logo
	\item Completely case-sensitive lists of characters
	\item Improved entry of bank details
	\item Keyboard support in the tree structure of payees, categories and budget charges
	\item Clickable automatic term maturities
	\item Breakdown of due dates
	\item Transactions convertible into instalments
	\item Operations that can be moved to another account
\end{itemize}

\subsection{What's new in version 0.8}

\begin{itemize}%\itemsep=-3pt
	\item Budget module in the basic version
	\item Credit simulator and amortization tables with the possibility to print and export the data in a spreadsheet
	\item Amortization table for liability accounts with the ability to print and export data in a spreadsheet
	\item Local Settings Management (Date Format, Decimal Separators and Thousands)
	\item Embedding custom icons in the accounts file
	\item Shading of different periods in the scheduler
	\item Cloning scheduled operations
\end{itemize}

% page break for solidarity title

\newpage

\subsection{What's new in version 1.0}

\begin{itemize}%\itemsep=-3pt
	\item Graphics on forecasts
	\item Credit card management (immediate or deferred debit)
	\item Business accounting with a Chart of Accounts
	\item Grisbi icon in the \indexword{ \gls{SVG}} \index{svg} formats 
	\item Even more context menus on the right mouse button
	\item Importing category files into budget allocations
	\item Changing the account display order by drag and drop in the account list
	\item Calculation of the balance according to the date of the transaction or its valuation date
	\item ??Clear Credit and Flow fields option for auto-complete??
	\item View unused payees
	\item Full-screen display command by function key \key{F11}
	\item Keyboard shortcut \key{Ctrl} \key{T} for the call of a new operation
	\item Direct access to the user manual through the menu \menu{Help} or the keyboard shortcut \key{Ctrl} \key{H}
\end{itemize}

\subsection{And for the future?}

Future releases (unstable 1.1 and stable 1.2) will still use the \gls{GTK} version~2 library.

Future versions (unstable 1.3 and stable 1.4) will benefit from the new GTK version~3 library.

\section{Contacts\label{introduction-contacts}}

\strong{Translators Note:} Many of the following links are now out of date or invalid.  These links will be fixed once the French version of the manual has been updated.

In addition to emailing the authors, you have several mailing lists you can contact us or obtain information through.

To keep abreast of developments in Grisbi, you can register on the \lang{Information list}\footnote{\urlListInfoEmail{}} provided for this purpose.  You will then receive an email at the release of each new version.

If you want to participate in the development of Grisbi, there is a \lang{development list}\footnote{\urlListDevelEmail{}}.

If you want to regularly compile the latest version of Grisbi's code from \gls{Git}, you will find it advantageous to subscribe to the \lang{Git information list}\footnote{\urlListCVSEmail{}} to be notified of the latest \cmd{commit}.

In addition, we decided to undertake the internationalization of Grisbi and, if you wish to help us, you can first contact us on the development list.

Specialized lists will be created for each language as needed.

For now, only the \lang{English translation list}\footnote{\urlListAnglaiseEmail{}} exists.

To subscribe to one of these lists, simply go to the page
\urlListSF{}\emph{list} replacing \emph{list} with the name of the list you are interested in. Or go to 
\lang{Grisbi}\footnote{\urlGrisbi{}} and look for \cmd{Contacts} there.

In addition do not hesitate to regularly consult Grisbi's official website.

% page break for solidarity title

\newpage

\section{Authors and contributors\label{introduction-authors}}

{Cédric \familyname{Auger}}\footnote{\urlCedricAugerEmail{}} is the basis of the project, and continues to be a developer.

{Daniel \familyname{Cartron}}\footnote{\urlDanielCartronEmail{}} wrote the documentation up to version 0.4.0, provided accounting advice and ergonomics, and created the first Grisbi site. His passion for ultra-compliant accounts files brings an undeniable bonus to the discovery of unpublished bugs.

{André \familyname{Pascual}}\footnote{\urlAndrePascualEmail{}},

\lang{Linuxgraphic}\footnote{\urlLinuxGraphic{}}, is the author of our logo.

{Sébastien \familyname{Blondeel}}\footnote{\urlSebastienBlondeelEmail{}} wrote the scripts to produce the different formats of the documentation and those related to the conversion of images to the appropriate formats. He is also the architect of the adoption of \gls{LaTeX} for writing the documentation. In addition, his experience in electronic publishing makes him a valuable advisor and source of many suggestions.

{Benjamin \familyname{Drieu}}\footnote{\urlBenjaminDrieuEmail{}}, developer for Grisbi and official packager for \gls{Debian}.

{Alain \familyname{Portal}}\footnote{\urlDionysosEmail{}}, who was starting to get bored in \gls{RedHat} packaging. His love of a job well done and his obstinacy make him, for the moment, a bug fixer. He also participates in the compilation of the documentation. He wants to start coding in the unstable version.

{Loic \familyname{Breilloux}}\footnote{\urlLoicBreillouxEmail{}} has updated the documentation for version 0.5.1 and will try to update the documentation for future releases.

{Gerald \familyname{Niel}}\footnote{\urlGeraldNielEmail{}} replaced {Daniel \familyname{Cartron}} in the role of webmaster and is therefore the creator of the new version of \lang{Grisbi}\footnote{\urlGrisbi{}}. He is also responsible for \gls{Slackware} packages.

{Juliette \familyname{Martin}}\footnote{\urlJulietteEmail{}} has the thankless task of proofreading of the documentation. If there are any mistakes, it's certainly they were well hidden for escaping his attentive eyes

{François \familyname{Terrot}}\footnote{\urlFrancoisTerrotEmail{}} joined the team to create Grisbi's \gls{build} for Windows.

{Pierre \familyname{Biava}}\footnote{\urlPierreBiavaEmail{}} joined the development team in 2008.

{Didier\familyname{Chevalier}}\footnote{\urlDidierChevalierEmail{}},{William \familyname{Ollivier}}\footnote{\urlWilliamOllivierEmail{}} and {Mickael \familyname{Remars}}\footnote{\urlMickaelRemarsEmail{}} have also participated in development.

{Jean-Luc \familyname{Duflot}}\footnote{\urlJeanLucDuflotEmail{}} made a big update of the manual for the 0.6 version, which was needed since 2004, and continued on with the 0.8 and the 1.0 versions too.

{Alain \familyname{Letient}}\footnote{\urlAlainLetientEmail{}} tenaciously re-read the 0.6 manual and created its iconography, and also continued with versions 0.8 and 1.0.

{Guy \familyname{Lebègue}}\footnote{\urlGuyLebegueEmail{}}, first for version 0.8, then with {Michèle\familyname{Bondil}}\footnote{\urlMicheleBondilEmail{}} for 1.0, created the business accounting option, which requires many specialist accounting skills.

\section{Acknowledgments \label{introduction-thanks}}

Thanks to \lang{TuxFamily}\footnote{\urlTuxFamily{}} who has long made available to us all the tools we needed to develop Grisbi 
(website, ftp, CVS, mailing lists, etc.). Alas, the attacks inflicted by hackers in late 2003 - early 2004 on \lang{TuxFamily} have forced us to seek a new web home. So today we thank \lang{SourceForge}\footnote{\urlSourceForge{}}, the platform to which we migrated.  We wish a quick and quick recovery to \lang{TuxFamily} that is sorely lacking hundreds of free projects.

A big thank you also to all the contributors on the development list who helped Grisbi's evolution through their suggestions, remarks and bug reports, as well as to the many readers of the \menu{User Manual}, which contribute to make it a better tool.

\section{Licenses \label{introduction-licenses}}

The program is subject to the terms of the \gls{GNU General Public License}.  Bug fixes and updates are not guaranteed.  The manual is subject to the terms of the \gls{GNU Free Documentation License}.

Permission is granted to copy, distribute and / or modify this document under the terms of the GNU Free Documentation License Version 1.1 or any later version published by the Free Software Foundation.

\section{About this manual \label{introduction-manual}}

This is version \actuality{} 1.0 of the manual, dated \actuality{} February 12, 2014, which corresponds to the version 1.0 of Grisbi software.  This manual was written with the \gls{LaTeX} \index{latex @ \LaTeX} \gls{text formatter}.
% and is available in PDF or HTM \gls{file formats}}, with or without illustrations ( screenshots) in these two formats.

It can be accessed directly in the Grisbi software via the menu under \menu{Help - Manual} from the menu bar, in HTML format.
% and with illustrations.

%However, all these different formats and versions can be downloaded from the  \lang{Grisbi}\footnote{\urlGrisbi{}} site or the \lang{Sourceforge}\footnote{\urlSourceForgeDocumentation{}} site.

The tools needed to read the various manual formats are presented in the  \vref{introduction-manual-readers} section \menu{Reading software}.

\subsection{Introduction \label{introduction-manual-presentation}}

Although Grisbi is designed to be as intuitive as possible and most functions are immediately understandable, a reference manual is needed. This manual has been designed according to the following principles:

\begin{itemize}%\itemsep=-3pt
	\item the most comprehensive possible, covering all the program features;
	\item to facilitate navigation through the manual its chapters are organized as closely as possible according to a uniform format: presentation of the chapter - description of the display, description of the functions, 
	\item Editing recurring paragraphs from one chapter to another in the most identical way possible, to facilitate quick reading;
	\item search for information facilitated by numerous \gls{hyperlinks}, an index and a glossary.
\end{itemize}

% space for theme change: 5 mm

\vspacepdf{5mm}

Here is a brief description of the different chapters:

\begin{itemize}%\itemsep=-3pt
	\item \menu{Preamble} Introduces the English translation and explains the origin of the name given to this software;
	\item \menu{Introduction} Introduces the software, the manual, their authors and contacts;
	\item \menu{First start Grisbi} is the \emph{essential} chapter to help you to start using the software;
	\item \menu{Home} describes the main elements of the GUI and their manipulation with the mouse and the keyboard (shortcuts);
	\item \menu{Export and Import Accounts} describes how to exchange data with other software;
	\item \menu{Data Management} presents the options of the account files, backups and archives and their management;
	\item \menu{Account Management} describes the properties of accounts, their management and the different types of accounts with their use;
	\item \menu{Account Operations} describes the operations on accounts, the information and input fields used and their management
	\item \menu{Bank Reconciliation} details the procedure for reconciling an account and managing reconciliations;
	\item \menu{Schedule} describes the planning of future entries and their manipulation;
	\item \menu{Searches} takes stock of data search possibilities;
	\item \menu{Payees}, \menu{Categories}, \menu{budgetary allocations} and \menu{Exercises} describe the management of this data;
	\item \menu{Credit Simulation} describes methods and simulation management;
	\item \menu{Forecast Budgets} describes the tools and procedures for budget creation and depreciation tables, as well as their management;
	\item \menu{Credit card management and their prediction} describes the management of these cards, including deferred debit cards, and forecasting methods;
	\item \menu{Business Accounting} presents two introductions for business managers;
	\item \menu{Reports} and \menu{Report creation} describe the management and creation of reports;
	\item \menu{Configuration of Grisbi} details all the possibilities of setting the software;
	\item \menu{Maintenance Tools} gives some tips to use in case of errors or bugs.
\end{itemize}

\subsection{Typographical conventions of this manual \label{introduction-manual-conventions}}

The following list defines and illustrates the typographic conventions used as visual aids for identifying particular elements of the text of the document:

\begin{itemize}%\itemsep=-3pt
	\item the interface components are window titles, icon and button names, menu names, and other options that appear on the monitor screen; they are presented as follows: \newline
\hspace {1.5cm} click \menu{Back};
	\item the keyboard key label represents what is written on the keyboard keys; it is presented as follows: \newline
\hspace {1.5cm} press the \key{Enter} key;
	\item key combinations are a series of keys to be pressed simultaneously (unless otherwise specified) to perform a single function; they are presented as follows: \newline
\hspace {1.5cm} press the key combination \key{Ctrl} \key{R};
	\item The commands that are part of an instruction and that must be entered are presented as follows:\newline
\hspace {1.5cm} type \cmd{grisbi} to start the program;
	\item file and directory names are shown as: \newline
\hspace {1.5cm} \file{graybi-n.n.n.rpm} and \file{/ usr / local / bin};
	\item the command lines consist of a command and can include one or more possible parameters of the command; they are presented as follows: \newline
\hspace {1.5cm} \cmd{rpm -Uvh graybi-n.n.n.rpm};
	\item any sequence of alphanumeric characters in blue, in the document in PDF or HTML format, is a hypertext link, referring to either an image, another part of the document, an indexed word or to the glossary (for PDF only);
	\item The words or groups of words referenced in the index are highlighted in the chapters as follows:\newline
\hspace {1.5cm} \textsf{referenced term} for PDF format; \newline
\hspace {1.5cm} in brown for HTML.
\end{itemize}

% space before Attention or Note: 5 mm

\vspacepdf{5mm}

In addition, a \textbf{Note} underlines a particular point to take into account, while a \strong{Attention} indicates either a very important point for the understanding, or an error not to do under pain of important damage for your data ; a \strong{Warning} is \emph{to be respected}.

\subsection{Reading software \label{introduction-manual-readers}}

To read this document, we recommend the use of free software, which respects all your privacy and the confidentiality of your data; the following software has the features of \gls{hyperlinks}:

\begin{itemize}%\itemsep=-3pt
	\item for PDF format:
\begin{itemize}%\itemsep=-3pt
	\item Linux: Evince, Epdfviewer, Firefox, Ghostscript, MuPDF, Okular,
	\item Mac: Okular, Vindaloo, Xpdf,
	\item Windows: Evince, Firefox, MuPDF, Okular, SumatraPdf;
\end{itemize}
	\item for the HTML format:
\begin{itemize}%\itemsep=-3pt
	\item Linux: Arora, Dillo, Firefox, Kazehakase, Links2, Midori, SeaMonkey,
	\item Mac: Arora, Camino, Firefox, SeaMonkey, Shiira,
	\item Windows: Arora, Firefox, Iron, K-Meleon, Midori, SeaMonkey.
\end{itemize}

\end{itemize}

% space for theme change: 5 mm

\vspacepdf{5mm}

In short, you have the choice!

% space for theme change: 5 mm

\vspacepdf{5mm}

These programs are all downloadable on their own website and are all under a license of \gls{free software}, and you can also find them on the \lang{Framasoft}\footnote{\urlFramasoftLogiciels{}} site.
			% TODO update screenshots and text "intro", uncomment when finished
%\cleardoubleemptypage

%\include{03-grisbi-manuel-entrance-en}		% TODO update screenshots and text "entrance", uncomment when finished
%\cleardoubleemptypage

%\include{04-grisbi-manuel-start-en}			% TODO update screenshots and text "start", uncomment when finished
%\cleardoubleemptypage

%\include{05-grisbi-manuel-home-en}			% TODO update screenshots and text "home", uncomment when finished
%\cleardoubleemptypage

%\include{06-grisbi-manuel-QIF-en}			% TODO update screenshots and text "QIF", uncomment when finished
%\cleardoubleemptypage

%\include{07-grisbi-manuel-datamanagement-en}	% TODO update screenshots and text "datamanagement", uncomment when finished
%\cleardoubleemptypage

%\include{08-grisbi-manuel-accounts-en}		% TODO update screenshots and text "accounts", uncomment when finished
%\cleardoubleemptypage

%\include{09-grisbi-manuel-transactions-en}	% TODO update screenshots and text "transactions", uncomment when finished
%\cleardoubleemptypage

%\include{10-grisbi-manuel-reconciliation-en}	% TODO update screenshots and text "reconciliation", uncomment when finished
%\cleardoubleemptypage

%\include{11-grisbi-manuel-planned-en}		% TODO update screenshots and text "planned", uncomment when finished
%\cleardoubleemptypage

%\include{12-grisbi-manuel-search-en}			% TODO update screenshots and text "search", uncomment when finished
%\cleardoubleemptypage

%\include{13-grisbi-manuel-third-en}			% TODO update screenshots and text "third", uncomment when finished
%\cleardoubleemptypage

%\include{14-grisbi-manuel-categories-en}		% TODO update screenshots and text "categories", uncomment when finished
%\cleardoubleemptypage

%\include{15-grisbi-manuel-budgetlines-en}	% TODO update screenshots and text "budgetlines", uncomment when finished
%\cleardoubleemptypage

%\include{16-grisbi-manuel-financialyear-en}	% TODO update screenshots and text "financialyear", uncomment when finished
%\cleardoubleemptypage

%\include{17-grisbi-manuel-credit-en}			% TODO update screenshots and text "credit", uncomment when finished
%\cleardoubleemptypage

%\include{18-grisbi-manuel-budget-en}			% TODO update screenshots and text "budget", uncomment when finished
%\cleardoubleemptypage

%\include{19-grisbi-manuel-bankcardmanagement-en}	% TODO update screenshots and text "bankcardmanagement", uncomment when finished
%\cleardoubleemptypage

%\include{20-grisbi-manuel-association}		% fr version only
%\cleardoubleemptypage

%\include{21-grisbi-manuel-reports-en}			% TODO update screenshots and text "reports", uncomment when finished
%\cleardoubleemptypage

%\include{22-grisbi-manuel-reports-creation-en}	% TODO update screenshots and text "reports-creation", uncomment when finished
%\cleardoubleemptypage

%\include{23-grisbi-manuel-setup-en}				% TODO update screenshots and text "setup", uncomment when finished
%\cleardoubleemptypage

%\include{24-grisbi-manuel-maintenance-en}		% TODO update screenshots and text "maintenance", uncomment when finished
%\cleardoubleemptypage



%% Files editors not used anymore, so useless 
%%\include{grisbi-manuel-todo}
%%\cleardoubleemptypage
%
%% Files editors not used anymore, so useless 
%%\include{grisbi-manuel-problems}
%%\cleardoubleemptypage
%
%% Useless so don't include it
%%\include{grisbi-manuel-XML}
%
%% Useless so don't include it
%%\include{grisbi-manuel-FDL}


% Prints the index
\printindex


% Displays a note at the beginning of the glossary
\renewcommand{\glossarypreamble}{\textbf{Note}: most of the definitions in this glossary are taken from articles of the same name in the free, collaborative encyclopaedia \lang{Wikipédia}
\footnote{\urlWikipedia{}}. Although these texts have been modified and adapted to the specific context of this glossary, the author would like to thank Wikipedia for providing these references.\newline}


% Prints the glossary
% For pdf only; redefined in hva/macros.hva by an empty command in html
\printglossaries


\end{document}



