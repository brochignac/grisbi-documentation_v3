%%%%%%%%%%%%%%%%%%%%%%%%%%%%%%%%%%%%%%%%%%%%%%%%%%%%%%%%%%%%%%%%%
% Contents : The introduction chapter
% $Id : grisbi-manuel-intro.tex,v 0.4 2002/10/27 Daniel Cartron
% $Id : grisbi-manuel-intro.tex, v 0.5.0 2004/06/01 Loic Breilloux
% $Id : grisbi-manuel-intro.tex, v 0.6.0 2011/11/17 Jean-Luc Duflot
% $Id : grisbi-manuel-intro.tex, v 0.8.9 2012/04/27 Jean-Luc Duflot
% $Id : grisbi-manuel-intro.tex, v 1.0 2014/02/12 Jean-Luc Duflot
% $Id : grisbi-manuel-intro.tex, v 3.0 2024/11 Dominique Brochard :
% - update links and text
% - update pdf/html readers
% - add versions 0.1 to 0.4.4
% - use \label within \footnote command, and after use \footref to avoid duplicates in footnotes
% - add newsgroups in 2.4 Contacts
%%%%%%%%%%%%%%%%%%%%%%%%%%%%%%%%%%%%%%%%%%%%%%%%%%%%%%%%%%%%%%%%%

\chapter{Einführung\label{introduction}}


\section{Allgemeines\label{introduction-general}}


%Grisbi est un \gls{logiciel libre} de comptabilité personnelle, développé en \gls{Langage C} avec le support \gls{GTK}+ 3, originellement pour la plate-forme \gls{GNU/Linux}. Il y a maintenant un \indexword{\gls{portage}}\index{portage} sous \gls{Windows}, \gls{macOS}, FreeBSD, des paquetages pour différentes \gls{distributions Linux}, et d'autres possibilités à découvrir sur le site de \lang{Grisbi}\footnote{\urlGrisbi{}} ou celui de {Sourceforge}\footnote{\urlSourceForge{}}.
Grisbi ist eine \gls{Freie Software} für persönliche Buchhaltung, die in \gls{Programmiersprache C} mit \gls{GTK}+ 3-Unterstützung entwickelt wurde, ursprünglich für die Plattform {GNU/Linux}. Es gibt nun eine \gls{Portierung} unter \gls{Windows}, \gls{macOS}, FreeBSD, Pakete für verschiedene \gls{Linux-Distribution} und weitere Möglichkeiten, die Sie auf der Website von \lang{Grisbi}\footnote{\urlGrisbi{}} oder der von \lang{Sourceforge}\footnote{\urlSourceForge{}} entdecken können.


%Le principe de base est de vous permettre de classer de façon simple et intuitive vos opérations financières, quelles qu’elles soient, afin de pouvoir les exploiter au mieux en fonction de vos besoins.
Das Grundprinzip besteht darin, dass Sie Ihre Finanztransaktionen jeglicher Art einfach und intuitiv klassifizieren können, um sie entsprechend Ihren Bedürfnissen optimal nutzen zu können.

%Grisbi a pris le parti de la simplicité et de l'efficacité pour un usage de base, sans toutefois exclure la sophistication nécessaire à un usage plus avancé. Les fonctionnalités futures tenteront toujours de respecter ces critères.
Grisbi hat sich zum Ziel gesetzt, einfach und effizient für den grundlegenden Gebrauch zu sein, ohne jedoch die Raffinesse auszuschließen, die für einen fortgeschrittenen Gebrauch erforderlich ist. Zukünftige Funktionen werden immer versuchen, diese Kriterien zu erfüllen.


\section{Funktionalitäten\label{introduction-features}}


\subsection{Was Grisbi kann}%Ce que Grisbi sait faire

\begin{itemize}
	\item Von Franzosen entwickelte Software, die vollständig mit der französischen Logik der Buchführung übereinstimmt.%Logiciel développé par des Français, donc en totale conformité avec la logique française de la comptabilité
	\item Einfache, intuitive Benutzeroberfläche mit Vollbildschirmsteuerung.%Interface simple et intuitive, avec commande d'affichage plein écran
	\item Mehrkonten- und Mehrbenutzerverwaltung%Gestion multi-comptes et multi-utilisateurs
	\item Lokalisierung (Datum, Dezimal und Tausender Trennzeichen)%Gestion des paramètres locaux (dates, séparateurs décimal et des milliers)	
	\item Bankkonten, Bargeldkonten, Kreditkonten, Anlagekonten%Comptes bancaires, de caisse, d'actif et de passif
	\item Verwaltung von mehreren Währungen mit Wechselkursen und Wechselgebühren%Gestion multi-devises, avec prise en charge des taux de change et des frais de change
	\item Kreditkartenmanagement (sofortige oder verzögerte Belastung)%Gestion des cartes bancaires (débit immédiat ou différé)	
	\item Beschreibung der Buchungen mit: Datum, Valutadatum, Geschäftsjahr, Empfänger, Betrag, Währung, Kategorien und Unterkategorien, Haushaltszuweisung und Unterzuweisung (zur Analyse der Ausgaben), Notizen, Buchungsnummer (von Grisbi vergeben), Buchungsbelegnummer, Bankreferenz%Description des opérations avec : date, date de valeur, exercice, tiers, montant, devise, catégorie et sous-catégorie, imputation et sous-imputation budgétaires (permettant de tenir une comptabilité analytique), remarque, numéro d'écriture (attribué par Grisbi), numéro de pièce comptable, références bancaires
	\item Auto-Vervollständigung von Empfänger, Kategorien und Haushaltsverbuchungen mit automatischem Abruf von Buchungen und Unterbuchungen für einen bestimmten Empfänger%Auto-complètement des tiers, catégories et imputations budgétaires avec rappel automatique des  opérations et des sous-opérations pour un tiers donné
	%\item Option d'effacement des champs Crédit et Débit pour l'auto-complètement
	\item Berechnung vom Saldo mit Buchung oder Valutadatum%Calcul du solde en fonction de la date de l'opération ou de sa date de valeur
	\item Überweisungen zwischen Konten, auch in verschiedenen Währungen, mit automatischer Gegenbuchung%Virement entre comptes, y compris de devises différentes, avec contre-écriture automatique
	\item Terminierung von Buchungen mit automatischer oder manueller Validierung%Planification d'opérations avec validation automatique ou manuelle
	%\item Surveillance des échéances
	\item Verschieben und Klonen von Buchungen%Déplacement et clonage des opérations
	\item Klonen von geplanten Buchungen%Clonage des opérations planifiées
	\item Einfärbung der Ausgaben im Planer%Colorisation des débits dans l'échéancier
	\item Finanzanalyse und -berichte mit dem leistungsstarken Modul zur Erstellung von Berichten%Analyse et rapports financiers grâce au puissant module de génération d'états
	\item Mehrere vorformatierte Berichte verfügbar und anpassbar%Plusieurs états pré-formatés disponibles et personnalisables
	\item Virtuelle Empfänger, die durch Berichte erstellt werden%Tiers virtuels créés par les états
	\item Drucken von Berichten%Impression des états
	\item Kreditsimulationen und Tilgungstabellen mit Ausdruck und Datenexport%Simulation de crédits et tableaux d'amortissement avec impression et export des données
	\item Budgetvorschau mit Grafiken zu den Prognosen und historischen Daten%Budget prévisionnel avec graphiques sur les prévisions et les données historiques
	%\item Comptabilité d'association avec plan comptable
	%\item Ordre d'affichage des comptes par glisser/déposer dans la liste des comptes
	\item Datenimport in den Formaten \gls{QIF}, \gls{OFX}, \gls{CSV} und aus \gls{Gnucash}%Import de fichiers aux formats \gls{QIF}, \gls{OFX}, \gls{Gnucash} ou \gls{CSV}
	\item Datenimport für Kategorien und Gruppen%Import de fichiers de catégories dans les imputations budgétaires
	\item Datenexport in den Formaten \gls{QIF} und \gls{CSV}%Export de fichiers aux formats \gls{QIF} ou \gls{CSV}
	\item Benutzerdefinierte Icons in der Kontodatei%Icônes personnalisées dans le fichier de comptes
	\item Ikone von Grisbi im Format \indexword{\gls{SVG}}\index{SVG}%Icône de Grisbi au format \indexword{\gls{SVG}}\index{svg}
	\item Noch mehr Kontextmenüs auf der rechten Maustaste (Navigationsleiste)%Encore plus de menus contextuels sur le bouton droit de la souris (panneau de navigation)
	\item Zahlreiche Tastenkombinationen für eine gute Ergonomie%Nombreux raccourcis-clavier pour une bonne ergonomie
\end{itemize}


\subsection{Was Grisbi noch nicht weiß}%Ce que Grisbi ne sait pas encore faire

\begin{itemize}
	\item Automatische Aufschlüsselung von Kreditrückzahlungen%La ventilation automatique des remboursements d'emprunt
	\item Online Abgleich mit Geldinstitut%Le rapprochement par Internet
\end{itemize}


\section{Entwicklung der Software}%Évolution du logiciel


\subsection{Entwicklung und Versionen}Développement et versions

%Grisbi étant un logiciel en plein développement, toute remarque (idée, bogue, documentation\dots{}) est la bienvenue. Vous pouvez l'envoyer aux différentes listes indiquées dans la section \vref{introduction-contacts} \menu{Contacts} ou sur le site de \lang{Grisbi}\footnote{\urlGrisbi{}}.
Da Grisbi eine Software ist, die sich noch in der Entwicklung befindet, ist jedes Feedback (Ideen, Fehler, Dokumentation\dots{}) willkommen. Sie können sie an die verschiedenen Listen senden, die im Abschnitt \vref{introduction-contacts} \menu{Kontakte} oder auf der Website von \lang{Grisbi}\footnote{urlGrisbi{}} angegeben sind.

%Vous pouvez, si vous avez le goût de l'aventure, télécharger et compiler la dernière version en cours de développement sur le système de gestion de contenu \gls{GitHub}\footnote{\urlGitHubGrisbi{}\label{siteGitHubGrisbi}} utilisant \gls{Git}.
Wenn Sie abenteuerlustig sind, können Sie die neueste Version, die derzeit auf dem Content-Management-System \gls{GitHub}\footnote{\urlGitHubGrisbi{}\label{siteGitHubGrisbi}} entwickelt wird, unter Verwendung von \gls{Git} herunterladen und kompilieren.

%En effet, le code des nouvelles versions de Grisbi est bien souvent figé plusieurs semaines avant la distribution effective de celles-ci, le temps pour l'équipe de développement de vérifier et d'éradiquer les ultimes bogues. Pendant cette période, le format du fichier de comptes ne bouge plus, et vous pouvez donc, avec un minimum de précautions (sauvegardes fréquentes, etc.), bénéficier des dernières améliorations avec plusieurs semaines d'avance, et également participer au débogage.
Der Code neuer Grisbi-Versionen wird oft mehrere Wochen vor der eigentlichen Veröffentlichung eingefroren, damit das Entwicklungsteam die letzten Fehler überprüfen und beseitigen kann. Während dieser Zeit bleibt das Format der Account-Datei unverändert, so dass Sie mit einem Minimum an Vorsichtsmaßnahmen (häufige Backups etc.) die neuesten Verbesserungen mehrere Wochen im Voraus nutzen und auch an der Fehlerbeseitigung teilnehmen können.

%Vous pouvez enfin avoir accès à toutes les évolutions du code postérieures à la version 0.3.2 (version à partir de laquelle le code est disponible sur le site de \gls{GitHub}\footref{siteGitHubGrisbi}).
Schließlich haben Sie Zugriff auf alle Entwicklungen des Codes nach Version 0.3.2 (Version, ab der der Code auf der \gls{GitHub}\footref{siteGitHubGrisbi}-Website verfügbar ist).

%Notez qu'à partir de la version 0.6, les versions de numéro mineur pair (par ex. 0.\underline8) sont des versions stables, tandis que celles de numéro mineur impair sont instables et ne devraient pas être utilisées en conditions normales ; de ce fait seules les versions stables sont mentionnées ici.
Beachten Sie, dass ab Version 0.6 Versionen mit gerader Minor Number (z. B. 0.\underline8) stabile Versionen sind, während Versionen mit ungerader Minor Number instabil sind und unter normalen Umständen nicht verwendet werden sollten; daher werden hier nur die stabilen Versionen erwähnt.

% saut de page pour titre solidaire
\newpage

\subsection{Version 0.1 \textnormal{(10.04.2000)}}

\begin{itemize}
	\item Verwaltung mehrerer Konten%Gestion de plusieurs comptes
	\item Erstellen, Ändern und Löschen von Konten und Vorgängen (endlich das Minimum, um zu funktionieren\dots{})%Création, modification et suppression de comptes et d'opérations (enfin le minimum pour fonctionner\dots{})
	\item Fähigkeit, Buchungen aufzuzeigen%Possibilité de pointer les opérations
	\item Und die Datensicherung, ohne die das alles nicht viel Sinn machen würde!%Et la sauvegarde, sans laquelle tout cela n'aurait pas beaucoup d'intérêt !
\end{itemize}

\subsection{Version 0.2 \textnormal{(19.06.2000)}}

\begin{itemize}
	\item Automatischer Import von Kontodateien aus früheren Versionen%Importation automatique des fichiers de comptes de versions précédentes
	\item Verwenden von Listen für Empfänger, Kategorien\dots{} (bequemer als alles einzutippen)%Utilisation de listes pour les tiers, catégories\dots{} (plus pratique que de tout taper)
	\item Automatische Vervollständigung während der Eingabe und Ausfüllen am Ende des Vorgangs wie beim vorherigen ähnlich%Saisie automatique lors de la frappe et remplissage de la fin de l'opération comme la précédente semblable
	\item Verwaltung von Überweisungen zwischen Konten%Gestion des virements entre comptes
	\item Vereinfachte oder vollständige Anzeige der Buchungen%Affichage des opération simplifié ou complet
	\item Ausgleichen von Konten%Équilibrage de comptes
\end{itemize}

\subsection{Von Version 0.3.0 \textnormal{(10.12.2001)} bis Version 0.3.3 \textnormal{(15.11.2002)}}%De la version 0.3.0 à la version 0.3.3}

\begin{itemize}
	\item Verwaltung geplanter oder zyklischer Buchungen (Planer)%Gestion des opérations prévues ou cycliques (échéancier) 
	\item Übernahme von Währungen und der Umstellung auf den Euro (1. Januar 1999 bis 1. Januar 2002)%Prise en charge des devises et du passage à l'euro (du 1er janvier 1999 au 1er janvier 2002)
	\item Import/Export hinzugefügt und Import von \gls{QIF}-Dateien verbessert%Ajout de l'import/export et amélioration de l'import de fichiers \gls{QIF}
	\item Hinzufügen der Splittbuchung%Ajout de la ventilation
	\item Mehrbenutzerunterstützung%Support multi-utilisateur
	\item Für jede Bank und jedes Konto, Hinzufügen von Details wie%Pour chaque banque et compte, ajout des détails comme:
		\begin{itemize}
		\item[\textopenbullet] Vollständige Nummern des zugehörigen Kontos, des Schalters und der Bankleitzahl%numéros complets du compte associé, du guichet et du code banque
		\item[\textopenbullet] Unterschiedlicher Inhaber für jedes Konto mit persönlicher Adresse%titulaire différent pour chaque compte avec adresse personnelle
		\item[\textopenbullet] Kontaktdaten eines Korrespondenten%coordonnées d'un correspondant
		\end{itemize}
	\item Hinzufügen der Registerkarte Kategorien%Ajout de l'onglet catégories
	\item Möglichkeit, den Namen und den Code der Währungen zu ändern%Possibilité de changer le nom et le code des devises
	%\item Le classement alphabétique prend (enfin) en compte les accents
	\item Speichert das letzte Arbeitsverzeichnis und die zuletzt geöffneten Dateien%Mémorise le dernier répertoire de travail et les derniers fichiers ouverts
	\item Eine eindeutige Nummer pro Buchung unabhängig vom Konto%Un numéro unique par opération quel que soit le compte
	\item Die Tasten \key{+} (\key{-}) in einem Datumsfeld inkrementieren (dekrementieren) das Datum%Les touches \key{+} (\key{-}) dans un champ de date incrémentent (décrémentent) la date
	\item Mögliche Anzeige neuer Felder im Formular für die Eingabe von Buchungen%Affichage possible de nouveaux champs dans le formulaire de saisie des opérations 
	\item Einstellbare Fälligkeiten für einen bestimmten Zeitraum%Échéances paramétrables pour un laps de temps défini
	\item Anzeige des aktuellen gepunkteten Saldos%Affichage du solde courant pointé
	\item Möglichkeit, das Datum in der Form ttmm, ttmmjj oder ttmmjjj einzugeben%Possibilité d'entrer la date sous la forme jjmm, jjmmaa ou jjmmaaaa
	\item Transaktionssummen auf den Registerkarten Empfänger, Kategorien und Budgetzuweisungen%Totaux des opérations dans les onglets Tiers, Catégories et Imputations Budgétaires (IB)
	\item Logo und Schriftart der Anzeige können individuell gestaltet werden%Personnalisation possible du logo et de la police d'affichage.
	\item Einführung der Tools Automake und Autoconf, mit denen die Zusammenstellung der Quelldateien der Software vereinfacht werden kann%Mise en place des outils Automake et Autoconf permettant de simplifier la compilation des fichiers sources du logiciel
\end{itemize}

%\newpage			% saut de page pour titre solidaire

\subsection{Von Version 0.4.0 \textnormal{(15.02.2003)} bis Version 0.4.4 \textnormal{(25.03.2004)}}%De la version 0.4.0 à la version 0.4.4}

\begin{itemize}
	\item Anpassen des Layouts der Benutzeroberfläche%Personnalisation de l'agencement de l'interface
	\item Möglichkeit des Imports/Exports:%Possibilité d'import/export:
		\begin{itemize}
		\item[\textopenbullet] von Kategorielisten%de listes de catégories
		\item[\textopenbullet] Listen von Gruppen%de listes d'imputations budgétaires (IB)
		\item[\textopenbullet] von erstellten Berichten%d'états créés
		\end{itemize}	
	\item Finanzberichte wurden hinzugefügt%Ajout des rapports financiers
	\item Grisbi ist nun internationalisiert und die Übersetzungen wurden verbessert%Grisbi est désormais internationalisé et les traductions sont améliorées
	\item Anzeige von Salden in mehreren Währungen im Startbildschirm%Affichage des soldes multi-devises dans l'écran de démarrage
	\item Notizen können im Planner angezeigt werden%Les remarques peuvent être affichées dans l'échéancier
	\item Neues Tool, mit dem Beitragende die Grisbi-Dateien anonymisieren können, um die Vertraulichkeit vor dem Versand zu wahren%Nouvel outil permettant aux contributeurs d'anonymiser les fichiers Grisbi afin de garder la confidentialité avant de les envoyer
	\item Die Breite der Spalten im Fälligkeitsplan kann geändert werden%La largeur des colonnes de l'échéancier est modifiable
	\item Harmonisierung in den Datumsfeldern%Harmonisation dans les champs de date:
		\begin{itemize}
		\item[\textopenbullet] \key{Strg/Ctrl}\key{Enter} in einem Datumsfeld öffnet einen Kalender%\key{Ctrl}\key{Entrée} dans un champ de date ouvre un calendrier
		\item[\textopenbullet] Die Pfeiltasten sind in einem Kalender aktiv%les touches fléchées sont actives dans un calendrier
		\item[\textopenbullet] \key{Strg/Ctrl}\key{+} (\key{-}) in einem Datumsfeld erhöht (verringert) das Datum um etwa eine Woche%\key{Ctrl}\key{+} (\key{-}) dans un champ de date augmente (diminue) la date d'environ une semaine
		\item[\textopenbullet] \key{Bild auf} (\key{Bild ab}) erhöht (verringert) das Datum um etwa einen Monat%\key{Pg.Préc} (\key{Pg.Suiv}) augmente (diminue) la date d'environ un mois
		\item[\textopenbullet] \key{Strg/Ctrl}\key{Bild auf} (\key{Bild ab}) erhöht (verringert) das Datum um etwa ein Jahr%\key{Ctrl}\key{Pg.Préc} (\key{Pg.Suiv}) augmente (diminue) la date d'environ un an
		\end{itemize}	
\end{itemize}

\subsection{Von Version 0.4.5 \textnormal{(07.04.2004)} bis Version 0.6 \textnormal{(05.05.2010)}}%De la version 0.4.5 à la version 0.6}

\begin{itemize}
	\item Überarbeitung der grafischen Oberfläche%Refonte de l'interface graphique
	\item Umstellung auf \gls{GTK}+ 2%Support de la bibliothèque \gls{GTK}+ 2 pour un environnement plus sympathique et un portage simplifié sous Windows
	\item Entfernung \gls{Gnome} Abhängigkeiten%Plus aucune dépendance vis-à-vis de \gls{Gnome}
	\item Native Windows Version%Version Windows native (merci à François \familyname{Terrot})%\footnote{\urlFrancoisTerrotEmail{}}
	\item Vollständige \gls{UTF-8} Unterstützung%Support \gls{UTF-8} natif
	\item Berichte in \gls{LaTeX} drucken%Impression des états par \gls{LaTeX}
	\item Berichte in \gls{HTML} exportieren%Export des états en \gls{HTML}
	\item Verbesserte Benutzeroberfläche:%Amélioration de l'interface utilisateur:
		\begin{itemize}
		\item[\textopenbullet] Konfigurierbarkeit von Meldungen%amélioration des messages, les utilisateurs peuvent en ignorer
		\item[\textopenbullet] verbesserte Fehlerbehandlung%amélioration de la gestion des erreurs de segmentation
		\item[\textopenbullet] Neuanordnung von Menüpunkten%amélioration de la fenêtre des préférences
		\item[\textopenbullet] Kontextmenü in der Buchungsübersicht%menu contextuel sur la liste des opérations
		\item[\textopenbullet] zusätzliche Einstellungen%amélioration des items
		\end{itemize}
	\item Vollständig anklickbare Listen von Konten und Berichten%Listes de comptes et d'états complètement cliquables
	\item Speichern der Konfiguration ein einer \gls{XML} Datei%Configuration globale en \gls{XML} (merci à Axel Rousseau)
	\item Import und Export:%Réécriture de l'import de fichiers:
		\begin{itemize}
		\item[\textopenbullet] Unterstützung von \gls{QIF}, \gls{OFX}, \gls{Gnucash} und \gls{CSV} Formaten%support des formats \gls{QIF}, \gls{OFX}, \gls{Gnucash} ou \gls{CSV}
		\item[\textopenbullet] inkrementeller Import%import incrémental
		\item[\textopenbullet] automatische Abstimmung beim Import%rapprochement automatique
		\end{itemize}
	\item Neue Übersetzungen%Début de la traduction italienne par Giorgio \familyname{Mandolfo}
	%\item Taux de change cachés sur une session pour éviter de les ressaisir
	\item Unterstützung von Textattributen (fett, kursiv, groß, klein) für Berichte%Support des attributs de texte (gras, italique, grand, petit) pour les états
	\item Neues Logo mit Maskottchen Grisbi auf einem Eurozeichen (€)%Nouveau logo avec la mascotte Grisbi sur le symbole de l'euro (€) (merci à André \familyname{Pascual})
	\item Anzeige von Logo ist konfigurierbar%Logo d'attente animé modifiable
	\item Autovervollständigung ist nicht mehr case sensitive%Complètement des listes sensible à la casse des caractères
	\item Verwaltung von mehreren Banken%Amélioration de la saisie des détails des banques
	\item Tastaturbedienbarkeit für Empfänger, Kategorien und Gruppen%Support du clavier dans l'arborescence des tiers, catégories et imputations budgétaires
	\item Automatische klickbare Fälligkeitstermine%Échéances automatiques à terme cliquables
	\item Übersicht für geplante Buchungen%Échéances ventilables
	\item Geplante Buchungen können aus der Buchungsübersicht erstellt werden%Opérations convertibles en échéances
	\item Buchungen können auf andere Konten verschoben werden%Opérations déplaçables dans un autre compte
\end{itemize}


\subsection{Neuerungen in Version 0.8 \textnormal{(21.02.2011)}}%Nouveautés de la version 0.8}

\begin{itemize}
	\item Budgets (Basisversion)%Module budgétaire dans la version de base
	\item Kreditrechner mit Tilgungsplan (inklusive Druck und Export)%Simulateur de crédits et tableaux d'amortissement avec possibilité d'imprimer et d'exporter les données dans un tableur
	\item Abschreibungstabelle für Passivkonten mit der Möglichkeit, die Daten auszudrucken und in ein Tabellenkalkulationsprogramm zu exportieren%Tableau d'amortissement pour les comptes de passif avec possibilité d'imprimer et d'exporter les données dans un tableur
	\item Lokalisierung (Datum, Dezimal und Tausender Trennzeichen)%Gestion des paramètres locaux (format des dates, séparateurs décimal et des milliers)
	\item Konfigurierbare Symbole für Konten%Incorporation des icônes personnalisées dans le fichier de comptes
	\item Optische Hervorhebungen für Salden%Colorisation des débits dans l'échéancier
	\item Duplizieren von Buchungen%Clonage des opérations planifiées
\end{itemize}


\subsection{Neuerungen in Version 1.0 \textnormal{(17.03.2014)}}%Nouveautés de la version 1.0}

\begin{itemize}
	\item Grafiken zur Prognosen%Graphiques sur les prévisions
	\item Kreditkartenmanagement (sofortige oder verzögerte Belastung)%Gestion des cartes bancaires (débit immédiat ou différé)
	%\item Comptabilité d'association avec plan comptable		% for French accounting only
	\item Grisbi's Ikone im Format \indexword{\gls{SVG}}\index{SVG}%Icône de Grisbi au format \indexword{\gls{SVG}}\index{svg}
	\item Kontextmenü Erweiterungen%Encore plus de menus contextuels sur le bouton droit de la souris
	\item Datenimport für Kategorien und Gruppen%Import de fichiers de catégories dans les imputations budgétaires
	\item Anordnung von Konten ist mit Drag \& Drop änderbar%Modification de l'ordre d'affichage des comptes par glisser/déposer dans la liste des comptes
	\item Berechnung vom Saldo mit Buchung oder Valutadatum%Calcul du solde en fonction de la date de l'opération ou de sa date de valeur
	\item Zurücksetzen der Betragsfelder bei automatischer Datenbefüllung%Option d'effacement des champs Crédit et Débit pour l'auto-complètement
	\item Anzeige von unbenutzten Empfängern%Affichage des tiers inutilisés	
	\item Vollbildanzeige mit Funktionstaste \key{F11}%Commande d'affichage plein écran par la touche de fonction \key{F11}
	\item Erstellung Buchung mit Kurzbefehl \key{Strg/Ctrl}\key{T}%Raccourci-clavier \key{Ctrl}\key{T} pour l'appel d'une nouvelle opération
	\item Aufruf der \menu{Hilfe} mit Kurzbefehl \key{Strg/Ctrl}\key{H}%Accès direct au manuel de l'utilisateur par le menu \menu{Aide} ou le raccourci-clavier \key{Ctrl}\key{H}
\end{itemize}


\subsection{Neuerungen in Version 2.0 \textnormal{(11.01.2021)}}%Nouveautés de la version 2.0}

\begin{itemize}
	\item Umstellung auf \gls{GTK}+ 3 (seit Version 1.2.0)%Portage sous \gls{GTK}+ 3 (depuis la version 1.2.0)
	\item Hinzufügen eines Suchmoduls in der Liste der zugänglichen Buchungen, im Kontextmenü der Buchungen (Rechtsklick auf einen Buchung)%Ajout d'un module de recherche dans la liste des opérations accessibles, dans le menu contextuel des opérations (clic droit sur une opération)
	\item Automatische Erkennung/Integration von dunklen Farbschemen%Détection automatique/intégration de schéma de couleur foncée
	\item Neue Einstellungen für Farben%Nouveaux réglages pour les couleurs
	\item Konfiguration für allgemeine Schrittgröße%Configuration générale de la taille de police
	\item Verbesserte Darstellung auf Geräten mit geringer Auflösung%Affichage amélioré sur les dispositifs à faible résolution
	\item Überarbeitung vom Kreditmodul%Révision du module de crédit
	\item Importregeln für \gls{CSV} Dateien%Règles d'importation pour les fichiers \gls{CSV}
	\item Suchfunktionalität%Fonctionnalité de recherche
	\item Einstellung für die Löschung von alten Sicherungen%Réglage pour la suppression d'anciennes sauvegardes
	\item Fehlerbehebungen%Résolution de bogues
\end{itemize}


\subsection{Neuerungen in Version 3.0 \textnormal{(13.11.2023)}}%Nouveautés de la version 3.0}

\begin{itemize}
	\item Änderung der Suche nach dem Empfänger%Modification de la recherche du bénéficiaire
	\item Hinzufügen einer neuen Art von Konsumentenkredit%Ajout d'un nouveau type de prêt à la consommation
	\item Hinzufügen aller Transaktionen zum Archiv, wenn das Konto geschlossen wird%Ajout de toutes les transactions dans les archives lorsque le compte est clôturé
	\item Bereinigung des Codes%Nettoyage du code
	\item Vorbereitung der Umstellung auf \gls{GTK} 4%Préparation de la transition vers \gls{GTK} 4
	\item Fehlerbehebungen%Résolution de bogues
\end{itemize}

\subsection{Was ist mit der Zukunft?}%Et pour le futur ?}

Vollständige Portierung unter \gls{GTK} 4%Portage complet sous \gls{GTK} 4

%\newpage						% saut de page pour titre solidaire

\section{Kontakte\label{introduction-contacts}} 	% TODO à mettre à jour en profondeur / to be completely updated


%Outre les courriers aux auteurs, vous disposez de plusieurs listes de diffusion pour nous contacter ou obtenir des informations.
Neben den Briefen an die Autoren stehen Ihnen auch verschiedene Mailinglisten zur Verfügung, über die Sie uns kontaktieren oder Informationen erhalten können.

%Pour être tenu(e) au courant des évolutions de Grisbi, vous pouvez vous inscrire sur la \lang{liste d'information}\footnote{\urlListInfoEmail{}} prévue à cet effet. Vous recevrez juste un courriel à la sortie de chaque nouvelle version.
Um über die Entwicklung von Grisbi auf dem Laufenden gehalten zu werden, können Sie sich in die dafür vorgesehene \lang{Informationsliste}\footnote{\urlListInfoEmail{}} eintragen. Sie erhalten nur eine E-Mail, wenn eine neue Version herauskommt.

%Si vous souhaitez participer au développement de Grisbi, il existe une \lang{liste développement}\footnote{\urlListDevelEmail{}}.
Wenn Sie sich an der Entwicklung von Grisbi beteiligen möchten,gibt es eine \lang{Entwicklungsliste}\footnote{\urlListDevelEmail{}}.

%Par ailleurs, nous avons décidé d'entreprendre l'internationalisation de Grisbi et, si vous souhaitez nous aider, vous pouvez dans un premier temps nous contacter sur la liste de développement.
Außerdem haben wir beschlossen, die Internationalisierung von Grisbi in Angriff zu nehmen. Wenn Sie uns dabei helfen möchten, können Sie uns zunächst über die Entwicklungsliste kontaktieren.

%Pour vous abonner à l'une de ces listes, rendez vous simplement sur la page \urlListDiffGrisbi{} puis cliquez sur la/les liste(s) qui vous intéresse(nt).
Um eine dieser Listen zu abonnieren, gehen Sie einfach auf die Seite \urlListDiffGrisbi{} und klicken Sie auf die Liste(n), die Sie interessieren.

%% Ou bien allez sur le site de {Grisbi}\footnote{\urlGrisbi{}}} dans la rubrique \cmd{Contacts}. % <<Contacts>> non présent sur le site

%Vous pouvez également utiliser les forums de discussions (ou newsgroups) avec un logiciel appelé "lecteur de nouvelles" (comme Thunderbird par exemple) en renseignant \cmd{listes.grisbi.org} comme nom de serveur de groupes.
Sie können auch Diskussionsforen (oder Newsgroups) mit einer Software namens \glqq{}Newsreader\grqq{} (wie z. B. Thunderbird) nutzen, indem Sie \cmd{listes.grisbi.org} als Namen des Gruppenservers angeben.

%N'hésitez pas de toute façon à consulter régulièrement le site officiel de Grisbi.
Zögern Sie auf jeden Fall nicht, regelmäßig die offizielle Website von Grisbi zu besuchen.

%\newpage						% saut de page pour titre solidaire


\section{Autoren und Beiträge\label{introduction-authors}}		% TODO : section to be updated/completed


%Cédric \familyname{Auger} est à la base du projet de ce manuel.
Cédric \familyname{Auger} ist die treibende Kraft hinter dem Projekt dieses Handbuchs.

%Daniel \familyname{Cartron} a rédigé la documentation jusqu'à la version 0.4.0, fourni des conseils en comptabilité et en ergonomie, et créé le premier site de Grisbi. Sa passion des fichiers de comptes ultra compliqués a amené un plus indéniable à la découverte de bogues inédits.
Daniel \familyname{Cartron} verfasste die Dokumentation bis zur Version 0.4.0, gab Ratschläge zu Buchhaltung und Ergonomie und erstellte die erste Grisbi-Website. Seine Leidenschaft für ultrakomplizierte Kontodateien brachte einen unbestreitbaren Pluspunkt bei der Entdeckung bisher unbekannter Fehler mit sich.

%André \familyname{Pascual}, de \lang{Linuxgraphic}\footnote{\urlLinuxGraphic{}}, est l'auteur du premier logo (mascotte Grisbi sur le symbole de l'euro €).
André \familyname{Pascual} von \lang{Linuxgraphic}\footnote{\urlLinuxGraphic{}} ist der Autor des ersten Logos (Grisbi-Maskottchen auf dem Euro-Symbol €).

%Sébastien \familyname{Blondeel} a écrit les scripts permettant de produire les différents formats de la documentation et ceux relatifs à la conversion des images aux formats adéquats. Il est également l'artisan de l'adoption de \gls{LaTeX} pour la rédaction de la documentation. En outre, son expérience de l'édition électronique en fait un précieux conseiller, source de nombreuses suggestions.
Sébastien \familyname{Blondeel} hat die Skripte zum Erstellen der verschiedenen Formate der Dokumentation und der Bilderkonvertierung in die entsprechenden Formate geschrieben. Er ist außerdem der Architekt der Einführung von \gls{LaTeX} zum Schreiben von Dokumentationen. Darüber hinaus machen ihn seine Erfahrungen im elektronischen Publizieren zu einem wertvollen Berater und einer Quelle vieler Anregungen.

%Benjamin \familyname{Drieu}, développeur pour Grisbi et empaqueteur officiel pour \gls{Debian}.
Benjamin \familyname{Drieu}, Entwickler für Grisbi und offizieller Packer für \gls{Debian}.

%Alain \familyname{Portal}, qui commençait à s'ennuyer dans l'empaquetage \gls{RedHat}, s'est essayé au développement. Son amour du travail bien fait et son opiniâtreté ont fait de lui un correcteur de bogues. Il a participé également à l'élaboration de la documentation. Il s'est pris une envie de commencer à coder dans la version instable.
Alain \familyname{Portal}, dem die \gls{RedHat}-Verpackung zu langweilig wurde, versuchte sich als Entwickler. Seine Liebe zu gut erledigter Arbeit und seine Hartnäckigkeit machten ihn zu einem Fehlerbeseitiger. Er war auch an der Entwicklung der Dokumentation beteiligt. Er hatte Lust, mit dem Programmieren in der instabilen Version zu beginnen.

%Loic \familyname{Breilloux} a mis à jour la documentation pour la version 0.5.1 et a fait les mises à jour de la documentation pour les versions suivantes.
Loic \familyname{Breilloux} hat die Dokumentation für die Version 0.5.1 aktualisiert und hat die Dokumentation für die folgenden Versionen aktualisiert.

%Gérald \familyname{Niel} a remplacé Daniel \familyname{Cartron} dans le rôle de webmestre et a donc été le créateur de la nouvelle version du site de \lang{Grisbi}\footnote{\urlGrisbi{}}. Il a été également responsable des paquets \gls{Slackware}.
Gerald \familyname{Niel} ersetzte Daniel \familyname{Cartron} in der Rolle des Webmasters und war somit der Schöpfer der neuen Version der \lang{Grisbi}\footnote{\urlGrisbi{}}-Website. Er war auch für die Pakete von \gls{Slackware} verantwortlich.

%Juliette \familyname{Martin} a assuré la tâche ingrate de relecture de la documentation. S'il reste des erreurs, c'est certainement qu'elles étaient bien cachées pour avoir échappé à ses yeux attentifs.
Juliette \familyname{Martin} übernahm die undankbare Aufgabe, die Dokumentation zu überprüfen. Wenn es noch Fehler gibt, dann waren diese sicherlich gut versteckt, um ihren aufmerksamen Augen zu entgehen.

%François \familyname{Terrot}\footnote{\urlFrancoisTerrotEmail{}} a rejoint l'équipe afin d'assurer le \gls{portage} de Grisbi sous \gls{Windows}.
François \familyname{Terrot}\footnote{\urlFrancoisTerrotEmail{}} ist dem Team beigetreten, um die \gls{Portierung} von Grisbi unter \gls{Windows} sicherzustellen.

%Pierre \familyname{Biava}\footnote{\urlPierreBiavaEmail{}} a rejoint l'équipe de développement en 2008.
Pierre \familyname{Biava}\footnote{\urlPierreBiavaEmail{}} trat 2008 dem Entwicklungsteam bei.

%Didier \familyname{Chevalier}\footnote{\urlDidierChevalierEmail{}}, William \familyname{Ollivier}\footnote{\urlWilliamOllivierEmail{}} et Mickaël \familyname{Remars} ont eux aussi participé au développement.
Auch Didier \familyname{Chevalier}\footnote{\urlDidierChevalierEmail{}}, William \familyname{Ollivier}\footnote{\urlWilliamOllivierEmail{}} und Mickaël \familyname{Remars} waren an der Entwicklung beteiligt.

%Jean-Luc \familyname{Duflot}\footnote{\urlJeanLucDuflotEmail{}} a réalisé pour la version 0.6 une grosse mise à jour du manuel, qui en avait bien besoin depuis 2004, et a continué sur la 0.8 et la 1.0.
Jean-Luc \familyname{Duflot}\footnote{\urlJeanLucDuflotEmail{}} führte für die Version 0.6 eine große Aktualisierung des Handbuchs durch, die seit 2004 dringend nötig war, und setzte dies für die Versionen 0.8 und 1.0 fort.

%Alain \familyname{Letient}\footnote{\urlAlainLetientEmail{}} a effectué avec ténacité la relecture du manuel 0.6 et a réalisé son iconographie, et a aussi continué sur les versions 0.8 et 1.0.
Alain \familyname{Letient}\footnote{\urlAlainLetientEmail{}} hat hartnäckig das Lektorat des Handbuchs 0.6 durchgeführt und seine Ikonografie erstellt, und er hat auch an den Versionen 0.8 und 1.0 weitergearbeitet.

%Guy \familyname{Lebègue}, d'abord pour la version 0.8, puis avec Michèle \familyname{Bondil}\footnote{\urlMicheleBondilEmail{}} pour la 1.0, ont réalisé la partie Comptabilité d'association, qui nécessite bien des compétences de spécialiste en comptabilité.
Guy \familyname{Lebègue}, zuerst für die Version 0.8 und dann zusammen mit Michèle \familyname{Bondil}\footnote{\urlMicheleBondilEmail{}} für die Version 1.0, haben den Teil Vereinsbuchhaltung erstellt, der durchaus die Fähigkeiten eines Buchhaltungsexperten erfordert.

\section{Danksagungen\label{introduction-thanks}}


%Merci à \lang{TuxFamily}\footnote{\urlTuxFamily{}} qui a longtemps mis à notre disposition tous les outils dont nous avions besoin pour développer Grisbi (site Internet, ftp, CVS, listes de diffusions, etc.). Hélas, les attaques de pirates subies fin 2003 - début 2004 par \lang{TuxFamily} nous ont contraint à chercher un nouvel hébergement. Nous remercions donc aujourd'hui'hui \lang{SourceForge}\footnote{\urlSourceForge{}}, la plateforme sur laquelle nous avons migré, et souhaitons un prompt et rapide rétablissement à \lang{TuxFamily} qui fait cruellement défaut à des centaines de projets libres.
Vielen Dank an \lang{TuxFamily}\footnote{\urlTuxFamily{}}, die uns lange Zeit alle Werkzeuge zur Verfügung gestellt hat, die wir für die Entwicklung von Grisbi benötigten (Website, FTP, CVS, Mailinglisten, etc.). Leider haben uns die Hackerangriffe, denen \lang{TuxFamily} Ende 2003 und Anfang 2004 ausgesetzt war, dazu gezwungen, uns nach einem neuen Hosting umzusehen. Wir danken daher heute \lang{SourceForge}, der Plattform, auf die wir umgestiegen sind, und wünschen \lang{TuxFamily} eine schnelle und baldige Genesung, die Hunderte von freien Projekten schmerzlich vermissen.

%Un grand merci également à tous les contributeurs de la liste de développement qui participent à l'évolution de Grisbi par leurs suggestions, remarques et rapports de bogues, ainsi qu'aux nombreux relecteurs du \menu{Manuel de l'utilisateur}, qui contribuent à en faire un meilleur outil.
Ein herzliches Dankeschön auch an alle Mitwirkenden der Entwicklungsliste, die mit ihren Vorschlägen, Hinweisen und Fehlerberichten zur Weiterentwicklung von Grisbi beitragen, sowie an die vielen Korrekturleser des \menu{Grisbi Handbuch}, die dazu beitragen, Grisbi zu einem besseren Werkzeug zu machen.

\section{Lizenzen\label{introduction-licenses}}


%Le programme est soumis aux termes de la \gls{Licence Publique Generale GNU} (en anglais: \gls{GNU General Public License}). Les corrections de bogues et mises à jour ne sont pas garanties.
Das Programm unterliegt den Bedingungen der \gls{GNU General Public License}. Fehlerkorrekturen und Updates können nicht garantiert werden.

%Le manuel est soumis aux termes de la \gls{Licence de Documentation Libre GNU} (en anglais : \gls{GNU Free Documentation License}).
Das Handbuch unterliegt den Bedingungen der \gls{GNU-Lizenz für freie Dokumentation} (englisch: GNU Free Documentation License).

%Permission est accordée de copier, distribuer et/ou modifier ce document selon les termes de la Licence de Documentation Libre GNU version 1.1 ou toute version ultérieure publiée par la \gls{Free Software Foundation}.
Es wird die Erlaubnis erteilt, dieses Dokument unter den Bedingungen der GNU Free Documentation License Version 1.1 oder einer späteren Version, die von der \gls{Free Software Foundation} veröffentlicht wurde, zu kopieren, zu verbreiten und/oder zu verändern.

\section{Über dieses Handbuch\label{introduction-manual}}


%Vous avez sous les yeux la version 3.0 du manuel, en date de \actuality{}début 2025, qui correspond à la version 3.0 du logiciel Grisbi.
Vor Ihnen liegt die Version 3.0 des Handbuchs mit Stand von \actuality{}Anfang 2025, die der Version 3.0 der Grisbi-Software entspricht.

\vspacepdf{5mm}			% espace   : 5 mm

%% ONLY FOR FRENCH MANUEL %%
%\textbf{Note}: cette version 3.0 du manuel est une mise à jour de la version 2.0 sauf pour le chapitre \vref{association} \menu{Comptabilité d'association}, qui reste en version 2.0 %TODO to modify when it"ll be updated

\vspacepdf{5mm}			% espace   : 5 mm

%Ce manuel a été écrit avec le \gls{formateur de texte} \gls{LaTeX}\index{latex@\LaTeX}, et il est disponible sous le \gls{format de fichier} \gls{PDF} ou \gls{HTML}, avec illustrations (copies d'écran) dans ces deux formats. 
Dieses Handbuch wurde mit dem Textverarbeitungsprogramm \gls{LaTeX}index{latex@LaTeX} geschrieben und ist als \gls{Dateiformat} \gls{PDF}\index{PDF} oder \gls{HTML}\index{HTML}, mit Illustrationen (Screenshots) in beiden Formaten.

%Il est accessible directement dans le logiciel Grisbi par le menu \menu{Aide - Manuel} de la barre de menus, sous le format \gls{HTML} avec illustrations.
Es ist direkt in der Grisbi-Software über das Menü \menu{Hilfe - Handbuch} in der Menüleiste im Format \gls{HTML} mit Illustrationen zugänglich.

%Cependant, ces différents formats peuvent être téléchargés sur le site de \lang{Sourceforge}\footnote{\urlSourceForgeDocumentation{}}, ainsi que les versions correspondantes du logiciel sur \lang{Sourceforge}\footnote{\urlSourceForge{}} dans les dossiers << \textsf{grisbi stable} >> ou << \textsf{grisbi unstable} >>.
Diese verschiedenen Formate können jedoch von der \lang{Sourceforge}\footnote{\urlSourceForgeDocumentation{}}-Website heruntergeladen werden, ebenso wie die entsprechenden Versionen der Software auf \lang{Sourceforge}\footnote{\urlSourceForge{}} in den Ordnern \glqq{}\textsf{grisbi stable}\grqq{} oder \glqq{}\textsf{grisbi unstable}\grqq{}.

%Les outils nécessaires à la lecture de ces différents formats de documentation sont présentés dans la section \vref{introduction-manual-readers} \menu{Logiciels de lecture}.
Die Werkzeuge, die Sie zum Lesen dieser verschiedenen Dokumentationsformate benötigen, werden im Abschnitt \vref{introduction-manual-readers} \menu{Lesesoftware} vorgestellt.

\subsection{Präsentation\label{introduction-manual-presentation}}

%Bien que le logiciel Grisbi soit conçu pour être le plus intuitif possible et que la plupart des fonctions soient immédiatement compréhensibles, il est nécessaire de disposer d'un manuel de référence. Ce manuel a été conçu selon les principes suivants :
Obwohl die Grisbi-Software so intuitiv wie möglich gestaltet wurde und die meisten Funktionen sofort verständlich sind, ist es notwendig, ein Referenzhandbuch zu haben. Dieses Handbuch wurde nach den folgenden Grundsätzen erstellt:

\begin{itemize} 
	\item möglichst umfassend, also Beschreibung aller Funktionen;%le plus exhaustif possible, donc description de toutes les fonctionnalités;
	\item Kapitel, die nach einem möglichst standardisierten Raster organisiert sind:%chapitres organisés suivant une trame la plus standardisée possible:
		\begin{itemize}
		\item[\textopenbullet] Vorstellung des Kapitels,%présentation du chapitre,
		\item[\textopenbullet] Beschreibung der Darstellung,%description de l'affichage,
		\item[\textopenbullet] Beschreibung der Funktionen,%description des fonctions,
		\end{itemize}
	um die Orientierung im Dokument zu erleichtern;%pour faciliter le repérage dans le document;
	\item Abfassung von Absätzen, die von Kapitel zu Kapitel wiederkehren, auf möglichst identische Weise, um das schnelle Lesen zu erleichtern;%rédaction des paragraphes récurrents d'un chapitre à l'autre d'une manière la plus identique possible, pour faciliter la lecture rapide;
	\item Erleichterte Informationssuche durch zahlreiche \gls{Hyperlinks}, einen Index und ein Glossar.%recherche d'information facilitée grâce aux nombreux \gls{liens hypertexte}, à un index et à un glossaire.
\end{itemize}

% espace pour changement de thème : 5 mm
\vspacepdf{5mm}
Im Folgenden finden Sie eine kurze Beschreibung der einzelnen Kapitel:%Voici une description succincte des différents chapitres:

\begin{itemize}
	\item \menu{Präambel} erklärt die Herkunft des Namens, der dieser Software gegeben wurde;%{Préambule} explique l'origine du nom donné à ce logiciel;
	\item \menu{Einleitung} stellt die Software, das Handbuch, die Autoren und die Kontaktpersonen vor;%{Introduction} présente le logiciel, le manuel, leurs auteurs et les contacts;
	\item \menu{Erster Start von Grisbi} ist das \emph{grundlegende} Kapitel für den Einstieg in die Nutzung der Software;%{Premier démarrage de Grisbi} est le chapitre \emph{indispensable} pour commencer l'utilisation du logiciel;
	\item \menu{Startseite} beschreibt die wichtigsten Elemente der grafischen Benutzeroberfläche und ihre Bedienung mit Maus und Tastatur (Shortcuts);%décrit les éléments principaux de l'interface graphique et leur manipulation à la souris et au clavier (raccourcis);
	\item \menu{Kontenexport und -import} beschreibt, wie Sie Daten mit anderer Software austauschen können;%{Export et Import de comptes} décrit comment échanger des données avec les autres logiciels;
	\item \menu{Datenverwaltung} zeigt Optionen für Kontodateien, Backups und Archive sowie deren Verwaltung;%{Gestion des données} présente les options des fichiers de comptes, les sauvegardes et les archives ainsi que leur gestion;
	\item \menu{Kontoverwaltung} beschreibt die Eigenschaften von Konten, ihre Verwaltung und die verschiedenen Arten von Konten mit ihrer Verwendung;%{Gestion des comptes} décrit les propriétés des comptes, leur gestion et les différents types de comptes avec leur utilisation;
	\item \menu{Buchungen eines Kontos} beschreibt die Manipulation von Buchungen, die verwendeten Informations- und Eingabefelder und ihre Verwaltung, das Erstellen von Buchungen und ihre Verwaltung;%{Opérations d'un compte} décrit les manipulations sur les opérations, les champs d'information et de saisie utilisés et leur gestion, la création d'opérations et leur gestion;
	\item \menu{Buchungen abstimmen} beschreibt detailliert das Verfahren zur Abstimmung eines Kontos und die Verwaltung der Abstimmungen;%{Rapprochement bancaire} détaille la procédure de rapprochement d'un compte et la gestion des rapprochements;		
	\item \menu{Planer} beschreibt die Planung zukünftiger Buchungen und deren Handhabung;%{Échéancier} décrit la planification des opérations futures et leur manipulation;
	\item \menu{Suchen} gibt einen Überblick über die Möglichkeiten der Datensuche;%{Recherches} fait le point sur les possibilités de recherche de données;
	\item \menu{Empfänger}, \menu{Kategorien}, \menu{Gruppen} und \menu{Geschäftsjahre} beschreiben die Verwaltung dieser Daten;%{Tiers}, \menu{Catégories}, \menu{Imputations budgétaires} et \menu{Exercices} décrivent la gestion de ces données;
	\item \menu{Kreditrechner} beschreibt die Methoden und das Management von Simulationen;%{Simulation de crédit} décrit les méthodes et la gestion de simulations;
	\item \menu{Prognosen} beschreibt die Werkzeuge und Verfahren zur Erstellung von Budgets und Abschreibungstabellen sowie deren Verwaltung;%{Budgets prévisionnels} décrit les outils et les procédures de création de budget et de tableaux d'amortissement, ainsi que leur gestion;
	\item \menu{Bankkartenverwaltung und ihre Prognosen} beschreibt die Verwaltung von Bankkarten, insbesondere von Karten mit verzögerter Abbuchung, und die Methoden zur Erstellung von Prognosen;%{Gestion des cartes bancaires et leurs prévisions} décrit la gestion de ces cartes, notamment celles à débit différé, et les méthodes de prévision;	
	%%\item \menu{Comptabilité d'association} présente deux introductions pour les trésoriers d'association;ONLY FOR FRENCH
	\item \menu{Berichte} und \menu{Bericht erstellen} beschreiben die Verwaltung und das Erstellen von Berichten;%{États} et \menu{Création d'un état} décrivent la gestion et la création des états;
	\item \menu{Grisbi-Konfiguration} enthält detaillierte Angaben zu allen Einstellungsmöglichkeiten der Software;%{Configuration de Grisbi} détaille toutes les possibilités de réglage du logiciel;
	\item \menu{Wartungswerkzeuge} enthält einige Schlüssel, die Sie im Falle von Fehlern oder Bugs verwenden können.%{Outils de maintenance} donne quelques clés à utiliser en cas d'erreurs ou de bogues.
\end{itemize}


\subsection{Typografische Konventionen in diesem Handbuch\label{introduction-manual-conventions}}

%La liste suivante définit et illustre les conventions typographiques utilisées comme aides visuelles à l'identification d'éléments particuliers du texte du document:
Die folgende Liste definiert und illustriert typografische Konventionen, die als visuelle Hilfen zur Identifizierung bestimmter Elemente im Text des Dokuments verwendet werden:

\begin{itemize}
	\item Schnittstellenkomponenten sind Fenstertitel, Symbol- und Schaltflächennamen, Menünamen und andere Optionen, die auf dem Bildschirm erscheinen; sie werden so dargestellt:%les composants d'interface sont des titres de fenêtre, des noms d'icône et de bouton, des noms de menu et d'autres options qui apparaissent sur l'écran; ils sont présentés ainsi:
		\newline
		\hspace*{1.5cm}klicken Sie auf \menu{Enter}%cliquez sur \menu{Retour};
	\item Die Tastaturbeschriftung stellt dar, was auf den Tasten der Tastatur geschrieben wird; sie wird so dargestellt:%le libellé des touches du clavier représente ce qui est écrit sur les touches du clavier; il est présenté ainsi:
		\newline
		\hspace*{1.5cm}drücken Sie die Taste \key{Enter};%appuyez sur la touche \key{Entrée};	
	\item Tastenkombinationen sind eine Reihe von Tasten, die (sofern nicht anders angegeben) gleichzeitig gedrückt werden müssen, um eine einzige Funktion auszuführen:%les combinaisons de touches sont une série de touches à enfoncer 	simultanément (à moins d'indication contraire) pour réaliser une fonction unique; elles sont présentées ainsi:
		\newline
		\hspace*{1.5cm}drücken Sie die Tastenkombination \key{Strg/Ctrl}\key{R}%appuyez sur la combinaison de touches \key{Ctrl}\key{R};
	\item werden Befehle, die Teil einer Anweisung sind und eingegeben werden müssen, so dargestellt:%les commandes qui font partie d'une instruction et qui doivent être saisies sont présentées ainsi:
		\newline
		\hspace*{1.5cm}geben Sie \cmd{grisbi} ein, um das Programm zu starten%tapez \cmd{grisbi} pour démarrer le programme;
	\item werden die Datei- und Verzeichnisnamen so dargestellt:%les noms de fichier et de répertoire sont présentés ainsi:
		\newline
		\hspace*{1.5cm}\file{grisbi-n.n.n.rpm} und \file{/usr/local/bin};	
	\item Befehlszeilen bestehen aus einem Befehl und können einen oder mehrere mögliche Parameter des Befehls enthalten; sie werden so dargestellt:%les lignes de commande consistent en une commande et peuvent inclure un ou plusieurs paramètres possibles de la commande ; elles sont présentées ainsi:
		\newline
		\hspace*{1.5cm}	\cmd{rpm -Uvh grisbi-n.n.n.rpm}
	\item Jede alphanumerische Zeichenfolge in blauer Farbe in einem Dokument im Format \gls{PDF} oder \gls{HTML} ist ein Hyperlink, der entweder auf ein Bild, einen anderen Teil des Dokuments, ein Indexwort oder das Glossar (nur für \gls{PDF}) verweist;%toute suite de caractères alphanumériques en bleu, dans le document au format \gls{PDF} ou \gls{HTML}, est un lien hypertexte, renvoyant soit à une image, une autre partie du document, un mot indexé ou encore au glossaire (pour \gls{PDF} uniquement);
	\item Wörter oder Wortgruppen, auf die im Index verwiesen wird, werden in den Kapiteln so hervorgehoben:%les mots ou groupes de mots référencés dans l'index sont mis en valeur dans les chapitres ainsi:
		\newline
% TODO: to be updated
		\hspace*{1.5cm} \textopenbullet{}	\textsf{referierter Begriff} für das Format \gls{PDF}, %pour le format \gls{PDF};
		\newline 
		\hspace*{1.5cm} \textopenbullet{} in braun für das Format \gls{HTML};%en marron pour le format \gls{HTML}. 
% END_TODO
	\item \textbf{Notiz}: einen besonderen Punkt hervor, den es zu beachten gilt;
	\item \textcolor{red}{\strong{Achtung}}: entweder auf einen Punkt hinweist, der für das Verständnis sehr wichtig ist, oder auf einen Fehler, der nicht gemacht werden darf, da sonst Ihre Daten großen Schaden nehmen können; ein \textcolor{red}{\strong{Achtung}} ist \emph{unbedingt zu beachten}.
\end{itemize}

% espace avant Attention ou Note  : 5 mm
%\vspacepdf{5mm}
%De plus, une \textbf{Note} souligne un point particulier à prendre en compte, tandis qu'un \textcolor{red}{\strong{Attention}} indique soit un point très important pour la compréhension, soit une erreur à ne pas faire sous peine de dommage important pour vos données ; un \textcolor{red}{\strong{Attention}} est \emph{à respecter impérativement}.
%Außerdem hebt eine \textbf{Notiz} einen besonderen Punkt hervor, den es zu beachten gilt, während ein \textcolor{red}{\strong{Achtung}} entweder auf einen Punkt hinweist, der für das Verständnis sehr wichtig ist, oder auf einen Fehler, der nicht gemacht werden darf, da sonst Ihre Daten großen Schaden nehmen können; ein \textcolor{red}{\strong{Achtung}} ist \emph{unbedingt zu beachten}.

\subsection{Software zum Lesen\label{introduction-manual-readers}}

%Pour lire ce document, nous vous recommandons l'utilisation de logiciels libres, qui respectent tous votre vie privée et la confidentialité de vos données; les logiciels suivants disposent des fonctionnalités de \gls{liens hypertexte}:
Um dieses Dokument zu lesen, empfehlen wir Ihnen die Verwendung von Open-Source-Software, die Ihre Privatsphäre und den Schutz Ihrer Daten respektiert; die folgenden Programme verfügen über \gls{Hyperlinks}: 
\begin{itemize}
	\item für das Format \gls{PDF}\index{PDF}:%pour le format \gls{PDF}: 
		\begin{itemize}
			\item[\textopenbullet] Linux: Evince, Firefox, Xpdf, Ghostscript\textsuperscript{\textcolor{red}{\textbf{*}}}, MuPDF\textsuperscript{\textcolor{red}{\textbf{*}}}, Okular,
			\item[\textopenbullet] Mac: Okular, Skim, Xpdf,
			\item[\textopenbullet] Windows: Evince, Firefox, MuPDF\textsuperscript{\textcolor{red}{\textbf{*}}}, Okular, SumatraPdf;
		\end{itemize}
		\textsuperscript{\textcolor{red}{\textbf{*}}} Software, die es nicht erlaubt, das Inhaltsverzeichnis in einer Seitenleiste anzuzeigen.%Logiciel ne permettant pas d'afficher le sommaire dans un panneau latéral.
	\item für das Format \gls{HTML}\index{HTML}:%pour le format \gls{HTML}:
		\begin{itemize}
			\item[\textopenbullet] Linux: Firefox, Falcon, Links2, Midori, Dillo, SeaMonkey, NetSurf, Min,
			\item[\textopenbullet] Mac: Firefox, Falcon, Midori, SeaMonkey, NetSurf,
			\item[\textopenbullet] Windows: Firefox, Falcon, Midori, SeaMonkey, NetSurf.
		\end{itemize}
\end{itemize}

% espace pour changement de thème : 5 mm
\vspacepdf{5mm}
Sie haben die Wahl!%Bref, vous avez le choix!

% espace pour changement de thème : 5 mm
\vspacepdf{5mm}
%Ces logiciels sont tous téléchargeables sur leur propre site Internet et sont tous placés sous une licence de \gls{logiciel libre}, et vous pouvez, pour certains, en lire une présentation sur le site de \lang{Framasoft}\footnote{\urlFramasoftLogiciels{}}.
%These software packages can all be downloaded from their own websites and are all licensed as \gls{free software}, and you can read about some of them on the \lang{Free Software Directory}\footnote{\urlFreeSoftwareDirectory{}}or French \lang{Framasoft}\footnote{\urlFramasoftLogiciels{}} websites.
Diese Softwarepakete können alle von ihren eigenen Websites heruntergeladen werden und sind alle als kostenlose Software lizenziert, und Sie können über einige von ihnen auf den Websites \lang{Free Software Directory}\footnote{\urlFreeSoftwareDirectory{}} oder Französisch \lang{Framasoft}\footnote{\urlFramasoftLogiciels{}} lesen.
