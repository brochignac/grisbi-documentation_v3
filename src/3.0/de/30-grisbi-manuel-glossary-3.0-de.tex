%%%%%%%%%%%%%%%%%%%%%%%%%%%%%%%%%%%%%%%%%%%%%%%%%%%%%%%%%%%%%%%
% Contents: The glossary entries
% $Id: grisbi-manuel-glossary.tex, v 1.0 2014/02/12 Jean-Luc Duflot
% $Id: grisbi-manuel-glossary.tex, v 3.0 2024/04 Dominique Brochard : update
% $Id: grisbi-manuel-glossary.tex, v 3.0 2024/11 Dominique Brochard :
% - rename file to 30-xxx
% - add Windows and MacOS X entries
%%%%%%%%%%%%%%%%%%%%%%%%%%%%%%%%%%%%%%%%%%%%%%%%%%%%%%%%%%%%%%%% glossary entries file

%example
%%\newglossaryentry{exemple}{name=exemple, description={essai d'une entrée de glossaire}}

\newglossaryentry{Unicode control characters}{name=Unicode control characters, description={Viele Unicode-Zeichen werden verwendet, um die Interpretation oder Anzeige von Text zu steuern, aber diese Zeichen selbst haben keine visuelle oder räumliche Darstellung.  So wird beispielsweise das Null-Zeichen (U+0000 NULL) in C-Programmierumgebungen verwendet, um das Ende einer Zeichenkette anzuzeigen}}
\newglossaryentry{Verschlüsselung}{name=Verschlüsselung, description={see "zu verschlüsseln"}}
\newglossaryentry{zu verschlüsseln}{name=zu verschlüsseln, description={In der Kryptografie bezeichnet die Verschlüsselung eines Dokuments den Prozess der Verschlüsselung von Informationen. Dabei wird die ursprüngliche Darstellung der Informationen, der so genannte Klartext, in eine andere Form, den so genannten Chiffretext, umgewandelt. Im Idealfall können nur befugte Personen einen Chiffretext in einen Klartext zurückverwandeln und auf die ursprünglichen Informationen zugreifen}}
\newglossaryentry{Komprimierung}{name=Komprimierung, description={Unter Datenkompression versteht man die Kodierung von Informationen mit weniger Bits als die ursprüngliche Darstellung. Ein Gerät, das die Datenkompression durchführt, wird üblicherweise als Encoder bezeichnet, ein Gerät, das die Umkehrung des Prozesses (Dekompression) durchführt, als Decoder}}
\newglossaryentry{CSV}{name=CSV, description={Das Dateiformat CSV steht für englisch \lang{Comma-Separated Values} die Werte durch Kommas trennt}}
\newglossaryentry{CVS}{name=CVS, description={\lang{Concurrent Versions System} (CVS) ist ein Software-System zur Versionsverwaltung von Dateien, das hauptsächlich im Zusammenhang mit Software-Quelltext verwendet wird}}
\newglossaryentry{Debian}{name=Debian, description={Debian ist ein gemeinschaftlich entwickeltes freies Betriebssystem. Debian GNU/Linux basiert auf den grundlegenden Systemwerkzeugen des GNU-Projektes sowie dem Linux-Kernel. Debian wurde im August 1993 von Ian \familyname{Murdock} ins Leben gerufen und wird seitdem aktiv weiterentwickelt. Auf Debian basieren viele weitere Linux-Distributionen, von denen Ubuntu die bekannteste istest. Der Name des Betriebssystems leitet sich von den Vornamen des Debian-Gründers Ian \familyname{Murdock} und seiner damaligen Freundin und späteren Ehefrau Debra \familyname{Lynn} ab}}
\newglossaryentry{Linux-Distribution}{name=Linux-Distribution, description={Eine Linux-Distribution ist eine Auswahl aufeinander abgestimmter Software um den Linux-Kernel, bei dem es sich dabei in einigen Fällen auch um einen mehr oder minder angepassten und meist in enger Abstimmung mit Upstream selbst gepflegten Distributionskernel handelt. Distributionen, in denen \lang{GNU}-Programme eine essenzielle Rolle spielen, werden auch als \lang{GNU/Linux-Distributionen} bezeichnet}}
\newglossaryentry{Texteditor}{name=Texteditor, description={Ein Texteditor (von lateinisch textus ‚Inhalt‘ und editor für ‚Herausgeber‘ oder ‚Erzeuger‘) ist ein Computerprogramm zum Bearbeiten von Texten. Der Editor lädt die zu bearbeitende Textdatei und zeigt ihren Inhalt auf dem Bildschirm an, mit der Möglichkeit, durch positionieren eines Cursors einzelne Textzeichen hinzuzufügen oder zu löschen.Ein Texteditor (von lateinisch textus ‚Inhalt‘ und editor für ‚Herausgeber‘ oder ‚Erzeuger‘) ist ein Computerprogramm zum Bearbeiten von Texten. Der Editor lädt die zu bearbeitende Textdatei und zeigt ihren Inhalt auf dem Bildschirm an, mit der Möglichkeit, durch positionieren eines Cursors einzelne Textzeichen hinzuzufügen oder zu löschen. Im Gegensatz zu einem Textverarbeitungssystem und zu Desktop-Publishing-Software (DTP) bietet ein Texteditor in der Regel nur sehr eingeschränkte Layout- und Formatierungsfunktionen an und speichert den Text als reine Textdatei ohne Formatierungen}}
\newglossaryentry{Zeichenkodierung}{name=Zeichenkodierung, description={Eine Zeichenkodierung (englisch Character encoding, kurz Encoding) erlaubt die eindeutige Zuordnung von Schriftzeichen (i. A. Buchstaben oder Ziffern) und Symbolen innerhalb eines Zeichensatzes. In der elektronischen Datenverarbeitung werden Zeichen über einen Zahlenwert kodiert, um sie zu übertragen oder zu speichern}}
\newglossaryentry{Dateinamenserweiterung}{name=Dateinamenserweiterung, description={Die Dateinamenserweiterung (englisch filename extension), auch als Dateinamenerweiterung, Dateierweiterung, Dateiendung oder Dateisuffix bezeichnet, ist der letzte Teil eines Dateinamens und wird gewöhnlich mit einem Punkt abgetrennt. Die Dateiendung wird oft eingesetzt, um das Format einer Datei erkennbar zu machen. Zum Beispiel identifiziert name.txt eine einfache Textdatei}}
\newglossaryentry{Free Software Foundation}{name=Die \lang{Free Software Foundation} (\lang{FSF}, deutsch Stiftung für freie Software) ist eine nichtstaatliche Stiftung (NGO), die als gemeinnützige Organisation 1985 von Richard \familyname{Stallman} mit dem Zweck gegründet wurde, freie Software zu fördern und für diese Arbeit Kapital zusammenzutragen. Die Hauptaufgabe der FSF ist die finanzielle, personelle, technische und juristische Unterstützung des \lang{GNU}-Projekts}}
\newglossaryentry{Dateiformat}{name=Dateiformat, description={Als Dateiformat wird in der Informatik die vom Inhalt abhängige innere Struktur einer Datei bezeichnet. Der Inhalt einer Datei (Bilder, Filme, Grafiken, Musik, Texte, Videos, Zahlzeichen, Zeichnungen) entscheidet über das zu verwendende Dateiformat. In welchem Dateiformat eine Datei gespeichert wurde, lässt sich teilweise an ihrer Dateinamenserweiterung erkennen}}
\newglossaryentry{Schriftsatzsystem}{name=Schriftsatzsystem, description={ist die Bezeichnung für eine Software, die den Text so gestaltet, dass der Redakteur sich auf seinen Text konzentrieren kann, ohne dass die Anzeige des Ergebnisses seine kreative Tätigkeit stört. Das Ergebnis wird später durch Kompilierung des Quelldokuments im gewünschten Format, z. B. PDF oder PostScript, erzielt. Zum Beispiel ist \LaTeX{} ein Softwaresystem für den Schriftsatz von Dokumenten, im Gegensatz zum formatierten Text in WYSIWYG (What You See Is What You Get) Textverarbeitungsprogrammen wie Microsoft Word, LibreOffice Writer und Apple Pages}}
\newglossaryentry{Git}{name=Git, description={ist eine freie Software zur verteilten Versionsverwaltung von Dateien, die durch Linus \familyname{Torvalds} initiiert wurde}}
\newglossaryentry{Gnome}{name=Gnome, description={ursprünglich ein Akronym für \lang{GNU Network Object Model Environment},  ist eine Desktop-Umgebung für Unix- und Unix-ähnliche Systeme mit einer grafischen Benutzeroberfläche und einer Sammlung von Programmen für den täglichen Gebrauchest}}
\newglossaryentry{Gnucash}{name=Gnucash, description={ist eine freie Software zur Buchführung für Privatpersonen und kleine Unternehmen. Als Teil des \lang{GNU}-Projekts wurde es unter die \lang{GPL} gestellt. Eine grafische Benutzeroberfläche (\lang{GTK+}) erlaubt das Anlegen und das Verfolgen von verschiedenen Bankkonten und Wertpapierdepot}}
\newglossaryentry{GNU-Lizenz für freie Dokumentation}{name=\lang{GNU-Lizenz für freie Dokumentation}, description={(oft auch GNU Freie Dokumentationslizenz genannt; englische Originalbezeichnung \lang{GNU Free Documentation License}; Abkürzungen \lang{GNU FDL}, \lang{GFDL}) ist eine Copyleft-Lizenz, die für freiheitsgewährende Software-Dokumentationen gedacht ist, die aber auch für andere freie Inhalte verwendet wird. Die Lizenz wird von der \lang{Free Software Foundation} (\lang{FSF}), der Dachorganisation des \lang{GNU}-Projekts, herausgegebenest}}
\newglossaryentry{GNU General Public License}{name=\lang{GNU General Public License}, description={(kurz \lang{GNU GPL} oder \lang{GPL}; aus dem Englischen wörtlich für allgemeine Veröffentlichungserlaubnis oder -genehmigung) ist eine Softwarelizenz, die dem Nutzer gewährt, die Software auszuführen, zu studieren, zu ändern und zu verbreiten (kopieren). Software, die diese Freiheitsrechte gewährt, wird Freie Software genannt; und wenn die Software einem Copyleft unterliegt, so müssen diese Rechte bei Weitergabe (mit oder ohne Software-Änderung, -Erweiterung, oder Softwareteile-Wiederverwendung) beibehalten werden. Bei der \lang{GPL} ist beides der Fall. Die ursprüngliche Lizenz hat Richard \familyname{Stallman} von der \lang{Free Software Foundation} (\lang{FSF}) für das \lang{GNU}-Projekt geschrieben. Die \lang{FSF} empfiehlt die aktuelle, dritte Version (\lang{GNU GPL}v3), die im Jahr 2007 veröffentlicht wurde}}
\newglossaryentry{GNU/Linux}{name=GNU/Linux, description={Als Linux oder GNU/Linux bezeichnet man in der Regel freie, unixähnliche Mehrbenutzer-Betriebssysteme, die auf dem Linux-Kernel und wesentlich auf GNU-Software basieren. Die weite, auch kommerzielle Verbreitung wurde ab 1992 durch die Lizenzierung des Linux-Kernels unter der freien Lizenz \lang{GPL} ermöglicht. Einer der Initiatoren von Linux war der finnische Programmierer Linus \familyname{Torvalds}. Er nimmt bis heute eine koordinierende Rolle bei der Weiterentwicklung des Linux-Kernels ein und wird auch als Benevolent Dictator for Life (deutsch „wohlwollender Diktator auf Lebenszeit“) bezeichnet}}
\newglossaryentry{Grisbi}{name=Grisbi, description={Grisbi ist ein kostenloses persönliches Buchhaltungsprogramm}}
\newglossaryentry{GSB}{name=.gsb, description={ist die Erweiterung, die den Dateien der Konten von Grisbi gegeben wurde}}
\newglossaryentry{GTK}{name=GTK, description={GTK (früher GTK+, GIMP-Toolkit) ist ein freies GUI-Toolkit unter der \lang{LGPL}. GTK enthält viele Steuerelemente, mit denen sich grafische Benutzeroberflächen (GUI) für Software erstellen lassen}}
\newglossaryentry{GZ}{name=.gz, description={ist die übliche Erweiterung für Dateien, die mit der freien Komprimierungssoftware gzip (Akronym von GNU zip) komprimiert wurden. Komprimierte Archive werden in der Regel durch das Zusammenstellen von Dateisammlungen in einem einzigen tar-Archiv (auch \lang{tarball} genannt) und anschließendes Komprimieren dieses Archivs mit gzip erstellt. Die endgültige komprimierte Datei hat normalerweise die Erweiterung .tar.gz oder .tgz. est l'extension usuelle des fichiers compressés par le logiciel libre de compression gzip (acronyme de GNU zip). Les logiciels UNIX sont souvent distribués par des fichiers dont l'extension est .tar.gz ou .tgz, appelés \lang{tarball}. Ce sont des fichiers archivés avec le logiciel tar et compressés ensuite avec gzip. Cependant, depuis la fin des années 1990, de plus en plus de logiciels sont distribués par des fichiers d'extension .tar.bz2, archivés avec le logiciel tar et compressés ensuite avec le logiciel bzip2, parce que bzip2 permet de meilleurs taux de compression que gzip, au prix d'un temps de compression plus long}}
\newglossaryentry{HTML}{name=HTML, description={Die \lang{Hypertext Markup Language} (HTML, englisch für Hypertext-Auszeichnungssprache) ist eine textbasierte Auszeichnungssprache zur Strukturierung elektronischer Dokumente wie Texte mit Hyperlinks, Bildern und anderen Inhalten. HTML-Dokumente sind die Grundlage des World Wide Web und werden von Webbrowsern dargestellt}}
\newglossaryentry{IBAN}{name=IBAN, description={Die Internationale Bankkontonummer (englisch \lang{International Bank Account Number}, IBAN) ist eine internationale, standardisierte Notation für Kontonummern}}
\newglossaryentry{KDE}{name=KDE, description={Das Projekt wurde am 14. Oktober 1996 von Matthias \familyname{Ettrich} unter dem Namen Kool Desktop Environment ins Leben gerufen. KDE ist eine Community aus Programmierern, Künstlern und anderen Beitragenden, die sich der Entwicklung freier Software verschrieben hat. Sie ist primär bekannt für die \lang{Plasma}-Arbeitsumgebung (Versionen 6, 5 und 4) und das ursprüngliche \lang{K Desktop Environment} (Versionen 1–3; abgekürzt ebenfalls \lang{KDE}, wovon sich der Community-Name ableitet). Hinter der Community steht der eingetragene Verein \lang{KDE} e. V. mit Sitz in Berlin, der sich als juristische Person um finanzielle und rechtliche Aspekte des Projekts kümmert}}
\newglossaryentry{Programmiersprache C}{name=Programmiersprache C, description={C ist eine imperative und prozedurale Programmiersprache, die der Informatiker Dennis \familyname{Ritchie} in den frühen 1970er Jahren an den Bell Laboratories entwickelte. Die Anwendungsbereiche von C sind sehr verschieden. Sie wird zur System- und Anwendungsprogrammierung eingesetzt. Die grundlegenden Programme aller Unix-Systeme und die Systemkernel vieler Betriebssysteme sind in C programmiert. Zahlreiche Sprachen, wie C++, Objective-C, C\#, D, Java, JavaScript, LSL, PHP, Vala oder Perl, orientieren sich an der Syntax und anderen Eigenschaften von C}}
\newglossaryentry{LaTeX}{name=\LaTeX, description={ist ein plattformunabhängiges und freies Softwarepaket, das die Benutzung des Textsatzsystems TeX mit Hilfe von Makros vereinfacht. \LaTeX ist eine Auszeichnungssprache und ein Dateiformat, um insbesondere Texte, die mathematische Formeln enthalten, zu schreiben und für den Druck oder die Bildschirmansicht zu formatieren. LaTeX wurde Anfang der 1980er Jahre von Leslie Lamport entwickelt. Der Name bedeutet so viel wie Lamport TeX}}
\newglossaryentry{Hyperlink}{name=Hyperlink, description={(deutsch wörtlich „Über-Verknüpfung“, sinngemäß elektronischer Verweis) ist der Anglizismus für einen Link, der als Querverweis in einem Hypertext fungiert und einen Sprung zu einem anderen elektronischen Dokument oder an eine andere Stelle innerhalb eines Dokuments ermöglicht ist der Anglizismus für einen Link, der als Querverweis in einem Hypertext fungiert und einen Sprung zu einem anderen elektronischen Dokument oder an eine andere Stelle innerhalb eines Dokuments ermöglicht}}
%\newglossaryentry{Licence Publique Generale GNU}{name=Licence Publique Générale GNU, description={est le nom en français de la \lang{GNU General Public License} en anglais, communément abrégé en \lang{GPL}}}
%\newglossaryentry{Licence de Documentation Libre GNU}{name=Licence de Documentation Libre GNU, description={est le nom en français de la \lang{GNU Free Documentation License} en anglais, abrégé en \lang{GFDL}}}
\newglossaryentry{locale}{name=locale, description={Das Locale ist ein Einstellungssatz, der die Gebietsschemaparameter (Standortparameter) für Computerprogramme enthält. Dazu gehören in erster Linie die Sprache der Benutzeroberfläche, das Land und Einstellungen zu Zeichensatz, Tastaturlayout, Zahlen-, Währungs-, Datums- und Zeitformaten. Ein Einstellungssatz wird üblicherweise mit einem Code, der meist Sprache und Land umfasst, eindeutig identifiziert}}
\newglossaryentry{Freie Software}{name=Freie Software, description={Freie Software (freiheitsgewährende Software, englisch free software oder auch libre software) bezeichnet Software, die die Freiheit von Computernutzern in den Mittelpunkt stellt. Freie Software wird dadurch definiert, dass ein Nutzer mit dem Empfang}}
\newglossaryentry{Eingabemethode}{name=Eingabemethode, description={ist eine Methode zum Eingeben von Zeichen in einen Computer auch dann, wenn das Zeichen auf der Tastatur nicht vorhanden ist. Dies kommt vor, wenn es viel mehr Zeichen als Tasten gibt (wie etwa im Chinesischen oder Japanischen) oder wenn das Zeichen aus einer fremden Sprache ist, wie etwa kyrillische Buchstaben oder deutsche Umlaute bei englischer Tastaturbelegung}}
\newglossaryentry{OFX}{name=OFX, description={(\lang{Open Financial Exchange}) leitet sich von Microsofts Open Financial Connectivity und Intuits Open Exchange ab und ist ein Datenformat für den Finanzdaten-Austausch. Aktuelle Versionen von OFX basieren auf XML}}
\newglossaryentry{partition}{name=partition, description={Die meisten Betriebssysteme benötigen Partitionstabellen, können aber auch mit nur einer Partition betrieben werden. Eine Partitionstabelle und alle darin definierten Partitionen sind immer Teil eines Volumes, was physischem Datenspeicher wie z. B. einer Festplatte, einer NVMe-SSD oder einem USB-Stick entspricht. Die Partitionierung ermöglicht die Verwendung verschiedener Dateisysteme für unterschiedliche Arten von Dateien. Die bekanntesten Dateisysteme sind FAT32, NTFS, Ext2, Ext3 und Ext4}}
\newglossaryentry{PDF}{name=PDF, description={Das \lang{Portable Document Format} (englisch; kurz PDF; deutsch (trans)portables Dokumentenformat) ist ein plattformunabhängiges Dateiformat, das 1992 vom Unternehmen Adobe Inc. entwickelt. Neben Text, Bildern und Grafik kann eine PDF-Datei auch Hilfen enthalten, die die Navigation innerhalb des Schriftstückes erleichtern. Dazu gehören zum Beispiel anklickbare Inhaltsverzeichnisse und miniaturisierte Seitenvorschauen}}
%\newglossaryentry{plan comptable}{name=plan comptable, description={C'est l'ensemble des règles d'évaluation et de tenue des comptes qui constitue la norme de la comptabilité française. Le plan de comptes, c'est-à-dire la liste ordonnée des comptes, est un des éléments du plan comptable. C'est à tort que le langage usuel réduit souvent le plan comptable au seul plan de comptes}}
\newglossaryentry{PNG}{name=PNG, description={\lang{Portable Network Graphics} (PNG, englisch für „portable Netzwerkgrafik“, als Akronym auch silbisch ausgesprochen [pɪŋ]) ist ein Grafikformat für Rastergrafiken mit verlustfreier Datenkompression. PNG wurde als freier Ersatz für das ältere, bis zum Jahr 2006 mit Patentforderungen belastete Graphics Interchange Format (GIF) entworfen.  PNG unterstützt neben unterschiedlichen Farbtiefen auch Transparenz per Alphakanal}}
\newglossaryentry{Portierung}{name=Portierung, description={Unter Portierung oder Port (von englisch port) versteht man die Anpassung einer Software, meist durch Softwareentwickler, damit diese auf einer anderen Rechnerarchitektur lauffähig wird, beispielsweise einem anderen Betriebssystem, einer anderen Befehlssatzarchitektur oder einer anderen Plattform}}
\newglossaryentry{PostScript}{name=PostScript, description={ist eine Seitenbeschreibungssprache, die in den frühen 1980er Jahren von Adobe entwickelt wurde. Sie wird üblicherweise als Vektorgrafikformat für Dokumente und Drucker verwendet, stellt jedoch auch eine Turing-vollständige, stackorientierte Programmiersprache dar}}
\newglossaryentry{QIF}{name=QIF, description={(\lang{Quicken Interchange Format}) ist eine offene Spezifikation um Finanzdaten in Dateien abspeichern zu können}}
\newglossaryentry{RedHat}{name=Red Hat, description={Das Unternehmen Red Hat (engl. für: ‚roter Hut‘) ist ein US-amerikanischer Softwarehersteller mit Sitz in Raleigh, North Carolina, der unter anderem die weit verbreitete Linux-Distribution Red Hat Enterprise Linux (RHEL) vertreibt und am Fedora-Projekt beteiligt ist}}
\newglossaryentry{Slackware}{name=Slackware, description={[ˈslækwɛə] oder [ˈslækwɛɚ] ist die älteste aktive Linux-Distribution und die erste, die große weltweite Verbreitung fand. Wegen dieses frühen Erfolges und des konsequenten Verzichts auf unnötigen Ballast nach dem KISS-Prinzip bildete Slackware die Grundlage für andere bekannte Distributionen wie z. B. SuSE Linux}}
\newglossaryentry{SVG}{name=SVG, description={(\lang{Scalable Vector Graphics} (kurz auch SVG, englisch für skalierbare Vektorgrafik) ist die vom World Wide Web Consortium (W3C) empfohlene Spezifikation zur Beschreibung zweidimensionaler Vektorgrafiken. SVG, das auf XML basiert, wurde erstmals im September 2001 veröffentlicht. Praktisch alle relevanten Webbrowser können einen Großteil des Sprachumfangs darstellen.}}
\newglossaryentry{Sortierung}{name=Sortierung, description={In der Informatik versteht man unter Sortieren allgemein „den Prozess des Anordnens einer gegebenen Menge von Objekten in einer bestimmten Ordnung“ und nach einem definierten Kriterium, dem sogenannten Sortierschlüssel}}
\newglossaryentry{Primärsortierung}{name=Primärsortierung, description={ist der in Grisbi verwendete Begriff für eine Sortierung auf der Grundlage eines primären Sortierschlüssels, d. h. eine Sortierung auf erster Ebene. Wenn es nur die primäre Sortierung ohne sekundäre Sortierung gibt, dann handelt es sich um eine einfache Sortierung, die auf einem einzigen Sortierkriterium beruht}}
\newglossaryentry{Sekundäre Sortierung}{name=Sekundäre Sortierung, description={ist der in Grisbi verwendete Begriff für eine Sortierung auf der Grundlage eines sekundären Sortierschlüssels, d. h. eine Sortierung auf zweiter Ebene. Beispielsweise könnten Sie eine primäre Sortierung nach dem Wertstellungsdatum und eine sekundäre Sortierung nach dem Empfänger}}
\newglossaryentry{URL}{name=URL, description={Ein \lang{Uniform Resource Locator} (Abk. URL; englisch für „einheitlicher Ressourcenverorter“) identifiziert und lokalisiert eine Ressource, beispielsweise eine Webseite, über die zu verwendende Zugriffsmethode (zum Beispiel das verwendete Netzwerkprotokoll wie HTTP oder FTP) und den Ort (englisch location) der Ressource in Computernetzwerken.  Im allgemeinen Sprachgebrauch werden URLs auch als Internetadresse oder Webadresse bezeichnet}}
\newglossaryentry{UTF-8}{name=UTF-8, description={\lang{(Abkürzung für \lang{8-Bit UCS Transformation Format}, wobei UCS wiederum Universal Coded Character Set abkürzt) ist die am weitesten verbreitete Kodierung für Unicode-Zeichen}}
\newglossaryentry{XML}{name=XML, description={Die \lang{Extensible Markup Language} (dt. Erweiterbare Auszeichnungssprache), abgekürzt XML, ist eine Auszeichnungssprache zur Darstellung hierarchisch strukturierter Daten im Format einer Textdatei, die sowohl von Menschen als auch von Maschinen lesbar ist. Die Standardzeichenkodierung eines XML-Dokumentes ist UTF-8}}
\newglossaryentry{macOS}{name=macOS, description={früher Mac OS X und OS X, ist das Betriebssystem des kalifornischen Hard- und Software-Unternehmens Apple für Laptop- und Desktop-Computer der Mac-Reihe. Die aktuelle Version ist macOS Sequoia 15.1 vom 28. Oktober 2024}}
\newglossaryentry{Windows}{name=Windows, description={ist eine Reihe proprietärer grafischer Betriebssystemfamilien von Microsoft. Der Begriff Windows (englisch, Plural für „Fenster“) ist eine Bezeichnung für Software-Oberflächenelemente. Die erste Version von Windows im Jahr 1985 war lediglich eine grafische Benutzeroberfläche für MS-DOS, das auf IBM-Computern verwendet wurde. Es folgten die Versionen 2, 3, 95 (1995 veröffentlicht und auf fast allen PCs vorinstalliert verkauft, da die Computerhersteller aufgrund zahlreicher Exklusivverträge unter Androhung von Geldstrafen keine anderen Systeme installieren durften), XP, Vista, 7, 8 und 10. Version 11 ist die aktuelle Version im Jahr 2024}}

%% end of glossary entries
