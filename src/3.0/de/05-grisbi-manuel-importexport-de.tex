%%%%%%%%%%%%%%%%%%%%%%%%%%%%%%%%%%%%%%%%%%%%%%%%%%%%%%%%%%%%%%%%%
% Contents: The importexport chapter
% $Id: grisbi-manuel-QIF.tex, v 0.4 2002/10/27 Daniel Cartron
% $Id: grisbi-manuel-QIF.tex, v 0.5.0 2004/06/01 Loic Breilloux
% $Id: grisbi-manuel-QIF.tex, v 0.6.0 2011/11/17 Jean-Luc Duflot
% $Id: grisbi-manuel-QIF.tex, v 0.8.9 2012/04/27 Jean-Luc Duflot
% $Id: grisbi-manuel-QIF.tex, v 1.0 2014/02/12 Jean-Luc Duflot
% $Id: grisbi-manuel-importexport.tex, v 3.0 2025/09 Dominique Brochard
%%%%%%%%%%%%%%%%%%%%%%%%%%%%%%%%%%%%%%%%%%%%%%%%%%%%%%%%%%%%%%%%%


\chapter{Import und Export\label{importexport}}%Importing and exporting accounts\label{importexport}}
%\chapter{Import et export de comptes\label{importexport}}

Sie können Daten, die mit anderen persönlichen Buchhaltungsprogrammen erstellt wurden, nicht direkt in Grisbi verwenden und umgekehrt. Da diese Programme unterschiedlich funktionieren, sind ihre Daten unterschiedlich strukturiert, sodass Sie ihre Datenstruktur konvertieren müssen, bevor Sie sie verwenden können.
%You can not directly use data that has been created by other personal accounting applications in Grisbi, and vice versa. Because these applications work differently, their data is structured differently, so you need to convert their data structure before you can use it.
%Vous ne pouvez pas utiliser directement dans Grisbi des données qui ont été créées par d'autres applications de comptabilité personnelle, et réciproquement. Comme ces applications fonctionnent différemment, leurs données sont structurées différemment: il faut donc convertir leur structure de données avant de pouvoir les utiliser.

Diese Konvertierung kann nicht für alle Daten auf einmal durchgeführt werden, sondern muss für jedes von der Anwendung verwaltete Konto separat erfolgen. Um jedes dieser Konten zu konvertieren, müssen Sie sie zunächst aus der ursprünglichen Anwendung \dequote{exportieren} und dann in die Zielanwendung \dequote{importieren}.
%This conversion can not be done at once on all data, but must be done separately for each account managed by the application. To convert each of these accounts, you must first \enquote{export} them from the original application and then \enquote{import} them into the destination application.
%Cette conversion ne peut pas se faire d'un seul coup sur l'ensemble des données, mais doit se faire indépendamment pour chaque compte géré par l'application. Pour convertir chacun de ces comptes, il faut donc d'une part les \frquote{exporter} de l'application d'origine, puis les \frquote{importer} dans l'application de destination.

% espace avant Attention ou Note : 5 mm
\vspacepdf{5mm}

\Note{}: Verwechseln Sie die Grisbi-Datei mit der \gls{Dateinamenserweiterung} \file{\gls{GSB}}, die alle Daten aller für die Verwaltung einer Buchhaltungseinheit erstellten Konten enthält, nicht mit den \dequote{Kontodateien}, bei denen es sich um Dateien handelt, die jeweils nur Daten eines Kontos enthalten und nur zum Exportieren oder Importieren dieser Daten von einer Buchhaltungsanwendung in eine andere erstellt werden. Diese \dequote{Kontodateien} müssen ein \gls{Dateiformat} (oder eine \gls{Dateinamenserweiterung}) haben, das mit der ursprünglichen Anwendung UND der Zielanwendung kompatibel ist.
%\Note{}: do not confuse the Grisbi file, with the \gls{extension} \file{.gsb}, which contains all the data of all the accounts created for the management of an accounting entity, and the \enquote{account files}, which are files that contain only data from one account at a time, and created only to export or import that data from one accounting application to another. These \enquote{account files} must have a \gls{file format} (or \gls{extension}) that must be compatible with the original application AND the destination application.
%\Note{}: ne pas confondre le fichier Grisbi, d'\gls{extension} \file{.gsb} et qui contient tous les comptes avec leurs données, et les fichiers de ces mêmes comptes, qui sont des fichiers ne contenant que les données d'un seul compte à la fois, et créés uniquement pour importer ou exporter ces données d'une application de comptabilité à une autre. Ces fichiers de compte doivent avoir un \gls{format de fichier} (ou une \gls{extension}) obligatoirement compatible avec l'application d'origine ET l'application de destination.

% espace après Attention ou Note : 5 mm
\vspacepdf{5mm}

Grisbi unterstützt derzeit die persönlichen Buchhaltungsdatenformate \gls{Gnucash}, \gls{OFX}, \gls{CSV} und \gls{QIF}.
%Grisbi currently supports \gls{Gnucash}, \gls{OFX}, \gls{CSV} and \gls{QIF} personal accounting data formats.
%Grisbi supporte actuellement les formats de données de compte de comptabilité personnelle \gls{Gnucash}, \gls{OFX}, \gls{CSV} et \gls{QIF}.


\section{Importieren von Konten aus einer anderen Buchhaltungsanwendung\label{importexport-import}}
%\section{Importing accounts from another accounting application\label{importexport-import}}
%\section{Import de comptes d'un autre logiciel\label{importexport-import}}

Wenn Sie in Grisbi Kontodaten verwenden möchten, die in einer anderen Buchhaltungsanwendung erstellt wurden, müssen Sie zunächst jedes Konto dieser Anwendung einzeln in eine Reihe von Dateien exportieren und diese Dateien anschließend in Grisbi importieren.
%If you want to use account data that has been created in another accounting application in Grisbi, you must first export each of the accounts of this application individually to a set of files, then import these same files into Grisbi.
%Si vous voulez utiliser dans Grisbi des données de comptes qui ont été créés dans une autre application de comptabilité, vous devez d'abord exporter individuellement chacun des comptes de cette application dans un fichier, puis les importer dans Grisbi grâce à ces fichiers.

\subsection{Exportieren Sie eine Kontodatei aus der anderen Buchhaltungsanwendung\label{importexport-import-exportinit}}
%\subsection{Export an account file from the other accounting application\label{importexport-import-exportinit}}
%\subsection{Export d'un compte d'un autre logiciel\label{importexport-import-exportinit}}

Der erste Schritt besteht darin, in der ursprünglichen persönlichen Buchhaltungsanwendung jedes Konto in eine Datei im gewählten Format zu exportieren. Das gewählte Format muss mit den von der ursprünglichen Anwendung unterstützten Exportformaten kompatibel sein und sich für den Import in Grisbi eignen.
%The first step is, in the originating personal accounting application, to export each account in a file in the chosen format. The chosen format must be compatible with the export formats supported by the original application \emph{and} compatible with import to Grisbi.
%La première étape consiste, dans l'application de comptabilité personnelle d'origine, à exporter chaque compte dans un fichier au format choisi. Le format choisi doit être compatible à l'exportation par l'application d'origine \emph{et} compatible à l'importation par Grisbi.

Das Exportverfahren ist natürlich für jede Buchhaltungsanwendung unterschiedlich, daher sollten Sie die Dokumentation der jeweiligen Anwendung zu Rate ziehen. Wenn Sie alle Konten exportieren möchten, benötigen Sie so viele Dateien, wie Sie Konten in der Anwendung verwalten.
%The export procedure is obviously different for each accounting application, so refer to its documentation. If you want to export all accounts, you will need to get as many files as you have accounts managed by the application.
%La procédure d'exportation est bien évidemment différente pour chaque logiciel, donc référez-vous à sa documentation. Si vous voulez exporter tous les comptes, vous devrez obtenir autant de fichiers que vous avez de comptes gérés par l'application.


\subsection{Importieren von Kontodateien aus einer anderen Buchhaltungsanwendung in Grisbi\label{importexport-import-importinit}}
%\subsection{Importing account files from another accounting application to Grisbi\label{importexport-import-importinit}}
%\subsection{Import de fichiers de compte d'un autre logiciel dans Grisbi\label{importexport-import-importinit}}

\Note{}: Mit Grisbi können Sie eine oder mehrere Kontodateien in einem Arbeitsgang importieren. Sie können die Kontodateien zwar auch einzeln importieren, es ist jedoch wichtig, alle Kontodateien gleichzeitig zu importieren, damit Grisbi die Verknüpfungen zwischen den Konten wiederherstellen kann, insbesondere im Hinblick auf Überweisungsvorgänge.
%\Note{}: Grisbi allows you to import one or more account files in one operation. Although you can import the account files one by one, it is important to import all the account files at the same time, so that Grisbi can recreate the links between the accounts, especially with regard to the transfer operations.
%\Note{}: Grisbi permet d'importer un ou plusieurs fichiers de compte au cours de la même procédure. Bien que l'on puisse importer les fichiers de compte un par un, il est important de bien importer tous les fichiers de compte simultanément, afin que Grisbi puisse recréer les liens entre les comptes, particulièrement en ce qui concerne les opérations de virement.
% espace après Attention ou Note : 5 mm
\vspacepdf{5mm}

Weitere Informationen zu den \indexword{Kontoarten}\index{Kontoarten}, die Grisbi verwalten kann, finden Sie im Abschnitt \menus{Kontoarten}, \vref{accounts-type}.
%For more information on the \indexword{account types}\index{account types} that Grisbi can manage, see the \vref{accounts-type}, \menus{Grisbi account types} section.
%Pour plus de renseignements sur les \indexword{types de compte}\index{types de compte} que Grisbi sait gérer, voir la section \vref{accounts-type}, \menus{Types de comptes de Grisbi}.


Der Import kann im Menü \menus{Bearbeiten – Einstellungen} (\keys{Strg+Shift+P}), im Menü \menus{Allgemein – Import} und auf der Registerkarte \menus{Dateien importieren} konfiguriert werden (siehe Abschnitt \vref{setup-general-import-files}).
%The import can be configured in the \menus{Edit - Preferences} menu (\keys{Ctrl+Shift+P}), \menus{Generalities - Import} menu, \menus{Files import} tab (see section \vref{setup-general-import-files}).
%L'import est paramétrable dans le menu \menus{Édition - Préférences} (\keys{Ctrl+Maj+P}), menu \menus{Généralités - Importer}, onglet \menus{Importation des fichiers} (voir la section \vref{setup-general-import-files}).

Sie können festlegen, welches Datum für die Zuordnung eines Geschäftsjahres zu jedem importierten Vorgang verwendet wird, siehe \vref{setup-general-import-financialyear}, \menus{Geschäftsjahres Zuordnung}.
%You can define which date will be used for assigning a financial year to each imported operation, see \vref{setup-general-import-financialyear}, \menus{Definition of Financial Year}.
%Vous pouvez définir quelle date sera utilisée pour l'attribution d'un exercice à chaque opération importée, voir le paragraphe \vref{setup-general-import-financialyear}, \menus{Définition de l'exercice}.

Grisbi ermöglicht Ihnen auch, eine Zuordnung zwischen einer Zeichenfolge in dieser Datei und einem Empfänger herzustellen. Beispielsweise können alle Etiketten, die \dequote{Miete} enthalten, mit einem Empfänger verknüpft werden, der Ihren Vermieter darstellt. Dies muss im Menü \menus{Bearbeiten - Einstellungen} (\keys{Strg+Shift+P}), im Menü \menus{Allgemein - Import} und auf der Registerkarte \menus{Zuordnungen für Import} konfiguriert werden (siehe Abschnitt \vref{setup-general-importlinks}).
%Grisbi also allows you to establish an association between a string of characters in this file and a payee. For example, all labels containing \enquote{rent} may be associated with a payee that represents your landlord. This must be configured in the \menus{Edit - Preferences} menu (\keys{Ctrl+shift+P}), \menus{Généralities - Import} menu, \menus{Associations for import} tab (see the \vref{setup-general-importlinks} section).
%Grisbi vous permet aussi d'établir une association entre une chaîne de caractères de ce fichier et un tiers. Par exemple, tous les libellés contenant \frquote{loyer} peuvent être associés à un tiers qui représente votre propriétaire. Cela doit être configuré dans le menu \menus{Édition - Préférences} (\keys{Ctrl+Maj+P}), menu \menus{Généralités - Importer}, onglet \menus{Associations pour l'importation} (voir la section \vref{setup-general-importlinks}).

% espace pour changement de thème
\vspacepdf{5mm}

Wählen Sie im Menü Grisbi \menus{Datei} die Option \menus{Daten importieren} (oder verwenden Sie die Tastenkombination \keys{Strg+I}), wodurch sich das Fenster des Import-Assistenten öffnet. Der Import der Kontodateien erfolgt in fünf Schritten, zu denen für jedes weitere Konto ein Schritt hinzukommt:
%In the Grisbi \menus{File} menu, choose the option \menus{Import file} (or use the shortcut \keys{Ctrl+I}), which opens the import wizard window. The import of the account files takes place in five steps, to which one step must be added for each additional account:
%Dans le menu \menus{Fichier} de Grisbi, choisissez l'option \menus{Importer un fichier} (ou utilisez le raccourci-clavier \keys{Ctrl+I}), ce qui ouvre la fenêtre de l'assistant d'importation. L'importation d'un seul fichier de compte se déroule en cinq étapes, auxquelles il faudra rajouter une étape par compte supplémentaire:

\begin{enumerate}
	\item Startet den Importassistenten (Stufe 1/4): Bestätigen Sie mit der Schaltfläche \menus{Nachfolgendes Element};
	%Launches the import assistant (step 1/4): confirm with the \menus{Following} button;
	%Accueil de l'assistant d'importation (étape 1/4): validez avec le bouton \menus{Suivant};
	\item Auswahl der zu importierenden Kontodateien (Stufe 2/4) \refimage{importexport-import-files-select-img}:
	%Selection of the account files to import (step 2/4) \refimage{importexport-import-files-select-img}:
	%Sélection des fichiers de compte à importer (étape 2/4) \refimage{importexport-import-files-select-img}:
		\begin{enumerate}
			\item Klicken Sie auf die Schaltfläche \menus{Dateien für den Import hinzufügen}: Ein Dateibrowser-Fenster wird geöffnet;
			%click the button \menus{Add file to import...} button : a file manager window opens;
			%cliquez sur le bouton \menus{Ajouter un ou des fichiers...}: une fenêtre de gestionnaire de fichiers s'ouvre;
			\begin{figure}[htbp]
				\raggedleft
				%\begin{flushleft}
					\includegraphics[width=.95\textwidth]{image/screenshot/importexport_import_files_select}
				%\end{flushleft}
				\caption{Konten zum Importieren auswählen}%Selecting accounts to import%Sélection des comptes à importer}
				\label{importexport-import-files-select-img}
			\end{figure}
			\item Suchen Sie nach dem Verzeichnis, in dem sich diese Kontodateien befinden,
			%look for the directory where these account files are,
			%cherchez le répertoire où se trouvent ce ou ces fichiers de compte;
			\item Wählen Sie eine oder mehrere Kontodateien aus (Eine Mehrfachauswahl ist mit Strg + Klick oder Shift + Klick möglich); Sie können auch die \gls{Locale} (\gls{Zeichenkodierung}) der zu importierenden Dateien über das Dropdown-Menü \menus{Zeichensatz} ändern;
			% select one or more account files (with the combination \keys{Ctrl+Left-Click} and \keys{Shift+Left-Click}); you can also change the \gls{locale} (\gls{character encoding}) of the files to import from the \menus{Encoding} drop-down menu,
			%sélectionnez un ou plusieurs fichiers de compte (avec la combinaison \keys{Ctrl+Clic Gauche} et \keys{Maj \shift+Clic gauche}); vous pouvez aussi changer la \gls{locale} (\gls{encodage des caracteres}) des fichiers à importer dans le menu déroulant \menus{Codage};
			\item Bestätigen Sie das Fenster mit der Schaltfläche \menus{Öffnen}, um zum Fenster für die Auswahl der Kontodatei zurückzukehren;
			%validate the window with the \menus{Open} button to return to the account file selection window;
			%validez avec le bouton \menus{Ouvrir} pour revenir à la fenêtre de sélection des fichiers de compte;
			\item Sie können den Import von Kategorien deaktivieren, indem Sie die entsprechende Option aktivieren. Beim Importieren einer \gls{CSV}-Datei können Sie in einem neuen Fenster die Importeinstellungen auswählen \refimage{importexport-import-CSV-setup-img}:
			%you can choose not to import categories by ticking the appropriate option. When importing a \gls{CSV} file, a new window allows you to choose the import settings \refimage{importexport-import-CSV-setup-img}:
			%vous pouvez choisir de ne pas importer les catégories en cochant l'option adéquate. Dans le cas d'import d'un fichier \gls{CSV}, une nouvelle fenêtre vous permet de choisir les paramètres d'import \refimage{importexport-import-CSV-setup-img}:
				\begin{itemize}[label=\textopenbullet]
					\begin{figure}[htbp]
						\raggedleft
						\includegraphics[width=.95\textwidth]{image/screenshot/importexport_import_CSV_setup}
						\caption{Konfigurieren des Imports einer CSV-Datei}%Configuring the import of a CSV file}%Paramétrage de l'import d'un fichier CSV}
						\label{importexport-import-CSV-setup-img}
					\end{figure}
					\item \textbf{CSV Trennzeichen auswählen}: Das Trennzeichen zwischen den Daten kann aus der Dropdown-Liste im linken Fenster ausgewählt werden und wird im rechten Fenster angezeigt, wo Sie es auch ändern können.
					%\textbf{Choose CSV separator}: the separator between data can be selected from the drop-down list in the left-hand window and is displayed in the right-hand window, where you can also modify it.
					%\textbf{Choisissez le séparateur CSV}: le séparateur entre les données peut être sélectionné dans la liste déroulante de la fenêtre de gauche et s'affiche dans la fenêtre de droite, où vous pourrez aussi le modifier;
					\item \textbf{Datumsformat festlegen}: Das Datumsformat kann durch Anklicken des entsprechenden Kästchens und Auswahl aus der Dropdown-Liste festgelegt werden.
					%\textbf{Force date format}: the date format can be forced by ticking the appropriate box and selecting it from the drop-down list.
					%\textbf{Forcer le format de la date}: le format de date peut être forcé en cochant la case idoine et en le sélectionnant dans la liste déroulante;
					\item \textbf{CSV Felder auswählen}: Sie können die Datenzeilen markieren, die \textit{nicht} importiert werden sollen;
					%\textbf{Select CSV fields}: you can tick the data lines that \textit{will not be} imported;
					%\textbf{Sélectionnez les champs}: vous pourrez cocher les lignes de données qui \textit{ne seront pas} importées;
					\item Über die Schaltfläche \dequote{Eine Regel für diesen Import erstellen} (unten) können Sie eine Importregel erstellen, die Sie benennen müssen, um sie zu validieren. Sie finden es in einen Menüeintrag in der Buchungsliste (siehe \vref{transactions-functions}).
					%the \enquote{Create a rule for this import.} button (at the bottom) allows you to create an import rule that you will need to name in order to validate it. You will find it in the account toolbar (see \vref{transactions-functions}).
					%le bouton \frquote{Créer une règle pour cet import.} (en bas) permet de créer une règle d'import que vous devrez nommer pour la valider. Vous la retrouverez dans la barre d'outils du compte (voir \vref{transactions-functions}).
				\end{itemize}
			\item Nachdem Sie die gewünschten Dateien ausgewählt haben, können Sie Ihre Auswahl durch Klicken auf die Schaltfläche \menus{Nachfolgendes Element} bestätigen.
			%when the desired files are checked, you can validate the selection with the \menus{Following} button;
			%si les comptes choisis sont bien cochés, vous pouvez valider par le bouton \menus{Suivant};
		\end{enumerate}
	\item Importieren Sie die Kontodateien vollständig: Wenn alles gut gegangen ist, zeigt dieses Fenster die Liste der Kontodateien an, die importiert werden sollen; Setzen Sie den Import fort, indem Sie mit der Schaltfläche {Nachfolgendes Element} bestätigen;
	%Complete the import of the account files (step 3/5): if everything went well, this window gives the list of the account files which will be imported; continue the import by confirming with the \menus{Following} button;
	%Fin de la préparation de l'importation des fichiers de compte (étape 3/5): si tout s'est bien passé, cette fenêtre donne la liste des fichiers de compte qui seront importés; continuez l'importation en validant par le bouton \menus{Suivant};

	\item Erstellen und Konfigurieren jedes in Grisbi importierten Kontos (Stufe 4): Sie können jedes Konto überprüfen und folgende Aktionen auswählen \refimage{importexport-import-files-setup-img}:
	%Creating and configuring each account imported into Grisbi (step 4): you can review each account and choose the following actions \refimage{importexport-import-files-setup-img}:
	%Création et paramétrage de chaque compte importé dans Grisbi (étape 4): vous pouvez passer en revue chaque compte et y choisir les actions suivantes \refimage{importexport-import-files-setup-img}:
	\begin{figure}[htbp]
		\begin{center}
		\includegraphics[width=0.95\textwidth]{image/screenshot/importexport_import_files_setup}
		\end{center}
		\caption{Konfiguration jedes importierten Kontos}%Configuration of each imported account}%Paramétrage de chaque compte importé
		\label{importexport-import-files-setup-img}
	\end{figure}
		\begin{itemize}
			\item \menus{Ein neues Konto erstellen}: %Create a new account}:%Créer un nouveau compte
			Dadurch wird die ausgewählte Datei als neues Konto zu Ihrer Grisbi-Datei hinzugefügt. Im Dropdown-Menü \menus{Kontoart} unten können Sie den Kontotyp ändern;
			%this will add the selected file as a new account in your Grisbi file. The drop-down menu \menus{Account type} below will allow you to change the account type;
			% cela ajoutera le fichier sélectionné comme nouveau compte dans votre fichier Grisbi. Le menu déroulant \menus{Type de compte}, en dessous, vous permettra de modifier le type de compte;
			\item \menus{Buchungen einem Konto zuordnen}: %Add transactions to an account}:%Ajouter des opérations à un compte}:
			Wenn innerhalb des angegebenen Zeitintervalls geplante Vorgänge gefunden werden, öffnet sich ein spezielles Fenster, in dem Sie gefragt werden, wie Sie damit verfahren möchten: Entweder Sie führen diese geplanten Vorgänge mit den entsprechenden importierten Vorgängen zusammen oder Sie fügen die importierten Vorgänge zusätzlich hinzu (siehe Abschnitt \vref{setup-general-import-parameters}, \menus{Import Einstellungen}).
			%if scheduled operations are found within the specified time interval, a specific window opens to ask what you want to do with them: either merge these scheduled operations with the corresponding imported operations, or add the imported operations in addition to them (see the section \vref{setup-general-import-parameters}, \menus{Import settings}). The \menus{Account name} drop-down menu below will allow you to select the account to which the transactions will be added;
			% si des opérations planifiées sont trouvées dans l'intervalle de temps spécifié, une fenêtre spécifique s'ouvre pour savoir ce que vous voulez en faire: soit fusionner ces opérations planifiées avec les opérations importées correspondantes, soit ajouter les opérations importées en sus de celles-là (voir la section \vref{setup-general-import-parameters}, \menus{Paramètres pour l'import}). Le menu déroulant \menus{Nom du compte}, en dessous, vous permettra de sélectionner le compte auquel seront ajoutées les opérations;
			\item \menus{Buchungen einem Konto zuordnen und als verrechnet kennzeichnen}: %Mark transactions of an account}:%Marquer les opérations d'un compte
			Dadurch werden die Buchungen in der Spalte \dequote{V/A} (\vref{transactions-list-fields}) des betreffenden Kontos mit einem \dequote{A} gekennzeichnet. 
			%this will mark the transactions with a \enquote{T} in the \enquote{C/R} column (\vref{transactions-list-fields}) of the account concerned.
			%cela marquera les opérations avec un \frquote{T} dans la colonne \frquote{P/R} (\vref{transactions-list-fields}) du compte concerné. 
			Wenn \indexword{verwaiste Buchungen}\index{Buchung!verwaist} gefunden werden, öffnet sich am Ende des Imports ein Fenster, in dem Sie angeben können, was damit geschehen soll: entweder hinzufügen oder ignorieren. Über das Dropdown-Menü \menus{Kontoname} darunter können Sie das Konto auswählen, in dem die Transaktionen markiert werden sollen.
			%If any \indexword{orphan transactions}\index{transaction!orphan} are found, a window will open at the end of the import to ask what you want to do with them: either add them or ignore them. The \menus{Account name} drop-down menu below will allow you to select the account in which the transactions will be marked;
			%Si des \indexword{opérations orphelines}\index{opération!orpheline} sont trouvées, une fenêtre s'ouvrira en fin d'import pour savoir ce que vous voulez en faire: soit les ajouter, soit les ignorer. Le menu déroulant \menus{Nom du compte}, en dessous, vous permettra de sélectionner le compte dans lequel les opérations seront marquées;
			\item die Währung des Kontos festlegen (oder ein neues Konto erstellen);
			%set the currency of the account (or create a new one);
			%définir la devise du compte (ou bien en créer une nouvelle);
			\item \menus{Den Betrag der importierten Buchungen invertieren}: nützlich beispielsweise für Kreditkartenkonten;
			%\menus{Invert the amount of the imported transactions}: useful for credit card accounts for example;
			%\menus{Inverser le montant de l'opération importée}: utile pour les comptes de carte bancaire de la Banque Postale, par exemple;
			\item \menus{Eine Regel für diesen Import erstellen}:%\menus{Create a rule for this import}:%Créer une règle pour cet import
			ermöglicht es Ihnen, eine Schnellimportregel zu definieren, wenn die Datei im Format \gls{QIF}, \gls{Gnucash} oder \gls{OFX} vorliegt und nur, wenn Sie Buchungen zu einem Konto hinzufügen oder markieren. Diese Regel ist für jedes Konto spezifisch und muss benannt werden, um validiert zu werden. Sie finden sie in der Symbolleiste des Kontos; Sie finden sie in der Werkzeugleiste des Kontos (siehe \vref{transactions-functions});
			%allows you to define a quick import rule if the file is in \gls{QIF}, \gls{Gnucash}, or \gls{OFX} format and only if you add or mark transactions to an account. This rule is specific to each account and must be named in order to be validated. You can find it in the account toolbar (see \vref{transactions-functions});
			%\menus{Créer une règle pour cet import}: permet de définir une règle d'import rapide si le fichier est au format \gls{QIF}, \gls{Gnucash} ou \gls{OFX} et uniquement si vous ajoutez ou marquez des opérations à un compte. Cette règle est spécifique à chaque compte et devra être nommée pour être validée. Vous pourrez la retrouver dans la barre d'outils du compte (voir \vref{transactions-functions});
			\item Wenn alles korrekt ist, bestätigen Sie den Import mit der Schaltfläche \menu{Nachfolgendes Element};
			%when everything is correct, confirm the import with the \menu{Following} button;
			%quand tout est correct, validez l'importation par le bouton \menus{Suivant};
		\end{itemize}
	\item Bestätigen Sie das Ende des Imports: Bestätigen Sie mit der Schaltfläche \menus{Schließen}.
	%Confirm the end of the import: confirm with the \menus{Close} button.
	%Validation de la fin de l'importation: valider par le bouton \menus{Fermer}.
\end{enumerate}

Wenn Sie Ihre Grisbi-Datei unmittelbar vor dem Importieren der Kontodaten erstellt haben, kehren Sie zum Ende des Abschnitts \vref{start-newfile-end}, \menus{Erstellen einer neuen Kontodatei} zurück. Gehen Sie direkt zum Ende des Vorgangs zur Erstellung der Kontodatei, zum Absatz, der mit \emph{Auf die eine oder andere Weise\ldots{ }} beginnt und Sie auffordert, sofort weitere Konten zu erstellen.
%If, and only if you have created your account file just before this account data import, return to the end of the  \vref{start-newfile-end}, \menu{Creation of a new accounts file}. Go directly to the end of the account file creation process, at the paragraph beginning with \emph{In one way or another\ldots{ }}, which will prompt you to create other accounts right away.
%Si, et seulement si, vous venez de créer votre fichier Grisbi juste avant cette importation de données de comptes, revenez à la fin de la section \vref{start-newfile-end}, \menus{Création d'un nouveau fichier de comptes}. Allez juste après la fin de la procédure de création du fichier de comptes, au paragraphe commençant par \textbf{\emph{D'une manière ou d'une autre\ldots{ }}}, ce qui vous proposera de créer tout de suite d'autres comptes.

%espace pour changement de thème
\vspacepdf{5mm}
Andernfalls können Sie das soeben erstellte Konto verwenden.
%Otherwise, you can start using the account you just created.
%Sinon, vous pouvez commencer à utiliser le compte que vous venez de créer.

%TODO:following
\section{Export accounts from Grisbi\label{importexport-export}}%Export de comptes à partir de Grisbi


If you want to use account data created by Grisbi in another accounting application, you must first export this data to files and then import them into the other application using these files. The file format chosen must be compatible with the export by Grisbi \emph{and} compatible with the import by the destination application.
%Si vous voulez utiliser, dans une autre application de comptabilité, des données de compte qui ont été créées par Grisbi, vous devez d'abord exporter ces données dans des fichiers, puis les importer dans l'autre application grâce à ces fichiers. Le format de fichier choisi doit être compatible à l'exportation par Grisbi \emph{et} compatible à l'importation par l'application de destination.

In the \menus{File} menu select the \menus{Export accounts as QIF/CSV file} (ou use shortcut \keys{Ctrl+E}) that opens the Exporting Grisbi accounts wizard. Exporting accounts involves at least four steps:
%Dans le menu \menus{Fichier}, choisissez l'option \menus{Exporter vers un fichier QIF/CSV}  (ou utilisez le raccourci-clavier \keys{Ctrl+E}) qui ouvre l'assistant Export des comptes. L'exportation des comptes comporte à minima quatre étapes:

\begin{enumerate}
	\item Starting the assistant (step 1/3): this window indicates that, since the \gls{QIF} and \gls{CSV} file formats do not support currency, all transactions will be converted into the currency of their respective account; confirm with the \menus{Following} button;
	%Accueil de l'assistant (étape 1/3): cette fenêtre indique que, comme les formats de fichier \gls{QIF} et \gls{CSV} ne supportent pas les devises, toutes les opérations seront converties dans la devise de leur compte respectif; validez par le bouton \menus{Suivant};
	\item Selecting accounts and options \refimage{importexport-export-img}:
	%Sélection des comptes et des options
		\begin{figure}[htbp]
			\begin{center}
				\includegraphics[width=0.95\textwidth]{image/screenshot/importexport_export}
			\end{center}
			\caption{Exporting accounts}%Export des comptes
			\label{importexport-export-img}
		\end{figure}
		\begin{itemize}
			\item \textbf{Select accounts to export} (step 2/3): select the accounts to export by clicking in the corresponding box or on the \menus{Select all} button;
			% \textbf{Sélectionner les comptes à exporter} (étape 2/3): cliquez sur la ou les cases correspondantes à chaque compte à exporter ou sur le bouton \menus{Sélectionner tout};
			\item \textbf{Select options to export}:%Sélectionner les options pour l'export}:
				\begin{itemize}
					\item \menus{QIF format}: exports the selected account(s) in \gls{QIF} format; in addition, the option:
					%exporte le ou les comptes cochés au format \gls{QIF}; en plus l'option:
						\begin{itemize}
							\item \menus{Force US dates}:
							%saves the date in the format \enquote{month/day/year} (mm/dd/yyyy),
							%\menus{Force les dates au format US}: enregistre la date au format \frquote{mois/jour/année} (mm/dd/yyyy),
							\item \menus{Force US numbers}: 
							%uses the period \enquote{.} as the decimal separator and the comma \enquote{,} as the thousands separator;
						%\menus{Force les nombres au format US}: utilise le point \frquote{.} comme séparateur de décimale et la virgule \frquote{,} comme séparateur des milliers;
						\end{itemize}
					\item \menus{CSV format}: exports the selected account(s) in \gls{CSV} format; in addition to the options available in QIF format (above):
					% exporte le ou les comptes cochés au format \gls{CSV}; en plus des options disponibles au format QIF (ci-dessus):
						\begin{itemize}
							\item the separator between data can be selected from the drop-down list in the left window and is displayed in the right window, where you can also modify it.
						%le séparateur entre les données peut être sélectionné dans la liste déroulante de la fenêtre de gauche et s'affiche dans la fenêtre de droite, où vous pourrez aussi le modifier;
							%\item \menus{Quote the dates}: if checked (default), dates will be enclosed in quotation marks, like other data. Hovering over the menu with the mouse opens an information window \enquote{Uncheck to write a date field without quotes};
						% \menus{Citer les dates}: si cochée (par défaut), les dates seront mises entre guillemets, comme les autres données. Le survol du menu avec la souris ouvre une fenêtre d'information \frquote{Décocher pour ne pas mettre les dates entre guillemets};
						\end{itemize}
				\end{itemize}
			\item \menus{Name files automatically}: This option automatically exports the selected account(s):
					%\item \menus{Nommer les fichiers automatiquement}: cette option exporte le ou les comptes sélectionnés automatiquement:
					\begin{itemize}
					\item with a file name generated as follows:\newline
					%avec un nom de fichier généré comme suit:\newline
					%[name of accounting entity]-[account name].[selected extension]. Hovering over the menu with the mouse opens an information window \enquote{Generated filenames are [accounting entity]-[account name].[extension]},
					%[nom de l'entité comptable]-[nom du compte].[extension sélectionnée]. Le survol du menu avec la souris ouvre une fenêtre d'information \frquote{Les noms générés sont [nom de l'entité comptable]-[nom du compte].[extension]},
					\item and saves it to the folder where the Grisbi file was opened. If the exported file(s) already exist(e), Grisbi will ask you to confirm whether you want to overwrite them;
					%et dont la destination est le dossier où a été ouvert le fichier Grisbi; si le ou les fichiers exportés existent déjà, Grisbi vous demandera confirmation pour le ou les écraser;
					\end{itemize}
					\Note:
					%it is recommended not to use the character \enquote{\slash{}} in the account name, otherwise an error may occur when exporting to \gls{Linux distributions}.
					%\Note: il est recommandé de ne pas utiliser le caractère \frquote{\slash{}} dans le nom du compte sous peine d'avoir une erreur lors de l'export sur des \gls{distributions Linux}.
					\item confirm with the \menus{Following} button;
					%validez par le bouton \menus{Suivant};
				\end{itemize}
			\item For each account, define the name of the file, the destination directory and the export format; the option \menus{Name files automatically} bypasses this step. Then confirm with the \menus{Following} button;
		%pour chaque compte, définissez le nom du fichier, le répertoire de destination et le format d'exportation; l'option \menus{Nommer les fichiers automatiquement} shunte cette étape. Puis validez par le bouton \menus{Suivant}
	\item the export completion window appears; confirm with the \menus{Close} button;
%la fenêtre de fin de l'exportation s'affiche; validez par le bouton \menus{Fermer}.
\end{enumerate}

\Attention{}: in general, it is inadvisable to have accents or spaces in the names of directories and files used by Grisbi. If so, rename them now. For example, spaces can be replaced by underscores (\_).
%d'une manière générale, il est déconseillé d'avoir des accents ou des espaces dans les noms des répertoires et fichiers utilisés par Grisbi. Si c'est le cas, renommez-les maintenant. Par exemple, les espaces peuvent être remplacées par des tirets bas (\_).