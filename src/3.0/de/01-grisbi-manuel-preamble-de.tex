%%%%%%%%%%%%%%%%%%%%%%%%%%%%%%%%%%%%%%%%%%%%%%%%%% %%%%%%%%%%%%%%%
% Contents: The preamble chapter
% $ Id: gray-manual-preamble.tex, v 0.4 2002/10/27 Daniel Cartron
% $ Id: gray-manual-preamble.tex, v 0.6.0 2011/11/17 Jean-Luc Duflot (no change)
% $ Id: gray-manual-preamble.tex, v 0.8.9 2012/04/27 Jean-Luc Duflot (typo changes)
% $ Id: gray-manual-preamble.tex, v 1.0 2014/02/12 Jean-Luc Duflot
%%%%%%%%%%%%%%%%%%%%%%%%%%%%%%%%%%%%%%%%%%%%%%%%%% %%%%%%%%%%%%%%%
% deleted because it does not work in pdf and creates a problem in html
% \Begin{small}
% star command: deleted star for index entries with correct link display in html
\chapter{Preamble\label{preamble}}

%% commented for index entries with correct link display in html
%% see also the preamble chapter moved in \mainmatter in manuel.tex
% \markboth{}{Preamble}
% \ifpdf
% \addcontentsline{toc}{chapter}{Preamble}
% \else
% \fi

\section{Forward to the English Translation\label{preamble-foreword}}

%This English translation is a work in progress and constitutes a revision of version 1.0 of this manual.
%This manual is currently being translated into English version 3.0, which corresponds to version 3.0 of the software. It is a combination of the revision of English version 1.0 and the translation of the French version 3.0.
Die Handbuchversion 3.0 entspricht der Softwareversion 3.0.\\
Die deutsche Version ist eine Kombination aus der deutschen Version 2.0, die von Martin Stromberger im Jahr 2020 erstellt wurde, und der Übersetzung der französischen/englischen Version 3.0.

It was started by Bob \familyname{Anderson}\footnote{\urlBobAndersonEmail{}} with the version 1.0 in order to give a little back to the \gls{Free and Open Source Software} (FOSS) Community.  He had no particular financial or linguistic skills, but as a long-standing user of the software and managing without the help of the manual, he knew how necessary this translation was.

%This translation is a strict paragraph by paragraph  translation, it is important to keep to this translation format in order to simplify the addition to the English translation of any changes made to the original French Manual.  In some places you will also come across a short paragraph or footnote introduced by the phrase \\
%\strong{Translators Note:} \\
%where an extra paragraph or footnote has had to be added to explain a problem with the translation, something specific to the English language interface or flag a section which needs revision in both the French and English manual versions.  However in order to ensure the strict one to one correspondence between the French and English translations; it is essential that the French Manual remains the authoritative document and thus revisions to the English Manual will only be made \strong{AFTER} the French revision is accepted.

This translation is word for word. It is important to keep to this format to make it easier to add changes to the English translation of the original French manual.  You may also see a short paragraph or footnote with the note \strong{Translators Note:}. If a new paragraph or footnote is added to explain a problem with the translation, or to flag a section that needs revision in both the French and English versions, it must be done in a way that ensures the French and English versions are exactly the same. This means that any changes to the English version will only be made \strong{after} the French version is accepted.

%\subsection*{How to help make an English manual}

%There are many ways you can help to complete this English translation.  As yet only a handful of chapters have been translated.  The figures are currently showing the French interface version and all the example account entries in the illustrations are also in French.  Those chapters that have been translated could probably benefit from better phrasing in places.

%For detailed guidance on how you can help please see the document HelpTranslate-en.md on the manual\lang{Github}\footnote{\urlGitDoc{}} site.

%\subsection*{Acknowledgements and a dedication}

%Most of my translations were done with the aid of \lang{Google Translate}\footnote{\urlGtrans{}} using \lang{Translate Shell}\footnote{\urlGtransShell{}}

%The English translation of this manual is dedicated to the memory of my father John Anderson FCA\footnote{FCA = Fellow of the Institute of Chartered Accountants.  John Anderson (AJ)  studied accountancy while serving in the Royal Air-force as a Radar Technician during the second World War.  On being demobbed at the end of the war he then joined the accountancy firm started by my Grandfather: H J Anderson and Co.  Any accountancy skills I may have picked up are thanks to him.}

\subsection*{The translation from the French Manual starts on the next page}

%The translation from the French Manual starts with a the next section heading \strong{Why call it Grisbi?}. I have changed the section heading from the original \strong{Préambule}. This section is a commentary by the original author on the etymology of the word Grisbi.
%The translation from the French manual begins with the title of the subsequent section \strong{Why refer to it as Grisbi?}, modifying the original title \strong{Préambule}. This section comprises the Daniel \familyname{Cartron} original author's commentary on the etymology of the word \lang{grisbi}.

The translation from the French manual begins with the title of the next section, \strong{Why refer to it as Grisbi?}, which has been modified by Bob \familyname{Anderson} from the original title, \strong{Etymology}. This section includes the original author Daniel \familyname{Cartron}'s commentary on the etymology of the word \lang{grisbi}.

%I should give a word of warning about my translation efforts on this first section about the origin of the word Grisbi.  I have tried to retain the sense of the various French dictionary entries,  although most of the words are translated to English the accuracy of the translation should not be relied upon for more than giving the reader an overall feel of the original French text.  If you are very interested in the etymology then you had better read the original French version of this section of the manual.  

Bob \familyname{Anderson} provides a cautionary note regarding his translation efforts. He has endeavoured to maintain the denotation and connotation of the French dictionary entries. While the majority of the words have been translated into English, the accuracy of these translations should not be relied upon, except to provide a general impression of the original French text. For those with a keen interest in etymology, it would be beneficial to consult the original French version of this manual section.

\newpage 
\section{Why refer to it as Grisbi?\label{preamble-etymology}}

Several \gls{Grisbi} users have asked Daniel \familyname{Cartron} to insert in the manual a brief reminder of the meaning of this word, which, to his chagrin, fell back into disuse.

His first (brief) research did not bring him any results worth publishing, He had then dropped it until one day he had the opportunity to spend some time in a well-stocked library containing dictionaries of all kinds, there the harvest was abundant. It was so abundant that Daniel \familyname{Cartron} hesitated a long time to know what he was going to keep and what he was going to eliminate \dots{}
Finally he decided to keep everything, even if we go from a paragraph to four pages. He just made a compilation of the shortest items to the longest ones.
Indeed he find it interesting to study the differences between the different dictionaries, but even more to note the similarities, at this point striking that one could title this chapter not \emph{The game of the seven errors} but \emph{Who copied who?} It's up to you to find\dots{}
And he also added a passage on the film because for those who still know what \indexword{grisbi}\index{Grisbi} means is essentially due to the deserved reputation of this work.

Here are some sources on the etymology \index{etymology} of the word grisbi:

\subsection*{Dictionary of French language --- Hachette}

[grizbi] n. m. Slang. Silver --- Of grey (grey money, cd rouchi griset [1834], ``liard''), and suff. pop. -bi; 1895, spread in 1953.

\subsection*{Grand Larousse of the French language}

[grizbi] n. m. (of \emph{grey [and]}, piece of six liards [1834, Esnault] --- der.
from \emph{grey}, because of the color [cf. also \emph{grisette}, ``coin'' --- 17th century. ---, and \emph{white and grey coin}, 1784, Esnault] --- with the suffix.slang. \emph{-bi}; 1896, Delesalle).
\emph{Slang.} Money: \emph{Do not touch the grisbi} (title of a novel by Albert Simonin [1953]).

\subsection*{Historical Dictionary of the French language --- Robert}

n.m. appeared in 1895 (\emph{grisbis}) and spread from 1953 by the novel \emph{Do not touch the grisbi} by A. Simonin, would be composed of \emph{gris} ``grey money'' (1784: see le rouchi \emph{griset} ``six-liard coin'', 1834, and \emph{grisette} ``coin'', v. 1634 ) and the \emph{bi} element of obscure origin: \emph{grisbi}, ``silver'' in slang, could be a tautological compound of \emph{grey} and \emph{bis}.

\subsection*{Dictionary of French slang and its origins --- Larousse}

Very controversial origin: either of griset, ``coin'', and a mysterious suffix -bi, or bread both grey and bis, or English slang crispy, silver; we propose to see a metonymic use of gripis 1628 [Cheneau], grispin, grisbis 1849 [Halbert], ``meunier'', that is to say, ``one who has at his home wheat'' 1895 [Delsalle] but reintroduced by ``Touch the grisbi'', famous novel of A. Simonin, published in 1953.

VARIANTS --- grijbi: 1902 [Esnault] --- grèzbi: around 1926 [id.]

DERIVATIVES --- grisbinette n.f. One hundred old franc coins: 1957 [Sandry-Carrère].

\subsection*{Treasures of the French language}

\emph{Slang.} Silver. Synon. pop. \emph{money, cake, pèze, cash. The grisbi I'm big enough to pick it myself! (\dots{}) Riton who had not even known how to behave like a man (\dots{}) as soon as he felt enough grisbi} (\familyname{Simonin}, \emph{Touch not to the grisbi}, 1953, p 231).
\strong{Pronunciation}: [grizbi]. \strong{Étymol. and Hist.} 1896 \emph{grisbis}
slang. ``money'' (\familyname{Delesalle}, \emph{French-Slang and Slang-French Dict.}). Word composed of rad. of \emph{griset}, in the sense of this six-liard piece'' (1834 ds \familyname{Esn.}), der. of \emph{grey}, because of the color (\emph{cf.} also \emph{ca} 1634 \emph{grisette} ``coin'', \emph{The Norman Muse} by D. Ferrand , Ed A. Heron, II, 91, 1784, Brest, \emph{white and grey coin} in \familyname{Esn.}, and a second part of obscure origin which represents maybe the suff. pop. -bi, to be close to \emph{nerbi} ``very black'' (from \familyname{Esn.}). It is not impossible that \emph{grisbi} (formerly \emph{grisbis}) is a tautological compound of \emph{grey} and \emph{bis}.

\strong{Bbg.} \familyname{Rigaud} (A.). L'arg. litt. \emph{Life lang.} 1972, pp. 114-117.

\subsection*{Grand Robert of the French language}

[grizbi] n. m. --- 1895: spread 1953 by Simonin's novel \emph{Touch not to grisbi}; the word was rare or archaic v. 1950: from \emph{grey} ``grey money'' (see rouchi \emph{griset} ``liard'', 1834), and suff. pop. \strong{Slang} Money. \emph{T'as du grisbi?}

1 --- This expression: ``Do not touch the grisbi'' becomes a variation of ``Do not mess with the nippes''. This is the keyword that leads the chronicle of these knights of ill-gotten fortune who gave mobility to Peter \familyname{Cheyney}'s cape and machine gun novels.
\familyname{P. Mac Orlan}, \emph{in} Albert \familyname{Simonin}, Do not touch the grisbi, Preface, p. 6.

%2 --- ``Don't sweat the small talk \dots{} First we have no time if you want me to find you Ali. It all depends on what he has of grisbi digging; if he is armed, one has a chance to find him in the flames, in the Carillon part.''
2 --- ``Forget bein' nice, fam\dots{} We ain't got time if you need me to track down Ali. It all boils down to how much grisbi (crispy) he's stacking in his pockets. If he’s packing, there’s a shot at spotting him at the casino, at the Carillon joint''

Albert \familyname{Simonin}, do not touch the grisbi, p. 147.

\subsection*{Dictionary of Unconventional French --- Hachette}

n.m. (Grisby)

Silver (intrinsically).

%??\emph{To the little boss of the team, a muffle with a torpedo cap, blue heaters and pumps varnished, the kid had just said she was frying only for the right motive, to relieve me of my hundred bags. He had replicated, the naughty jealousy: --- The \emph{grisbi}, I'm big enough to pick it up myself! They both said they were both, and they were equally ready for anything emph{grisbi} (loot). They and their little friends. Like Angelo-la-Tante and Josy-la-Peau-de-Vache; like Ali-le-Fumier and his garbage of espingos; like Riton, who had not even known how to be a man with his child, as soon as he had felt enough of \emph{grisbi}; like Marco and his little Wanda, if honest, but not hesitating to be stepped over by the man \emph{grisbi}! like the same Lulu, no doubt, waiting patiently for me to turn down with my \emph{grisbi}!}??

``To the little kingpin of the team, a kid in a torpedo cap, blue overalls and patent leather pumps, the little girl had just said she was watching only for the right motive, to relieve me of my hundred notes. He had replicated, the naughty jealousy: ``The \emph{grisbi}, I'm big enough to pick it up myself!'' They were both right, both equally ready to do anything for the \emph{grisbi} (loot). They and their little friends. Like Angelo-la-Tante and Josy-la-Peau-de-Vache; like Ali-le-Fumier and his garbage of espingos; like Riton, who hadn't even known how to behave like a man with his brat, as soon as he felt he had enough \emph{grisbi}; like Marco and his little Wanda, so honest, but who didn't hesitate to be stepped over by the \emph{grisbi}! like also the young Lulu, no doubt, who waited patiently at home for me to come back, with my \emph{grisbi}''

A. \familyname{Simonin}, Do not touch the grisbi, p. 233

HIST. --- 1895, but probably little used: A. \familyname{Bruant} and L. \familyname{Blédort}, who occasionally accumulate synonyms (\emph{weigh, bone,}, etc.), do not use \emph{grisbi}, although \familyname{Bruant} records it in 1901 (\emph{grisbis}).
% The deserved success of A. Simonin's novel in 1953 ??made first use of?? the word, which does not seem really integrated in the series of alternative words for money, like \emph{wheat, sorrel, flouze} or \emph{fric}.
The deserved success of A. \familyname{Simonin}'s novel in 1953 gave a new lease of life to the word, which nevertheless does not seem to be truly integrated into the series of designations of money, such as \emph{wheat, sorrel, flouze} or \emph{fric}.

Rouchi \emph{griset}, ``a six-liard piece'' (1834), so called because of its color. But the only explanation currently available given by \familyname{Esnault},  is not satisfactory; on the one hand, the element \emph{bi} remains unexplained, if not by a ``suffix'' unknown; on the other hand, \familyname{Bruant} writes \emph{grisbis}, and it is possible (if not probable) that the central \emph{s} pronounced only since 1953; which would lead to an explanation: \emph{gris-bis}, in the series of alternative words for bread, \emph{wheat, carmine, biscuit, pancake}, etc.

Finally, if the \lang{metonymy}\footnote{\urlMetonymyDef{}} of colour is actually used to denominate money, it is always a precise category of money: ``cash''. Thus \emph{jaunet, white, white, copper}, are not interchangeable nor usable for ``money'' abstract.

%We will also recall the meaning of \emph{grey}: ``expensive'' (V. \emph{grisol}) and the possibility of the pseudo-suffix augmentative \emph{bi}, ``very'' , even rare. We would then have: \emph{grey-bi}, ``very expensive''? But the hypothesis is speculative.

We would also point out the meaning of \emph{gris}: ``expensive'' (V. \emph{grisol}) and the possibility of the augmentative pseudo-suffix \emph{bi}, ``très'', even rare. We would then have: \emph{gris-bi}, ``very expensive''? But the hypothesis is adventurous.

%TODO: the suite

\section{Bibliography\label{preamble-biblio}}

\emph{Do not touch the Grisbi!} by Albert \familyname{Simonin}

\begin{itemize}
\item Gallimard, Folio Policier No. 183, first edition in 1953, reissued in 2014 (always edited)
\item Gallimard, Folio No. 2068, first edition in 1989
\item Gallimard, Carré Noir No. 94, first edition in 1972
\item Le Livre de Poche No. 1152, first edition in 1953
\item Gallimard, Série Noire No. 148, first edition in 1953

\end{itemize}

\section{Filmography\label{preamble-filmography}}

\strong{Do not touch the grisbi!} (French for \strong{Don't touch the loot}), released as \strong{Honour Among Thieves} in the United Kingdom and \strong{Grisbi} in the United States

French-Italian film (1954). Gangster-film genre. Duration: 1h 34 min

Original title: Grisbi

Distribution:

\begin{itemize}
    \item Jean \familyname{Gabin}: Max the liar
    \item Rene \familyname{Dary}: Riton
    \item Dora \familyname{Doll}: Lola
    \item Vittorio \familyname{Sanipoli}: Ramon
    \item Marilyn \familyname{Buferd}: Betty
    \item Gaby \familyname{Basset}: Marinette
    \item Paul \familyname{Barge}: Eugene
\end{itemize}

Director: Jacques \familyname{Becker}

\subsection*{Synopsis}

Max-the-liar and Riton have just pulled off the greatest heist of their lives: stealing 50 million francs worth of gold bars at Orly. With this ``grisbi'', both gangsters expect to enjoy a peaceful retirement. But Riton can not resist telling his mistress Josy about money. The younger burlesque-dancer girlfriend passes the valuable information to Angelo, a drug dealer with whom she is cheating on Riton with. Angelo kidnaps the old mobster and demands ``grisbi'' from Max as ransom \dots{}

\subsection*{Trivia} 

\paragraph{A well-oiled tandem:} Jean \familyname{Gabin} and René \familyname{Dary} are considered two sacred figures of cinema before the war.
\paragraph{\familyname{Becker} father and son:} Jacques' son \familyname{Becker}, Jean, makes his film debut here as an assistant director. He is only fifteen years old!
\paragraph{Albert \familyname{Simonin}:} Writer and screenplay by Albert \familyname{Simonin}, who here adapts  his own novel, will make four more movies with \familyname{Gabin}, all dialogues by \familyname{Audiard}: \emph{Le cave se rebiffe} (1961) and \emph{The gentleman of Epsom} (1962) by Gilles \familyname{Grangier}, \emph{Mélodie en sous-sol} (1963) by Henri \familyname{Verneuil} and \emph{The Pasha} (1967) by Georges \familyname{Lautner}. After adapting his \emph{Les Tontons flingueurs} for Georges \familyname{Lautner} (1963), he became his screenwriter for \emph{Les Barbouzes} (1964).

\subsection*{Zone 2 DVD Version}

\begin{itemize}
    \item Interactivity: Home menu, access to scenes, filmographies drop-downs from the director, from Lino \familyname{Ventura} and John \familyname Gabin{}
    \item Cinema format: full screen
    \item Sound version: VF in mono
    \item Subtitles: none
    \item France --- 1953 --- Black \& white
    \item Length: 92 min --- 1 ~ disc --- 1 ~ side --- 1 ~ layer
    \item Release date: September 19, 2001
    \item Publisher: Studio Canal
\end{itemize}

\subsection*{Version Blu-ray zone 2}

\begin{itemize}
	\item Cinema format: 4/3 respected format 1.33
	\item Sound version: VF in stéréo
	\item France --- 1953 --- Black \& white --- 94 min
	\item Subtitles: none
	\item Release date: March 10, 2017
	\item Publisher: Studio Canal
\end{itemize}

\subsection*{Versions DVD / Blu-ray zone 1}

\begin{itemize}
	\item Cinema format: 4/3 respected format 1.33
	\item Sound version: VF in stéréo
	\item France --- 1953 --- Black \& white --- 94 min
	\item Subtitles: English
	\item Release date: 21 Aug. 2017
	\item Publisher: Studio Canal
\end{itemize}

The zone 1 versions have English subtitles and include interviews with filmmaker Jean \familyname{Becker} and actress Jeanne \familyname{Moreau}.

% deleted because it does not work in pdf and creates a problem in html
% \End{small}
