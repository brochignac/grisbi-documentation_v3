%%%%%%%%%%%%%%%%%%%%%%%%%%%%%%%%%%%%%%%%%%%%%%%%%% %%%%%%%%%%%%%%%
% Contents: The preamble chapter
% $ Id: gray-manual-preamble.tex, v 0.4 2002/10/27 Daniel Cartron
% $ Id: gray-manual-preamble.tex, v 0.6.0 2011/11/17 Jean-Luc Duflot (no change)
% $ Id: gray-manual-preamble.tex, v 0.8.9 2012/04/27 Jean-Luc Duflot (typo changes)
% $ Id: gray-manual-preamble.tex, v 1.0 2014/02/12 Jean-Luc Duflot
%%%%%%%%%%%%%%%%%%%%%%%%%%%%%%%%%%%%%%%%%%%%%%%%%% %%%%%%%%%%%%%%%
% deleted because it does not work in pdf and creates a problem in html
% \Begin{small}
% star command: deleted star for index entries with correct link display in html
\chapter{Präambel\label{preamble}}

%% commented for index entries with correct link display in html
%% see also the preamble chapter moved in \mainmatter in manuel.tex
% \markboth{}{Preamble}
% \ifpdf
% \addcontentsline{toc}{chapter}{Preamble}
% \else
% \fi


Die Handbuchversion 3.0 entspricht der Softwareversion 3.0.\\

Die deutsche Version ist eine Kombination aus der deutschen Version 2.0, die von Martin Stromberger im Jahr 2020 erstellt wurde, und der Übersetzung der französischen/englischen Version 3.0.

Mehrere Benutzer von \gls{Grisbi} baten Daniel \familyname{Cartron}, den Verfasser der ersten Version des Handbuchs, eine kurze Erinnerung an die Bedeutung dieses Wortes in das Handbuch einzufügen, da es zu seinem Leidwesen wieder in Vergessenheit geraten ist.

Nachdem seine ersten (kurzen) Recherchen keine Ergebnisse erbracht hatten, die es wert gewesen wären, veröffentlicht zu werden, hatte er es aufgegeben, bis er sich eines Tages die Zeit nahm, einige Zeit in einer Bibliothek zu verbringen, die mit Wörterbüchern aller Art gut bestückt war. Und dort war die Ernte reichlich. So reichlich, dass Daniel \familyname{Cartron} lange zögerte, was er behalten und was er aussortieren sollte\dots{}

Schließlich entschied er sich dafür, alles zu behalten. In der französischen Version finden Sie übrigens (viel) mehr Definitionen.

Und er hat auch eine Passage über diesen berühmten französischen Film hinzugefügt, da es für diejenigen, die noch wissen, was \indexword{grisbi}\index{Grisbi} bedeutet, hauptsächlich um den verdienten Ruhm dieses Werks geht.


\section{Etymologie\label{preamble-etymology}}

Hier sind also einige Quellen zur \index{Etymologie} des Wortes Grisbi:



\subsection*{Wörterbuch der deutschen Sprache}
[ grizbi ] n. Herr Arg. Geld von grauen (graue Währung; cd. rouchi Grauhai [ 1834 ], "Schwarzpappel") 1895, verbreitet im Jahre 1953

\subsection*{Grand Larousse der französischen Sprache}
[ grizbi ] n. Herr (von gris[ et ], Stück von sechs Schwarzpappeln [ 1834, Esnault ] der von grauen wegen der grisette Farbe [ vgl. auch, "Währung" XVIIe s. und weiße und graue Währung, 1784, Esnault ] mit dem suff.arg. - Bi; 1896, Delesalle)
Wenn es Nacht wird in Paris (Touchez pas au Grisbi), Titel eines Romans von Albert Simonin [ 1953 ]

\subsection*{Historisches Wörterbuch der deutschen Sprache}
n.m. erschienen im Jahr 1895 (grisbis), und ab 1953 durch den Roman von A. Simonin verbreitet, aus grau "grauer Währung". "Geld" in Argot könnte eine tautologische Zusammensetzung von grau und bis sein

\subsection*{Wörterbuch der Umgangssprache Französisch und von seinen Anfängen}
Sehr umstrittener Ursprung: das heißt von Grauhai, "Münze" und eines mysteriösen Suffixes - Bi oder das sowohl graue Brot als auch bis oder der Slang für "Englisches Crispy - Geld". Touchez pas au Grisbi, der der berühmte Roman von A. Simonin

\subsection*{Schätze der französischen Sprache}
Prononc.: [ grizbi ]. Etymol. und Hist. 1896 grisbis arg. "Geld" (Delesalle, aus dem Rad zusammengesetztes Dict. arg. fr. und fr.-arg. Mot. von der Grauhai im Sinne von "Stück von sechs Schwarzpappeln" (1834 ds Esn.). von grauen wegen der Farbe (vgl. auch CA 1634 grisette "Währung" Muse normannische ed D. Ferrand. A. Heron, II 91; 1784, Brest, weiße Währung und graues ds Esn.) und von einem zweiten Teil von orig. obsc., der vielleicht das suff. pop. darstellt - Bi, nerbi näher zu bringen "sehr schwarz" (d' apr. Esn.). Es ist nicht unmöglich, dass grisbi (früher grisbis), eine tautologische Zusammensetzung von grauen und von bis ist

\subsection*{Grand Robert de la langue française}
[ grizbi ] n. m. 1895: verbreitet 1953 durch den Roman von Simonin betreffen das grisbi; das Wort war ein seltenes oder überlegtes v. 1950: von grau "grauer Währung" (vgl. rouchi Grauhai "Schwarzpappel" 1834) und suff. pop. Argot. Geld


\section{Bibliografie\label{preamble-biblio}}

%\emph{Do not touch the Grisbi!} by Albert \familyname{Simonin}
\emph{Touchez pas au Grisbi!} von Albert \familyname{Simonin}

\begin{itemize}
%\item Gallimard, Folio Policier No. 183, first edition in 1953, reissued in 2014 (always edited)
%\item Gallimard, Folio No. 2068, first edition in 1989
%\item Gallimard, Carré Noir No. 94, first edition in 1972
%\item Le Livre de Poche No. 1152, first edition in 1953
%\item Gallimard, Série Noire No. 148, first edition in 1953	
\item Gallimard, Folio Policier Nr. 183, Erstausgabe 1953, Neuauflage 2014 (immer bearbeitet)
\item Gallimard, Folio Nr. 2068, in der ersten Ausgabe 1989
\item Gallimard, Carré Noir Nr. 94, in der ersten Ausgabe 1972
\item Taschenbuch Nr. 1152, in der ersten Ausgabe 1964
\item Gallimard, Schwarze Serie Nr. 148, in der ersten Ausgabe 1953
\end{itemize}




%\begin{itemize}
%\item Gallimard, Folio Policier No. 183, first edition in 1953, reissued in 2014 (always edited)
%\item Gallimard, Folio No. 2068, first edition in 1989
%\item Gallimard, Carré Noir No. 94, first edition in 1972
%\item Le Livre de Poche No. 1152, first edition in 1953
%\item Gallimard, Série Noire No. 148, first edition in 1953

\end{itemize}

\section{Filmography\label{preamble-filmography}}

\strong{Do not touch the grisbi!} (French for \strong{Don't touch the loot}), released as \strong{Honour Among Thieves} in the United Kingdom and \strong{Grisbi} in the United States

French-Italian film (1954). Gangster-film genre. Duration: 1h 34 min

Original title: Grisbi

Distribution:

\begin{itemize}
    \item Jean \familyname{Gabin}: Max the liar
    \item Rene \familyname{Dary}: Riton
    \item Dora \familyname{Doll}: Lola
    \item Vittorio \familyname{Sanipoli}: Ramon
    \item Marilyn \familyname{Buferd}: Betty
    \item Gaby \familyname{Basset}: Marinette
    \item Paul \familyname{Barge}: Eugene
\end{itemize}

Director: Jacques \familyname{Becker}

\subsection*{Synopsis}

Max-the-liar and Riton have just pulled off the greatest heist of their lives: stealing 50 million francs worth of gold bars at Orly. With this ``grisbi'', both gangsters expect to enjoy a peaceful retirement. But Riton can not resist telling his mistress Josy about money. The younger burlesque-dancer girlfriend passes the valuable information to Angelo, a drug dealer with whom she is cheating on Riton with. Angelo kidnaps the old mobster and demands ``grisbi'' from Max as ransom \dots{}

\subsection*{Trivia} 

\paragraph{A well-oiled tandem:} Jean \familyname{Gabin} and René \familyname{Dary} are considered two sacred figures of cinema before the war.
\paragraph{\familyname{Becker} father and son:} Jacques' son \familyname{Becker}, Jean, makes his film debut here as an assistant director. He is only fifteen years old!
\paragraph{Albert \familyname{Simonin}:} Writer and screenplay by Albert \familyname{Simonin}, who here adapts  his own novel, will make four more movies with \familyname{Gabin}, all dialogues by \familyname{Audiard}: \emph{Le cave se rebiffe} (1961) and \emph{The gentleman of Epsom} (1962) by Gilles \familyname{Grangier}, \emph{Mélodie en sous-sol} (1963) by Henri \familyname{Verneuil} and \emph{The Pasha} (1967) by Georges \familyname{Lautner}. After adapting his \emph{Les Tontons flingueurs} for Georges \familyname{Lautner} (1963), he became his screenwriter for \emph{Les Barbouzes} (1964).

\subsection*{Zone 2 DVD Version}

\begin{itemize}
    \item Interactivity: Home menu, access to scenes, filmographies drop-downs from the director, from Lino \familyname{Ventura} and John \familyname Gabin{}
    \item Cinema format: full screen
    \item Sound version: VF in mono
    \item Subtitles: none
    \item France --- 1953 --- Black \& white
    \item Length: 92 min --- 1 ~ disc --- 1 ~ side --- 1 ~ layer
    \item Release date: September 19, 2001
    \item Publisher: Studio Canal
\end{itemize}

\subsection*{Version Blu-ray zone 2}

\begin{itemize}
	\item Cinema format: 4/3 respected format 1.33
	\item Sound version: VF in stéréo
	\item France --- 1953 --- Black \& white --- 94 min
	\item Subtitles: none
	\item Release date: March 10, 2017
	\item Publisher: Studio Canal
\end{itemize}

\subsection*{Versions DVD / Blu-ray zone 1}

\begin{itemize}
	\item Cinema format: 4/3 respected format 1.33
	\item Sound version: VF in stéréo
	\item France --- 1953 --- Black \& white --- 94 min
	\item Subtitles: English
	\item Release date: 21 Aug. 2017
	\item Publisher: Studio Canal
\end{itemize}

The zone 1 versions have English subtitles and include interviews with filmmaker Jean \familyname{Becker} and actress Jeanne \familyname{Moreau}.

% deleted because it does not work in pdf and creates a problem in html
% \End{small}
