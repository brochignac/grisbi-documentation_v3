%%%%%%%%%%%%%%%%%%%%%%%%%%%%%%%%%%%%%%%%%%%%%%%%%% %%%%%%%%%%%%%%%
% Contents: The preamble chapter
% $ Id: gray-manual-preamble.tex, v 0.4 2002/10/27 Daniel Cartron
% $ Id: gray-manual-preamble.tex, v 0.6.0 2011/11/17 Jean-Luc Duflot (no change)
% $ Id: gray-manual-preamble.tex, v 0.8.9 2012/04/27 Jean-Luc Duflot (typo changes)
% $ Id: gray-manual-preamble.tex, v 1.0 2014/02/12 Jean-Luc Duflot
%%%%%%%%%%%%%%%%%%%%%%%%%%%%%%%%%%%%%%%%%%%%%%%%%% %%%%%%%%%%%%%%%
% deleted because it does not work in pdf and creates a problem in html
% \Begin{small}
% star command: deleted star for index entries with correct link display in html
\chapter{Präambel\label{preamble}}

%% commented for index entries with correct link display in html
%% see also the preamble chapter moved in \mainmatter in manuel.tex
% \markboth{}{Preamble}
% \ifpdf
% \addcontentsline{toc}{chapter}{Preamble}
% \else
% \fi


Die Handbuchversion 3.0 entspricht der Softwareversion 3.0.\\

Die deutsche Version ist eine Kombination aus der deutschen Version 2.0, die von Martin Stromberger im Jahr 2020 erstellt wurde, und der Übersetzung der französischen/englischen Version 3.0.

Mehrere Benutzer von \gls{Grisbi} baten Daniel \familyname{Cartron}, den Verfasser der ersten Version des Handbuchs, eine kurze Erinnerung an die Bedeutung dieses Wortes in das Handbuch einzufügen, da es zu seinem Leidwesen wieder in Vergessenheit geraten ist.

Nachdem seine ersten (kurzen) Recherchen keine Ergebnisse erbracht hatten, die es wert gewesen wären, veröffentlicht zu werden, hatte er es aufgegeben, bis er sich eines Tages die Zeit nahm, einige Zeit in einer Bibliothek zu verbringen, die mit Wörterbüchern aller Art gut bestückt war. Und dort war die Ernte reichlich. So reichlich, dass Daniel \familyname{Cartron} lange zögerte, was er behalten und was er aussortieren sollte\dots{}

Schließlich entschied er sich dafür, alles zu behalten. In der französischen Version finden Sie übrigens (viel) mehr Definitionen.

Und er hat auch eine Passage über diesen berühmten französischen Film hinzugefügt, da es für diejenigen, die noch wissen, was \indexword{grisbi}\index{Grisbi} bedeutet, hauptsächlich um den verdienten Ruhm dieses Werks geht.


\section{Etymologie\label{preamble-etymology}}

Hier sind also einige Quellen zur \index{Etymologie} des Wortes Grisbi:

\subsection*{Wörterbuch der deutschen Sprache}
[grizbi] n. Herr Arg. Geld von \emph{grauen} (graue Währung; cd. rouchi Grauhai [1834], \glqq{}Schwarzpappel\glqq{}) 1895, verbreitet im Jahre 1953

\subsection*{Grand Larousse der französischen Sprache}
[grizbi] n. Herr (von \emph{gris} [et], Stück von sechs Schwarzpappeln [1834,\familyname{Esnault}] der von \emph{grauen} wegen der \emph{grisette} Farbe [vgl. auch, \glqq{}Währung\grqq{} XVIIe s. und \emph{weiße und graue Währung}, 1784, \familyname{Esnault}] mit dem suff.arg. - Bi; 1896, \familyname{Delesalle})
\emph{Wenn es Nacht wird in Paris} (\emph{Touchez pas au Grisbi}), Titel eines Romans von Albert \familyname{Simonin} [1953]

\subsection*{Historisches Wörterbuch der deutschen Sprache}
n.m. erschienen im Jahr 1895 (grisbis), und ab 1953 durch den Roman von A. \familyname{Simonin} verbreitet, aus \emph{grau} \glqq{}grauer Währung\grqq{}. \glqq{}Geld\grqq{} in Argot könnte eine tautologische Zusammensetzung von \emph{grau} und \emph{bis} sein

\subsection*{Wörterbuch der Umgangssprache Französisch und von seinen Anfängen}
Sehr umstrittener Ursprung: das heißt von Grauhai, \glqq{}Münze\grqq{} und eines mysteriösen Suffixes - Bi oder das sowohl \emph{graue} Brot als auch \emph{bis} oder der Slang für \glqq{}\emph{Englisches Crispy} - Geld\grqq{}. \emph{Touchez pas au Grisbi}, der der berühmte Roman von A. \familyname{Simonin}

\subsection*{Schätze der französischen Sprache}
Prononc.: [grizbi]. Etymol. und Hist. 1896 grisbis arg. \glqq{}Geld\grqq{} (\familyname{Delesalle}, aus dem Rad zusammengesetztes Dict. arg. fr. und fr.-arg. Mot. von der Grauhai im Sinne von \glqq{}Stück von sechs Schwarzpappeln\grqq{} (1834 ds Esn.). von grauen wegen der Farbe (vgl. auch CA 1634 \emph{grisette} \glqq{}Währung\grqq{} Muse normannische ed D. \familyname{Ferrand}. A. \familyname{Heron}, II 91; 1784, Brest, weiße Währung und graues ds Esn.) und von einem zweiten Teil von orig. obsc., der vielleicht das suff. pop. darstellt - Bi, nerbi näher zu bringen \glqq{}sehr schwarz\grqq{} (d' apr. Esn.). Es ist nicht unmöglich, dass grisbi (früher grisbis), eine tautologische Zusammensetzung von \emph{grauen} und von \emph{bis} ist

\subsection*{Grand Robert de la langue française}
[grizbi] n. m. 1895: verbreitet 1953 durch den Roman von \familyname{Simonin} betreffen das grisbi; das Wort war ein seltenes oder überlegtes v. 1950: von \emph{grau} \glqq{}grauer Währung\grqq{} (vgl. rouchi Grauhai \glqq{}Schwarzpappel\grqq{} 1834) und suff. pop. Argot. Geld


\section{Bibliografie\label{preamble-biblio}}

%\emph{Do not touch the Grisbi!} by Albert \familyname{Simonin}
\emph{Touchez pas au Grisbi!} von Albert \familyname{Simonin} veröffentlicht 1953

\begin{itemize}\itemsep=-3pt
%	\setlength{\itemsep}{0pt}
	\item Gallimard, Kollektion Schwarze Serie Nr. 148, in der ersten Ausgabe 1953
	\item \emph{Wenn es Nacht wird in Paris} Friedrich \familyname{Hagen} (Mitwirkende), Desch, erschienen 1958
	\item Taschenbuch Nr. 1152, in der ersten Ausgabe 1964
	\item Gallimard, Kollektion Carré Noir Nr. 94, in der ersten Ausgabe 1972
	\item Gallimard, Kollektion Folio Nr. 2068, in der ersten Ausgabe 1989
	\item Gallimard, Kollektion Folio Policier Nr. 183, Erstausgabe 1953, Neuauflage 2014 (immer bearbeitet)
\end{itemize}

\section{Filmografie\label{preamble-filmography}}

\strong{Touchez pas au grisbi} (Französisch für \strong{Wenn es Nacht wird in Paris})

%French-Italian film (1954). Gangster-film genre. Duration: 1h 34 min
Film Italienisch, Französisch (1954). Krimi. Dauer: 1:34
%Original title: Grisbi

Künstlerische Besetzung:

\begin{itemize}\itemsep=-3pt
    \item Jean \familyname{Gabin}: Max der Lügner
    \item Rene \familyname{Dary}: Riton
    \item Dora \familyname{Doll}: Lola
    \item Vittorio \familyname{Sanipoli}: Ramon
    \item Marilyn \familyname{Buferd}: Betty
    \item Gaby \familyname{Basset}: Marinette
    \item Paul \familyname{Barge}: Eugene
\end{itemize}

Filmemacher: Jacques \familyname{Becker}

\subsection*{Synopsis}

%Max-the-liar and Riton have just pulled off the greatest heist of their lives: stealing 50 million francs worth of gold bars at Orly. With this ``grisbi'', both gangsters expect to enjoy a peaceful retirement. But Riton can not resist telling his mistress Josy about money. The younger burlesque-dancer girlfriend passes the valuable information to Angelo, a drug dealer with whom she is cheating on Riton with. Angelo kidnaps the old mobster and demands ``grisbi'' from Max as ransom \dots{}

Max, der Lügner, und Riton haben gerade den Coup ihres Lebens gelandet: Sie haben in Orly Goldbarren im Wert von 50 Millionen Francs gestohlen. Mit diesem \glqq{}grisbi\grqq{} wollen die beiden Gangster einen ruhigen Ruhestand genießen. Doch Riton kann es nicht lassen, seiner Geliebten Josy von dem Geld zu erzählen. Die Trainerin gibt die wertvolle Information an Angelo weiter, einen Drogenhändler, mit dem sie Riton betrügt. Angelo entführt den alten Ganoven und verlangt von Max das \glqq{}grisbi\grqq{} als Lösegeld \dots{}

\subsection*{Anekdoten} 

\paragraph{Ein gut geöltes Tandem:} Jean \familyname{Gabin} und René \familyname{Dary} werden als zwei heilige Monster des Vorkriegsfilms angesehen.
\paragraph{\familyname{Becker} Vater und Sohn:} Jacques \familyname{Beckers} Sohn Jean gibt hier sein Filmdebüt als Regieassistent. Er ist jedoch erst fünfzehn Jahre alt!
\paragraph{Albert \familyname{Simonin}:} Der Schriftsteller und Drehbuchautor Albert \familyname{Simonin}, der hier seinen eigenen Roman verfilmt, wird vier weitere Filme mit \familyname{Gabin} drehen, die alle von \familyname{Audiard} dialogisiert werden: \emph{Le cave se rebiffe} (1961) und \emph{Le gentleman d'Epsom} (1962) von Gilles \familyname{Grangier}, \emph{Lautlos wie die Nacht} (Originaltitel: \emph{Mélodie en sous-sol}, franz. \glqq{}Melodie im Untergrund\grqq{}) (1963) von Henri \familyname{Verneuil} und \emph{Der Bulle} (Originaltitel: \emph{Le pacha}) (1967) von Georges \familyname{Lautner}. Nachdem er sein Werk \emph{Mein Onkel, der Gangster} (Originaltitel: \emph{Les Tontons flingueurs}) für Georges \familyname{Lautner} (1963) adaptiert hatte, wurde er dessen Drehbuchautor für \emph{Mordrezepte der Barbouzes} (Originaltitel: \emph{Les Barbouzes}) (1964).

\subsection*{DVD-Version Zone 2}

\begin{itemize}\itemsep=-3pt
%    \item Interactivity: Home menu, access to scenes, filmographies drop-downs from the director, from Lino \familyname{Ventura} and John \familyname Gabin{}
    \item Seitenverhältnis: 4:3 - 1.33:1
    \item Sprache: Deutsch (Dolby Digital Mono), Französisch (Dolby Digital Mono)
    \item Untertitel: Deutsch
    \item Frankreich --- 1954 --- Schwarz \& white
    \item Laufzeit: 1 Stunde und 32 Minuten
    \item Erscheinungstermin: 3. August 2017
    \item Studio: Studiocanal
\end{itemize}

\subsection*{Blu-ray Version Zone 2}

\begin{itemize}\itemsep=-3pt
	\item Seitenverhältnis: 4:3 - 1.33:1
	\item Sprache: Deutsch (Dolby Digital Mono), Französisch (Dolby Digital Mono)
	\item Untertitel: Deutsch
	\item Frankreich --- 1954 --- Schwarz \& white
	\item Laufzeit: 1 Stunde und 37 Minuten
	\item Erscheinungstermin: 3. August 2017
	\item Studio: Studiocanal
\end{itemize}

\subsection*{DVD / Blu-ray Zone 1 Versionen}

\begin{itemize}\itemsep=-3pt
	\item Cinema format: 4/3 respected format 1.33
	\item Sprache: Französisch (Dolby Digital Stéreo)
	\item France --- 1954 --- Black \& white --- 94 min
	\item Untertitel: Englisch
	\item Erscheinungstermin: 21. August 2017
	\item Studio: Studiocanal
\end{itemize}

%The zone 1 versions have English subtitles and include interviews with filmmaker Jean \familyname{Becker} and actress Jeanne \familyname{Moreau}.
Die Zone-1-Versionen haben englische Untertitel und enthalten Interviews mit dem Filmemacher Jean \familyname{Becker} und der Schauspielerin Jeanne \familyname{Moreau}.
% deleted because it does not work in pdf and creates a problem in html
% \End{small}
